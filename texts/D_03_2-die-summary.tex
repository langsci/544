The leaves of the tree contain a substance that is used as red colouring. Unfortunately, I was unable to identify these trees botanically. But \textcite{Fyfe:2011by} offers a good description of dyeing techniques in the Upper Sepik and provides identifications of some of the species used to dye string bag threads.

Whenever possible, materials were dyed with plants found in the forest. The dyeing of grass skirts took one day, including preparation and the drying process. Apparently, the skirts were left in the colouring solution for about six hours. The colour must have been non-fading.

Naturally, the women did not go into the deep bush alone, they accompanied their husbands. While the men went hunting, the women would search for the plants they needed.