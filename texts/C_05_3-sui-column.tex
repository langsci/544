\begin{Parallel}{0.47\textwidth}{0.47\textwidth}
    \ParallelLText{\textit{Yena duyo molo. Diri eweno ilo ako ikep roise. Duyo mapap bi lipel. Lipeliwepu, puenpowepu. Diri baka royenamo, ewe ki baka. Boyopuno diri miso biep lu. Inakap. Diri mar diri nomar. Due an kinika so mar. Ewe umul nek, ``Diri kopi suloimoina mari. Yala ko asa pi? Yala ko ba pi?'' Dob reyo, ``Oke yala suem.'' Due an bakapuno diri kep sui. Basuiko. Paliya.}}
    \ParallelRText{Some people set off to the bush. Two brothers went together with their wives. They were staying in the bush and looked for animals. They found many and cut up the meat of the killed animals. The older brother gave half of the meat to his younger brother, and he himself took the other half. Then the younger brother also shot a boar. The families stayed on in the bush. But the younger brother became sick. For five days already he was very sick. So the older brother thought: ``My brother is seriously ill. What am I going to do now? What should I do?'' He looked at him and thought: ``This one will die soon.'' After five more nights the younger brother dies. He died. He is dead.}
\end{Parallel}
    \medskip
\begin{Parallel}{0.47\textwidth}{0.47\textwidth}
    \ParallelLText{\textit{Ewe wo. Ewe dob pul seku wo mop. Yala ko asa pi? Yeni rileyo po. Yol apla rileyo po. Ako kep muelnoro: ``Uki depi yala nuko parno lole, parno panepue.'' Wono lolo, yolyo layo. Sû mappo. Sû kamappapno sû beri rap. Ako kep muelnoro: ``Nuko ukeli yilauyo.'' Diri sukei wonino: ``Ko ruri depi ako depi ko yilauyo ukeli. De boyo wulimonpep!'' Yilauyo paeau wo roise. ``De ba po? De aso?'' ``Diri kopi basuik. Ko ke layepaneko duyo. Du yipyo lili. Parno layo yolyo.''}}
    \ParallelRText{The older brother cried. His tears fell down, and he sighed: ``What am I going to do now?'' He built a platform up in a tree. He put a grid of planks up in the tree. Then he said to his brother's wife: ``Now let's wrap your husband in a bark mat and put him up with the mat around his body.'' They tied the mat with a rope and put the body on the grid. Then they lit a fire. After lighting it, the blazing fire was burning the body. He said to his younger brother's wife: ``We'll take ourselves to the village.'' To his brother's spirit he called out: ``I take your children and your wife back to the village. You follow me later!'' Crying they reached the village and were asked: ``What did you do? How are you?''  ``My younger brother has died. I left his body in the bush, near the bush house. I put it on a grid. We had wrapped a bark mat around it.''}
\end{Parallel}
    \medskip
\begin{Parallel}{0.47\textwidth}{0.47\textwidth}
    \ParallelLText{\textit{Ewe kep sû mappap. Sû solo mappeuli\-pop. Ewe sû mimappeke. Sosoli nakap. Diri kep biso baslaupoko. Klapno par nepeipana. Am ari, nikip. Midorilolo, wono layeko, midorilayeko. Mi yilauyo pulo. Nakero, due an dupua dor dupua. Makina amanikiipe. Mi boyo lo. Am due lakoipe. Klapno miriyeko. Par mi\-nepeipana. Diri kep banikiko. Kili solo ulap, puaku kili roise sappi kili roise. Krapiyapno ropyo niskûno, mel yilau\-yo. Yilauyo paeau. Yip bîyo wena. Laliyo\-we ki yo nuipno.}}
    \ParallelRText{The brother tended the fire. He was tending the fire for a long time. The older brother went off to light the fire anew when it had extinguished. Tending the fire kept him busy. The younger brother's corpse has already dried like meat. When the older brother had come there again one day, he took off the bark mat and looked: not yet.  The corpse was still smelling. He wrapped him with the rope again and put him up in the tree. He put him back on the lattice again. Then he returned to the village. He stays there for twenty days. Well, before that the corpse would still smell. Later he went again, yet he counted the days first. When he had come, he glanced at the corpse again. He took off the bark mat. His brother had smelled. But now there isn't any smell left. There were only the bones, together with the skull and with the jaw bone. Having gathered the bones, he put them in a basket and carried them to the village. He arrived in the village. He brought the bones into his house. There he hung them up over the place, where he slept.}
\end{Parallel}
    \medskip
\begin{Parallel}{0.47\textwidth}{0.47\textwidth}
    \ParallelLText{\textit{Imiyu pulupi rilina. Sukei kep yala ewe kep muelne: ``De due kunuem! Imiyu ere pulupi. De puanap! De nakap dob pep!'' Riyopuno kaepul kaepi uroyo ule, klokni piyo. Ani yala rino pue. Monomno le. Du mono leipe. Epueyo puine, epuemno le. Boyopuno uro kep dob riye: ``Uro kopi sepiye. Imiyu pulupi. Ko pauiyo laliye.'' Mono bayana lo. Mi epuemno lo yilauyo, paeau. Ako kep muelno: ``Uro kopi sepiyo. Ko ro riyo. Mi ko ro pulo. Ko epemna pulo. Bi ko lipelou, mi ko ar lu.''}}
    \ParallelRText{Sorcerers might come to disturb the older brother's family. Then the young\-er brother's spirit warns his older brother: ``You must not sleep! The sorcerers come here. Wake up, stay awake and watch out.'' So he put the kneecap into his netbag. He took only one bone. During the day he walks around with the netbag. He walks along a path. First he walks along the bush track. Then he branches off in the undergrowth and walks through the undergrowth. Later he looks at his netbag: ``My netbag is shaking. Sorcerers are coming. I'll hang it over my shoulder.'' He chose another route. He went through the undergrowth back to the village, arrived there and said to his wife: ``My netbag shook, I saw this. I came back here. I walked fast. A pig I sought in vain, again I did not shoot anything.''}
\end{Parallel}
    \medskip