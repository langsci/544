This story is about the loss of a knife that was returned by a friendly spirit. The spirit lives at the bottom of the river into which the knife had fallen. This knife was only borrowed and the bearer was afraid that he would not be able to return it to its owner. Both journeys of the man Wau, who was eventually lucky, ended successfully: his journey down to the bottom of the river and the journey up and back to the village and to the owner of the knife. Interestingly, there seems to exist a kind of parallel world at the bottom of the river where people live. They go to work and come back from work. These underwater creatures are potentially dangerous for the visitor Wau and he is advised to leave this place as quickly as possible. Margaret Osi did not comment on the nature of the underwater people.

Kilmeri does not have a term for `lending, borrowing'. Instead, something is given \textit{biapno} `for a while' and the duration of use has to be negotiated. However, there is a clear obligation to return the borrowed item.

The opening formula of the text is in Tok Pisin. There are also some sentences in Tok Pisin later on.