Before Usikul told his story in Kilmeri, he narrated the clan's history in Tok Pisin. The content of the Tok Pisin version differs slightly from the Kilmeri version. Usikul mentions names of people and places that are not mentioned in the Kilmeri version, and vice versa. Interestingly, the narrator speaks of two \textit{tokples}, or vernacular languages/words, that must have existed on the southern slopes of the Oenake Mountains when the Bu clan arrived (cf. 4th paragraph below). These vernacular languages are hinted at by the words \textit{ru} and \textit{bar}, which are repeated in a long parlando. The parlando mode, which is used for important names throughout the narrative, gives these words a similar significance. On another level, these words are proof that the Kilmeri people migrated to this area from elsewhere. See also \textref{sc0202} below, in which the ancestor Si is described as ``the appropriator of the land.''

The people who lived there prior to the arrival of the Kilmeri were Sko-speaking people (\cite[5]{Donohue2004hr}; \cite[3]{Corris2006gr}). In present-day Vanimo [Dusur], we find the word \textit{va-nu-(pa)} [person-which-(particle)] meaning `who' (\cite[100]{Ross1980fe}). The single words \textit{va} `person' and \textit{nu} `which' could be related to \textit{bar} and \textit{ru}, included by Usikul in his clan history. One could also think of the Barupu language, the easternmost language of the Sko family (\cite[2]{Corris2006gr}). The place Asue, now a hamlet, may have been the place where the women handed over their male babies to the four legendary women mentioned by Usikul in his Kilmeri version of the clan history.

The ancestral clan branched out into many lines and lineages. In Tok Pisin, the words \textit{bruk} `split' and \textit{kil} `line, mountain ridge' are used to refer to this.