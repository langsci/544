The following procedural texts are concerned with items of the material culture of the Kilmeri people. The production of sago and the manufacture of oil and brooms continue as described in Texts \ref{sc0504}, \ref{sc0506}, and \ref{sc0508}. However, there is another tradition that continues to this day, namely the manufacture of net bags, or bilums, as they are commonly known across Papua New Guinea. Many women wanted to make and sell me a net bag in order to receive some money. They almost competed to see who would come up with the most beautiful design. The only difference from the past is that nowadays the women use acrylic wool, which they buy in a supermarket in Vanimo. Traditionally, they dyed the natural bark fibers in a lye solution, the dye being extracted from the bark of certain trees (cf. \textref{sc0503}).

\begin{figure}
    \centering
    \includegraphics[width=0.9\linewidth]{figures/KiniReginaBilums_0007.jpg}
    \caption{Kini and Regina with almost finished bilums}
    \label{fig:bilum}
\end{figure}