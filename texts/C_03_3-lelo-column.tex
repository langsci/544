\begin{Parallel}{0.47\textwidth}{0.47\textwidth}
    \ParallelLText{\textit{Koyo dueyo iloi, epe ko ikap roise. Koyo due wosonip. Epe kopi sonip, ko eku nakap dob ponap. Nek kau ropyo lupi. Ropyo lupuwapo, yeiyo meli. Sepue royepana pekolyo. Sepue apulyo lelweip. Riyopuno saryo wapo. Due pul sepueyo kûno. Pu ipiyo nekyo sipepo. Sipako, due lulip. Kululilpno nek kesiyo\-wolo. Nek lupuapoko, meli\-pulo, yeiyo lulip. Due lulip. Due lulip. Ko nakap. Ko unakap. Ko nakoro, epe kopiro ko muelno: ``Ko el sui. Ko lelo piu lipeli riyepi.''}}
    \ParallelRText{My mother and I went to the sago swamp. We pounded sago pith together. My mother was pounding, while I was sitting down on a palm rib and watching her. There was plenty of sago pith which she shovelled into a basket. She shovelled pith into several baskets and carried it to the kneading trough for sago washing. She put the trough in a circle of wooden sticks. The trough was stably placed in the middle of it. Then she put sago pith into a palm rib container. The pith went down into the trough. She bucketed water and poured it on the sago pith. She poured it down and was washing the sago. Having washed for some time, the sago pith was used up. Then she shovelled more sago pith, brought it, and washed it in the trough. She was washing sago. She was washing sago incessantly, while I was sitting there. I was just sitting idly. At last I said to my mother: ``I feel hungry. I will seek geckos and frogs. I will look for some for food.''}
\end{Parallel}
    \medskip
\begin{Parallel}{0.47\textwidth}{0.47\textwidth}
    \ParallelLText{\textit{Ko lo, ko numomo ppuo. Ko piu riyepo lelo roise. Ko ppue dob seku: piu unake. Ko wiyo, ko pako yeloyo. Numomo kwe epi baka ko dob seku: lelo unake. Ko wiyo, ko pako yeloyo. Ko kûno yeloyo. Ko piu lelo piyowe, rupueno lolowe. Ko numomo ba ppuo. Numomo kwe dob seku: piu unake. Ko wiyo, yeloyo pako. Kwe epi baka dob seku: piu dupua ina\-kap. Ko wiyowe, ko yeloyo pakowe. Ko kûno yeloyo. Piu rondupua rokini rupueno lolo, meli yeiyo. ``Piu lelo de luwapo?'' ``Ko baluwapoko.''  Ko sû mappo\-ipe. Ko rupueno re, piyo sûyo. Sûyo rap. Ko puliyo sûyo. Ko piu rupue nepe. Epe kopi piu ko namo, dupua. Ko ike piu dupua leloyo. Koyo inoi.}}
    \ParallelRText{Off I went. I climbed a \textit{numomo}-sago palm and looked for frogs and geckos. I climbed up and looked down (into a palm rib): Here is a frog. I caught it and threw it on the ground. I looked also into the \textit{numomo}-palm rib on the other side of me: Here is a gecko. I caught it and threw it on the ground. I went down to the ground. I took the frog and the gecko and wrapped them in a leaf. I climbed another sago palm and looked down into the \textit{numomo}-palm rib: Here is a frog. I caught it and threw it on the ground. I looked also into the palm rib on the other side of me: Two frogs were sitting there. I caught them and threw them on the ground. Then I went down to the ground. I wrapped the three frogs in a leaf to carry them to the trough where my mother was. ``Did you catch frogs and geckos?''  ``I have caught some.'' I lit a fire first. Then I took them in the leaves to get done in the fire. They were roasting in the fire. I took the frogs off the fire and took away the leaves. I gave two frogs to my mother. I myself had two frogs and the gecko. We ate.}
\end{Parallel}
    \medskip
\begin{Parallel}{0.47\textwidth}{0.47\textwidth}
    \ParallelLText{\textit{Due bapusiyoko. Due dû ep sipamu. Sepue piyamu due dû roise. Ropyo pokûno. Due dû rop roise piapo, puaku\-yo wakayo puo. Ani kululipno yipyo iloi. Yilauyo paeau. Yipyo wolomno ppuo, due dû wapoyo wolo. Epe sû mappo, yaup yowo sûyo. Yaup mol. Epe kopi due piyo wîlyo, ya mappo. Ya kamappapno ya sui. Ya kusuipno, rupueyo supopo. Lelpanapno bese ipino si. Bese kisipno krapno, puliyo. Paepu si. Ipiyo sikûno, sûyo yowo. Paepu bese roise baroko. Puliyo, wîl royo, rupopo. Uke roinen. Uke ilo. Uke bailoko rapue kesiyo. Mapap.}}
    \ParallelRText{The sago is washed. My mother tipped out
    the rinse water. She took the trough together with the sago flour. Then she filled the flour into baskets and lifted the heavy baskets up on her head and shoulders. We walked back. Having washed sago all day we went to our house. We reached the village. My mother climbed the ladder and set the sago flour on the porch. She lit a fire and boiled water on the fire. The water boiled. My mother put the sago in a dish and stirred water in. Soon the sago is ready. When it was done, she cut it in pieces and put the portions on leaves. She set them aside. Now she cooked \textit{tulip}-vegetable in a pot. When the \textit{tulip} was done, she took it out. Then she cooked mushrooms. She put them in a pot and cooked them over the fire. Soon the mushrooms and the \textit{tulip}-vegetable are done. She took them out of the pot, put the plates on the floor, distributed the vegetables, and gave them to us. We ate. We finished the meal. All the food is eaten up, and we were sitting and resting.}
\end{Parallel}
    \medskip
\begin{Parallel}{0.47\textwidth}{0.47\textwidth}
    \ParallelLText{\textit{Bo kuru.}}
    \ParallelRText{End of the story.}
\end{Parallel}
    \medskip