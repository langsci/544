The text is Margaret Osi's personal recollection of the time and lifestyle before modern development came along with the Australian colonial administration and the Catholic mission. Margaret was about twenty years old at the time the mission was established. As a teenager, she would accompany her father to Vanimo and experienced the early days of the town. However, life in the bush was still very traditional back then. Margaret's generation is thus the last to have grown up on the firm ground of indigenous heritage.

The stories of \textit{Am} (\textref{sc0201}), \textit{Sakou} (\textref{sc0302}), and \textit{Haus Tambaran} all revolve around the availability of food. The heroes Am/Bu and Sakou are well known for providing food, and the stories are dedicated to their memory. The ceremonies that took place at the \textit{haus tambaran} likewise served the purpose of securing food. The staple food that is mentioned again and again is sago, together with game meat. Such food is the basic prerequisite for any further prosperity. Interestingly, rice is also mentioned as a foodstuff in the story of Am (Text \ref{sc0201}, Sequence \ref{ex:nrexpalime}).