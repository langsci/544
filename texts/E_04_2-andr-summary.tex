At my request, Andrew Wapi recounted his life from the beginning to the present day in 2000, during which time he lost three relatives: both his parents and his first wife. In 2001, his second son Vincent died. When naming his children, Andrew Wapi first lists the indigenous name and then adds the Christian name. For him, the indigenous names are the real names, the others are just modern additions.

The following report of Andrew Wapi's life contains quite some Tok Pisin vocabulary. Some of this is found with other speakers, for example \textit{laip stori} `story of one's life', \textit{storimpi} `to tell', \textit{nem} `name', \textit{taim} `when', \textit{karimpi} `to beget'. Additionally, Andrew uses \textit{lukautimpi} `to take care of', \textit{pinispi} `to finish', and \textit{bilong mi} `of me'.

The word \textit{yilewi} `place' -- which I have only heard in this report -- is possibly related to I'saka \textit{i'} `village' and/or \textit{wéi} `house'. The emphatic suffix \textit{-ya} is also of I'saka origin (cf. \tabref{clannames} above). 

Also noteworthy is the frequent use of Kilmeri \textit{boyo} `later'. Here it functions almost as a coordinator for `and then'. Note also the metaphorical use of \textit{pue} `go' in Sequence \ref{ex:nrexkrisma}, which is used to indicate the passing of time (cf. \cite[950-953]{Gerstner-Link:2018un}).