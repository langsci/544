A woman and her husband were staying in a bush camp when she was seduced by a bush spirit. She gave birth to twins, a human-like child and a snake-like child. In the early years of their childhood, the two unequal brothers lived together peacefully. As young adults, however, the snake-like brother began to behave like a real bush spirit. When they both went hunting, it killed its human brother. The clueless parents find the snake at home and beat it to death. In the end, the parents lost both of their sons.

The story may convey the following message: A husband should always keep an eye on his wife. If she gets pregnant by some other man, the sons will fight over their rank. Murder and revenge may ruin the family.

There is repeated code switching to Tok Pisin, for example in Sequences \ref{ex:nrexloples}, \ref{ex:nrexilaiki}, \ref{ex:nrexlaikig}, \ref{ex:nrexpasimt}, and \ref{ex:nrexlukaut}. Andrew Wapi's personal Kilmeri style makes very frequent use of the subordinating verb form for backgrounding (12 times) and tail-head linkage (10 times) for discourse coherence.