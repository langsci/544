Sago processing is one of the women's main occupations. They take their babies or toddlers with them when they spend a day in the sago swamp. Sometimes the children are carried in a big bilum while they work, sometimes they are placed on a \textit{pangal} platform to sleep. This is an opportunity for the spirits to invade the child's mind and disturb its rest. The child can no longer be soothed by breastfeeding. A magical ritual must be performed with a specific plant to drive the spirits away. The leaves and bark used for the ritual could not be identified. The bark, which is chewed and spat on the child, may contain a calming substance. 

One possible natural reason for the baby's restlessness is the omnipresent insects in the sago swamp and, to a lesser extent, in its own home.

\begin{figure}
    \centering
    \includegraphics[width=0.75\linewidth]{figures/12Pounding Sago_20250222_0022.jpg}
    \caption{Pounding sago}
    \label{fig:poundingsago}
\end{figure}