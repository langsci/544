This story depicts the encounter of a hunter with a huge lizard that is later interpreted as a bush spirit. The hunter who roamed the bush alone is killed by the lizard. The term \textit{urual} is glossed here as `lizard', because this is the translation that the people themselves gave. In order to understand the physical power of the creature it makes sense to search for its biological and scientific identity. The description in the story and further explanations by the story tellers, especially by Susan Bisam, led to the species \textit{Varanus salvadorii} (Ordo: Squamata; without rank: Toxicofera). In English it is called ``the Papuan monitor,'' which denotes a varanoid saurian endemic to New Guinea. The animal is more than 2.5 m long, and it weighs well over 6 kg. It has extremely sharp and long teeth. Its limbs and claws are very strong and look quite big compared to the body.

All these features were known to Susan Bisam, in particular the strength of its limbs and the dangerous teeth. There is no doubt that when Susan was growing up, the hunters of the time had to be wary of these animals. The habitat of \textit{Varanus salvadorii} is rainforests and swamps between the coast and the interior at altitudes of up to 600 metres. It lives in trees and jumps from branch to branch, using its long tail for balance. It feeds on small mammals, birds and eggs, but also on insects -- in swampy areas it may even feed on sago grubs. Humans are therefore not prey for these monitor lizards, but they can fall victim to their aggressiveness when competing with them for prey animals such as possums or tree kangaroos. Some species of the Varanidae family are toxic and have venom glands. It seems possible that \textit{Varanus salvadorii} also has poisonous glands, as its bites are apparently fatal.

What is said above seems to be a reasonable background for the following story. Reference to the protagonist as a ``lizard'' occurs 10 times (Sequences \ref{ex:nrexpoehur}(3x), \ref{ex:nrexmaroip}, \ref{ex:nrexkpiowo}, \ref{ex:nrexolekiu}, \ref{ex:nrexpuawom}, \ref{ex:nrexwaplay}, \ref{ex:nrexwepusu}, \ref{ex:nrexlebaul}), whereas it is referred to as ``bush spirit'' 6 times (Sequences \ref{ex:nrexolekiu}, \ref{ex:nrexpuawom}, \ref{ex:nrexlebaul}(4x)). The creature displays almost supernatural powers, as it is able to split the ironwood tree in two, in which the older brother (i.e., the victim) was hiding. This is, of course, a narrative exaggeration. Nevertheless, the base of its tail can be enormously strong.
 
I may add a comment on Sequence \ref{ex:nrexomosop}, which sounds more like a raid than a revenge, as it lists both traditional and modern items taken from the bush spirit. In the past, clans have indeed raided other settlements out of revenge or simply out of need.