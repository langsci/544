The text provides the genealogy of the clan of Lis Osi, late husband of Margaret Osi (Sequences \ref{ex:nrexsiyelo}-\ref{ex:nreximopwe}). The social organisation of the Kilmeri is strictly patrilineal. Lis Osi is a member of the greater Imop clan that split into several subclans whose members now live in Ossima, Isi I, and Isi II, respectively. Lis Osi is called \textit{Imop eme}, which means `of Imop origin'. %The same is true for his six sons and their sons.

The second part of the text describes Si's land appropriation, starting with Sequence \ref{ex:nrexyaukil}. Some of the sections of bush claimed by him are named and explicitly listed here, and the names go back to people who died a long time ago. In Sequence \ref{ex:nrexaewibe}, the separation of the lineage between Isi Daru and Ossima is mentioned. The long journey of Si corresponds to the fact that the eastern villages, including Ossima, are the furthest away from the swampy basin around Bewani, which is where the first arrivals from the west came. This migration route of the Kilmeri can also be verified linguistically (\cite{Gerstner-Link:2023lo}). The journey itself is not mentioned; the main purpose of this narrative is to substantiate the land claims in the Puwani-Pual basin.

The importance of this kind of oral tradition is shown by the fact that Margaret Osi recited it to me twice, three years apart. Today, land claims are crucial for the distribution of royalties from logging.