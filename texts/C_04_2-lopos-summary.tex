The house that Margaret's father built is a family home. The events described took place around six decades ago, but the construction technique she describes is still used today. The only difference is the use of nails instead of ropes. My own house was also built the way Margaret describes. The supporting posts, which are anchored in the ground, are usually made of ironwood (\textit{Intsia bijuga}), a hard wood. The floor of the house is about 1 metre (or even higher) from the ground.

The use of six support posts results in a house that is about 6 metres long and 3 metres wide. That's a rather small floor plan. As we see in Sequence \ref{ex:nrexbaesli}, the fireplace was inside the house. Nowadays, people often have a separate kitchen house (\textit{haus kuk} in Tok Pisin) of about 6 square metres where all meals are prepared. Traditionally, the Kilmeri had single-family houses owned by one family. The communal houses included a house for adolescent men and the men's house for ritual purposes. The latter were absolutely taboo for women (see \textref{sc0204}). Each village had its own men's house (\textit{haus tambaran} in Tok Pisin), and the men would sleep and eat there (see Sequence (\ref{ex:nrexdupuni}) below). This was also the place for conversations and discussions. The spheres of life of men and women were thus relatively separate - a fact that Margaret Osi often repeats when she comments on the ``modern'' way of life. Above all, women did not have as many children as they do today. According to her, four children were normal compared to six to eight children today.  

Some remarks to Margaret Osi's narrative style: The work steps are usually described in two clauses. First the action clause appears in the punctual past tense, and then the result is given in a separate clause in the modality of resultative factuality. The descriptive sequence of action and result, repeated for each work step, depicts the process of building the house perfectly. Note also the list of words and phrases indicating daybreak in Sequence (\ref{ex:nrexdupuni}). This adds a poetic touch to the scenery and the narrative. The sunrise itself cannot be seen in the forest. But you can see the change from black to grey to light.

\begin{figure}
    \centering
    \includegraphics[width=0.75\linewidth]{figures/13Houses1_20250222_0004.jpg}
    \caption{Houses at Ossima Asples}
    \label{fig:housesossimaasples}
\end{figure}
