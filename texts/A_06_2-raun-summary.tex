The story is about how a clan got punished for neglecting two orphaned children. The orphaned children left their village, went into the bush and met their ``grandmother'' who dwells like a bush spirit at the bottom of a pond. The grandmother supported the children and she causes heavy rainfall through magic (Sequence \ref{ex:nrexsilpok}). The first effect of the rain was frogs and fish appearing in abundance, which the people collected. The second effect was a flood in which people drowned (Sequence \ref{ex:nrexinerna}). The remaining people were killed by a man named Buoko, whose identity is unclear. Two entire clans were wiped out (Sequence \ref{ex:nrexinuges}). A second man named Amou, whose pregnant wife died in the flood, takes revenge and kills Buoko. (Sequences \ref{ex:nrexolasau}/\ref{ex:nrexsepuep}). Amou was the last surviving member of the Inuges clan.

Originally the story was told by Sei Walup (see Section \ref{saiwalup}). For better understanding, I suggested that he should first narrate it in Tok Pisin and then repeat the story in Kilmeri. Unfortunately, the Kilmeri version is only an abridged version of the detailed Tok Pisin version and is barely comprehensible. In addition, the second version features so much code-switching, that one cannot speak of a Kilmeri text. In retrospect, it would have been better to arrange a second visit to him, focussing solely on Kilmeri.

There are some narrative differences between Sei's Tok Pisin version and Margaret's Kilmeri version. Firstly, Sei starts with an overview embedding the story into the clan context. In particular, he says that the story is not `his story', but he says in Tok Pisin: \textit{na stori bilong mi --- i no bilong mi, bilong olgeta man, man bilong mipela wanpela tokples i stap}, which means `and my story --- it isn't my story, it belongs to all people, the people of us are one language (community) for ever'. The ownership of the story is extended to the entire community of the fifteen Kilmeri-speaking villages, although it only reports on the fate of a single clan. This is an important indication that the people have a strong sense of shared identity and history. The story relates directly to the village and people of Isi. In fact, Sei mentions three place names: Ia, Isi, and Awol. Sei also confirms that the disputed body of water had another, older name: Kisi. In his introduction he says: \textit{Mi laik kirapim dispela stori nau. Nem bilong mi Sei Walup. Mi laik wokim mipela Awol stret. Mi wokim wara nau. Na Ppulae, wara Ppulae. Nem bilong em Kisi.} `I want to start with the story. My name is Sei Walup. I want to tell a story about us people in Awol. I'll tell about a small lake, about ``Bad water'' lake. Before it was called lake Kisi.' Only after the dramatic incident described in the story did the lake become known as \textit{ppulae} `bad'. 

Furthermore, Sei mentions the spirit woman who lives at the bottom of the original pond. She is called \textit{memi} `grandmother' by the children. This detail is relevant, because in Sei's version it is the husband Beko, not the spirit woman, who performs the rain magic. Thus, the husband is responsible for the death of the clans. An old Nimboran myth (\cite[31]{Kouwenhoven:1956mg}) tells of the female supernatural being \textit{Indjo}, who lives in the mountain rivers in the south of the Nimboran territory and whose power was feared. Should she become angry, she would send rain and thunder and drown the earth. Given the presumed homeland of the Kilmeri people, there may be a narrative link between the Nimboran oral tradition and this Kilmeri story: Apparently the villagers broke a social rule by neglecting orphaned children, and this was penalised by the spirit woman through her command over the rain and her ability to flood the earth. Margaret Osi preserved the female nature of the creator of the flood, while Sei Walup chose a male one. He also tells us that the two orphans drown in the rising waters. The role of Buoko in Margaret's version is not entirely clear. He may have killed some survivors of the flood or people from another village nearby. 

Finally, concluding the story Sei reinforces his reliability with the Tok Pisin words \textit{Olsem mi no giaman, trupela tok mi tok, \textbf{bo makina kopi muli}, tokples nau ... em tasol, sapos \textbf{app} i nogat lus em tasol, dispela stori em tasol} (Kilmeri phrases are in bold font), which translates as: `Thus I don't lie, I said the truth, I am speaking my words good, in (my) vernacular now ... that's it, if the sky is not lost, that's it, this story now ends'. This is much more than the usual formulae used at the end of stories. He emphasises the dramatic and perhaps even traumatic events.

The location of the small mountain or hill called Asaul in Sequence (\ref{ex:nrexolasau}) is unclear. However, the location of \textit{Pu ppulae}, the ‘bad water’, is known. It is a lake about 1 km long and 100-120 m wide west of the Puwani River and south-east of the settlements of Awol. Its natural origin and geological composition are unclear. While the course of the Puwani has changed to some extent over the decades -- people are aware of this -- the lake is unlikely to be an ancient branch of the river. The Google Earth view shows a blue coloured body of water, while the rivers and old river branches in the area are brownish in colour because of the sediments they carry. The lake can therefore be said to be a stagnant body of water with no obvious inflow or outflow. It is difficult to access as it is completely surrounded by reeds, and a swamp area extends about 300 metres to the east. The quality of the water is said to be poor, probably a little salty or alkaline, and it is compared to the salty sea, \textit{olsem solwara} in Tok Pisin.
 
\textit{Pu ppulae} is a taboo place. The inhabitants of Awol and Iulep, to whose land it belongs, do not catch fish in the lake and do not go there to bathe. People are allowed to visit the place, but may not touch it. The collective memory of the place where an ancestral clan perished is still vivid, and it is believed that it is inhabited by the spirits of the dead. White people are not normally allowed to enter the lake, but my family were escorted to the lake by the storyteller Sei Walup and two other men from the village.

People are still familiar with the ginger magic that can be used to bring about rain. On one occasion, Margaret Osi told me about a man from Isi camp who knows rain magic and who performed the ritual in very hot and dry weather.

\begin{figure}
    \centering
    \includegraphics[trim=0 20 0 0,clip,width=0.6\linewidth]{figures/03VillageAwol_20250222_0007.jpg}
    \caption{The village of Awol}
    \label{fig:awolvillage}
\end{figure}