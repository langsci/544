\ea\label{ex:nrexsiyelo}
Si yelo ki piyo. Si dere Waiapi Yauyo.\\
\gll Si	yelo	ki	piyo	Si	dere	Waia-pi	Yau-yo\\
     Si	ground	\textsc{aph}	take.\textsc{pp}	Si	ancestor	Waia-\textsc{poss}	Yau-\textsc{loc}\\
\glt `Si appropriated the land. Si is Waia's ancestor and Yau's.'\footnote{Si is the ancestor of David Osi and Paul Osi.}
\z

\ea\label{ex:nrexsibuna}
Si Bu Nakei Woreau Peiyo Sui Senyo Bewo Yaewiyo Wesei Eppi ... Eppi, Dipiai ... Dipiai.\\
\gll Si	Bu	Nakei	Woreau	Pei-yo	Sui	Sen-yo	Bewo	Yaewi-yo	Wesei	Eppi	Eppi	Dipiai	Dipiai\\
     Si	Bu	Nakei	Woreau	Pei-\textsc{loc}	Sui	Sen-\textsc{loc}	Bewo	Yaewi-\textsc{loc}	Wesei	Eppi	Eppi	Dipiai	Dipiai\\
\glt `Si, Bu (and) Nakei, Woreau and Pei, Sui and Sen, Bewo and Yaewi, Wesei Eppi ... Eppi, Dipiai ... Dipiai.'\footnote{Bu is a direct, first generation descendant of Si, and as such he is the ancestor of the people of the village of Omoi located on the northern banks of the Puwani river across Ossima Asples. See Text \ref{sc0201}, story of Am. The name \textit{Dipiai} is actually a place name.}
\z

\ea\label{ex:nrexokyawa}
Oki Yau Waia Bilou.\\
\gll Oki	Yau	Waia	Bilou\\
     Oki	Yau	Waia	Bilou\\
\glt `Oki, Yau, Waia, Bilou.'
\z

\ea\label{ex:nrexwaiasa}
Waia Saewi Bilouyo karimpowe, Bilou Saewiyo. Bilou Bawi karimpo. Lis Waia karimpo.\\
\gll Waia	Saewi	Bilou-yo	karim-po-we	Bilou	Saewiyo	Bilou	Bawi	karim-po	Lis	Waia	karim-po\\
     Waia	Saewi	Bilou-\textsc{loc}	beget(\textsc{tp})-\textsc{lv}.\textsc{pp}-\textsc{du}.\textsc{o}	Bilou	Saewi-\textsc{loc}	Bilou	Bawi	beget(\textsc{tp})-\textsc{lv}.\textsc{pp}	Lis	Waia	beget(\textsc{tp})-\textsc{lv}.\textsc{pp}\\
\glt `Waia begot Saewi and Bilou, Bilou and Saewi. Bilou begot Bawi, Lis begot Waia.'
\z

\ea\label{ex:nreximopwe}
Saewi Lis karimpo. Lis Waiayo Yauyo karimpowepu, Imopwe ... Imopwe.\\
\gll Saewi	Lis	karim-po	Lis	Waia-yo	Yau-yo	karim-pi-wepu	Imop-we	Imop-we\\
     Saewi	Lis	beget(\textsc{tp})-\textsc{lv}.\textsc{pp}	Lis	Waia-\textsc{loc}	Yau-\textsc{loc}	beget-\textsc{lv}-\textsc{quant}.\textsc{o}.\textsc{pp} Imop-\textsc{du}.\textsc{o}	Imop-\textsc{du}.\textsc{o}\\
\newpage
\glt `Saewi begot Lis, Lis begot Waia and begot Yau ... begot two Imops, two Imops.'\footnote{The sons of Lis, Waia and Yau, are both members of the Imop clan. Margaret wants to point at this by the repetition \textit{Imopwe Imopwe}. Actually, this is a special construction, since the verbal suffix \textit{-we} is cliticised to a noun, probably short for \textit{Imop karimpowe}.}
\z

\ea\label{ex:nrexyaukil}
Yau Kili neuli, Kili du roki, Imop du, Dipei du. Basupuliko. Kili mi roke mape, banapwepuko.\\
\gll Yau	Kili	ne-uli	Kili	du	ro-ki	Imop	du	Dipei	du	ba-supuli-ko	Kili	mi	ro-ke	mape	ba-napi-wepi-ko\\
     Yau	Kili	go.thither-\textsc{prog}	Kili	bush	\textsc{prox}-\textsc{aph}	Imop	bush	Dipei	bush	\textsc{fac}-die.\textsc{pl}-\textsc{fac}	Kili	then	\textsc{prox}-\textsc{aph}	live.\textsc{pl}	\textsc{fac}-come.inside.\textsc{pl}-\textsc{quant}.\textsc{s}-\textsc{fac}\\
\glt `Yau is naming Kili: the Kili bush is here, the Imop bush, the Dipei bush. (The persons whose names are given to sections of bush all) have died. Then the Kili (people) live here. They have come to the site.'
\z

\ea\label{ex:nrexsikedu}
Si ke du piyo. Si du piyoro, Oimu Iwanbi Lûli Saepuyo, Dupuopli. Lûli Saepuyo Dupuopli. Yelo roke piyo. Si inakei ruri kep.\\
\gll Si	ke	du	piyo	Si	du	piyo=ro	Oimu	Iwanbi	Lûli	Saepu-yo	Dupuopli	Lûli	Saepu-yo	Dupuopli	yelo	ro-ke	piyo	Si	i-nake-i	ruri	kep\\
     Si	\textsc{aph}	bush	take.\textsc{pp}	Si	bush	take.\textsc{pp}=\textsc{emph}	Oimu	Iwanbi	Lûli	Saepu-\textsc{loc}	Dupuopli	Lûli	Saepu-\textsc{loc}	Dupuopli	ground	\textsc{prox}-\textsc{aph}	take.\textsc{pp}	Si \textsc{du}.\textsc{s}-live-\textsc{du}.\textsc{s}	child	3\textsc{sg}.\textsc{poss}\\
\glt `Si took the bush. Si appropriated the forest: Oimu, Iwanbi, Lûli and Saepu, Dupuopli. Lûli and Saepu, Dupuopli. This land he took. Si lives (there with) his sons.'
\z

\ea\label{ex:nrexaewibe}
Yaewi Bewoyo. Yaewi ppue Isi Daruka, Bewo yo nake.\\
\gll Yaewi	Bewo-yo	Yaewi	ppue	Isi	Daru-ka	Bewo	yo	nake\\
     Yaewi	Bewo-\textsc{loc}	Yaewi	go.up	Isi	Daru-\textsc{path}	Bewo	location	stay\\
\glt `Yaewi and Bewo. Yaewi goes up towards Isi Daru, Bewo stays here.'
\z

\ea\label{ex:nrexleroki}
Si lero, Kili neuli Isi. Ko Si Ossimayo oki nake. Yelo ki piyo, yelo ki napo.\\
\gll Si	le=ro	Kili	ne-uli	Isi	ko	Si	Ossima-yo	o-ki	nake	yelo	ki	piyo	yelo	ki	napo\\
     Si	go=\textsc{emph}	Kili	go.thither-\textsc{prog}	Isi	ko	Si	Ossima-\textsc{loc}	\textsc{prox}-\textsc{aph}	live	ground	\textsc{aph}	take.\textsc{pp}	ground	\textsc{aph}	come.inside.\textsc{pl}.\textsc{pp}\\
\glt `Si walks and walks, he names Kili and Isi. I am Si, I live here in Ossima. He appropriated the land, he came to the site.'\footnote{The names Isi and Osi are quite probably derived from deictics and the ancestor's name Si: Isi literally means `there-Si' and Osi means `here-Si'. However, the original deictic centre cannot be reconstructed. Most probably, the names should reassure his vast appropriation of land.}
\z

\ea\label{ex:nrexopliye}
Oimu Bu nake. Iwanbi Lûli Saepuyo Dupuopli ikepro, du yelo.\\
\gll Oimu	Bu	nake	Iwanbi	Lûli	Saepu-yo	Dupuopli	ikep=ro	du	yelo\\
     Oimu	Bu	live	Iwanbi	Lûli	Saepu-\textsc{loc}	Dupuopli	\textsc{poss}.\textsc{emph}=\textsc{emph}	bush	ground\\
\glt `Oimu. Bu lives there. Iwanbi Lûli Saepuyo Dupuopli is his, the bush, the ground.'\footnote{The possessive pronoun `his' refers to the lineage of Nakei, the second son of Si, while Bu is Si's first son.}
\z

\ea\label{ex:nrexnapoou}
Yelo ki napo: Omupaek Ouwin Ye Win Duppua Diyewi Yewi Bua Omal Bal Oli Lil Oipol Ipol Akos Ono Pusuwei. Du roke Waiapi Yaupi Jeromepi. Onoro piyeno, yelo bo yala ino moliye.\\
\gll yelo	ki	napo	Omupaek	Ouwin	Ye	Win	Duppua	Diyewi	Yewi	Bua	Omal	Bal	Oli	Lil	Oipol	Ipol	Akos	Ono	Pusuwei	du	ro-ke	Waia-pi	Yau-pi	Jerome-pi	ono=ro	piye-no	yelo	bo	yala	i-no	moliye\\
     ground	\textsc{aph}	come.inside.\textsc{pl}.\textsc{o}.\textsc{pp}	Omupaek	Ouwin	Ye	Win	Duppua	Diyewi	Yewi	Bua	Omal	Bal	Oli	Lil	Oipol	Ipol	Akos	Ono	Pusuwei	bush	\textsc{prox}.\textsc{emph}-\textsc{aph}	Waia-\textsc{poss}	Yau-\textsc{poss}	Jerome-\textsc{poss}	man=\textsc{emph}	take-\textsc{co}	ground	word	now	\textsc{dist}-\textsc{ins}	say.\textsc{pl}\\
\glt `(Si) came to the land: Omupaek, Ouwin, Ye, Win, Duppua, Diyewi, Omal, Bal, Oli, Lil, Oipol, Ipol, Akos, Ono, Pusuwei. This bush is Waia's, Yau's, Jerome's. If someone takes the land, they are now reclaiming it with these (names).'\footnote{Traditionally the father allots the bush sections of the family to his sons in life, and, by doing so, avoids disagreements after his death. The name Ye actually refers to a person, as Margaret inserts in Tok Pisin. This insert is not transcribed.}
\z

\newpage
\ea\label{ex:nrexsilkum}
Silkum Waiapi Yaupi, Silkum. Mounten Mamal i stap antap.\\
\gll Silkum	Waia-pi	Yau-pi	Silkum	maunten	Mamal	i	stap	antap.\\
     Silkum	Waia-\textsc{poss}	Yau-\textsc{poss}	Silkum	mountain	Mamal	\textsc{pred}	be.there	upriver\\
\glt `Silkum belongs to Waia, belongs to Yau, Silkum. The mountain Mamal is upriver.'\footnote{This is the end of the Kilmeri text. Here Margaret switches to Tok Pisin and continues in Tok Pisin with the reconfirmation of the land claims of Lis Osi's sons towards any other claims by far relatives or newcomers.}
\z
