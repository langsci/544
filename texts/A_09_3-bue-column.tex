\begin{Parallel}{0.47\textwidth}{0.47\textwidth}
    \ParallelLText{\textit{Kimike kimike dukiro dete mumuna\-no kaikai puno solo siulipop. Bue aska, puno solo. Riyopuno mumuna kopi nem kep ke Wumeye woppuo pul ppuo. Welro du mono. Oyo nu Wupepp yilauyo. Nuknoko epul so malap, bue welipi. Woppuo pul kupurapopno bueno mapa. So nep oke bue. Riyopuno dori\-pulo yilauyo. Yena bo mosaupoen: ``Bue ko ika riyeko, bue paeau. Bue ko ba\-riye\-ko.'' Riyopuno yena muelien. Mono kirewoloro, bue paeau. Somapaip bue, ou. Riyopuno ul kusukelipno bue isaeaupo, yilauyo melpulup. Bue nem kep bue Wumeye.}}
    \ParallelRText{In really ancient times, our ancestors cooked their food with water only. There was no salt, only water. Then my ancestor -- his name is Wumeye -- roamed the forest and climbed a \textit{wo\-ppuo}-tree to collect fruits. Then he carried them along the bush track. He spent the night here, at the place called Wupepp. Walking further he slept for some days in the bush and heard the sound of the sea approaching. Having opened the \textit{wo\-ppuo}-fruits, he tasted them with salt. He was eating and thought: ``Well, this is the taste of salt.'' Now he returned to the village. There he informed the people: ``The sea, I saw it there. I reached the sea, I have seen the sea.'' Then he talked to the people, and afterwards they cut a bush track to reach the sea. They were tasting the salty water, yes. Then they cut bamboo containers, filled them with salt water and carried them to the village. The salt's name is Wumeye salt.}
\end{Parallel}
    \medskip
\begin{Parallel}{0.47\textwidth}{0.47\textwidth}
    \ParallelLText{\textit{Kuru bo kopi.}}
    \ParallelRText{My story is finished.}
\end{Parallel}
    \medskip
\begin{figure}
    \centering
    \includegraphics[width=0.6\linewidth]{figures/Omoi-Sea.png}
    \caption{Brigitte Esau's sketch}
    \label{fig:sketchomoi}
\end{figure}
    \medskip