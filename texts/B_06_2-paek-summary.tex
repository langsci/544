A father is accompanied by his son on a hunting trip into the bush. The son is told to stay near the bush hut while the father goes hunting alone. Left alone, the son is visited by a bush spirit. The spirit measures the beds to find out which is his father's bed. It also warns the child that it will come back the next night and eat his father. The son is suspicious of the visitor and tells his father everything. The father doesn't believe a word he says and just wants him to enjoy the delicious food. But the son is right, and that night the bush spirit returns and kills the father. The clan then takes revenge and kills the bush spirit.

The moral of the story is that the deep forest is always a potentially dangerous place. One must always exercise caution. Certainly it is not the right place to eat large meals and then rest in a sound sleep without hearing warning noises. This is exactly what happened to the father. When transcribing the text with Margaret Osi, she told me that the father's sleeping position indicates deep sleep (Sequence \ref{ex:t2exvonuro}). So the father did not demonstrate the kind of behaviour required to withstand the challenges of the bush.

Some remarks to style and grammar of the narrator: Although grammatically mandatory, the narrator rarely uses the affixal dual of the verb. Instead, he often uses the prefix of accompaniment with a similar function. He also makes frequent use of topicalised pronouns for emphasis, while emphatic pronouns are not used. Furthermore, the text shows code-switching to Tok Pisin.