Margaret herself entitled this account of her life with Tok Pisin \textit{Laip stori bilong man bilong mi taim em i dai} (`Life story of my husband when he died'). This suggests that the story is about her husband's last days. In fact, however, it centres on the difficulties she faced after his death. This recollection of Margaret's way of life after her husband's death ties in with her first short version above (\textref{sc0601}). This second part of her narrative is very repetitive and shows the great emotional and economic tumult of living as a widow.

The text exhibits some code-switching. The following nine words are all Tok Pisin: \textit{kaikai} `food', \textit{famili} `family', \textit{baimpi} `to buy', \textit{maket} `market', \textit{moni} `money', \textit{rais} `rice', \textit{sop} `soap', \textit{omo} `detergent' (a brand name for the type of product), \textit{painimpi} `to look for', \textit{na} `and'. In the text, Tok Pisin verbs alternate with their Kilmeri equivalents. The same goes for Tok Pisin \textit{kaikai} `food' likewise. There are four nouns which do not have Kilmeri equivalents: \textit{famili}, \textit{maket}, \textit{sop}, and \textit{omo} and should be considered true loanwords. The grammar in which these Tok Pisin words are embedded is strictly Kilmeri.