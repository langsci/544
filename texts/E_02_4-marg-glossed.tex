\ea\label{ex:nrexbosaup}
    Ko yala bo saupi.\\
    \gll ko	yala	bo	saupi\\
    I	now	story	know\\
    \glt `Now I will tell a story.'
\z

\ea\label{ex:nrexikopib}
    Uki kopi basuiko. Ko kama nake.\\
    \gll uki	ko-pi	ba-sui-ko	ko	kama	nake\\
    husband	1\textsc{sg}-\textsc{poss}	\textsc{fac}-die-\textsc{fac}	I	alone	sit\\
    \glt `My husband has died. I live alone.'
\z

\ea\label{ex:nrexokamak}
    Ko kama nake na yala ko kaikai an ikapno lipeli.\\
    \gll ko	kama	nake	na	yala	ko	kaikai	an	ikap-no	lipeli\\
    I	alone	sit	and(\textsc{tp})	now	I	food(\textsc{tp})	hand	1\textsc{sg}.\textsc{poss}.\textsc{emph}-\textsc{ins}	find\\
    \glt `I live alone and (need) to find food with my own hands.'
\z

\ea\label{ex:nrexkoanik}
    Ko an ikapno piye kaikai. Due soni. Ko dû wepule yipyo.\\
    \gll ko	an	ikap-no	piye	kaikai	due	soni	ko	dû	wepule	yip-yo\\
    I	hand	1\textsc{sg}.\textsc{poss}.\textsc{emph}-\textsc{ins}	take	food(\textsc{tp})	sago	pound.sago.pith	I	sago.flour	bring	house-\textsc{loc}\\
    \glt `With my own hands I get it, the food. I pound sago and bring the flour to (my) house.'
\z

\ea\label{ex:nrexpekoni}
    Ya mappe. Ko ni kaikai.\\
    \gll ya\_mappe	ko	ni	kaikai\\
    stir.sago	I	eat	food(\textsc{tp})\\
    \glt `I stir the sago. I eat the food.'\footnote{In this clause the word order is
AVO as in Tok Pisin.}
\z

\ea\label{ex:nrexwapifa}
    Rapue wapi famili kopi roise. Uke ni. Ko ukeni.\\
    \gll rapue	wapi	famili	ko-pi	roise	uke	ni	ko	uke-ni\\
    vegetal.food	collect	family(\textsc{tp})	1\textsc{sg}-\textsc{poss}	together	we.\textsc{excl}	eat	I	jointly-eat\\
    \glt `I collect vegetables (in the bush), together with my family. We eat. I eat in company.'
\z

\ea\label{ex:nrexkaioyo}
    Ko kaikai oyo wili maketyo wili. Ko moni piye.\\
    \gll ko	kaikai	o-yo	wili	maket-yo	wili	ko	moni	piye\\
    I	food(\textsc{tp})	\textsc{prox}-\textsc{loc}	carry	market-\textsc{loc}	carry	I	money(\textsc{tp})	take\\
    \glt `I carry food here. I carry it to the market. (So) I will get some money.'
\z

\ea\label{ex:nrexsopomo}
    Moni ko wepule. Ko kaikai baimpi, rais baimpi, wal baimpi sop omo.\\
    \gll moni	ko	wepule	ko	kaikai	baimpi	rais	baimpi	wal	baimpi	sop	omo\\
    money(\textsc{tp})	I	bring	I	food(\textsc{tp})	buy(\textsc{tp})	rice(\textsc{tp})	buy(\textsc{tp})	fish	buy(\textsc{tp})	soap	detergent\\
    \glt `I get the money and buy food. I buy rice, buy tinfish, soap and detergent.'
\z

\ea\label{ex:nrexlipule}
    Ko rapiye. Melipule le pusiyena rais wal nina.\\
    \gll ko	rapiye	melipule	le	pusiye-na	rais	wal	ni-na\\
    I	fetch	bring.\textsc{pl}.\textsc{o}	things	wash-\textsc{purp}	rice(\textsc{tp})	fish	eat-\textsc{purp}\\
    \glt `I get it. I will bring a lot (with me), for washing my cloths, for eating rice and fish.'
\z

\newpage
\ea\label{ex:nrexloklok}
    Yala ko kama. Uki kopi basuiko. Balok. Balok.\\
    \gll yala	ko	kama	uki	ko-pi	ba-sui-ko	ba-le-ko	ba-le-ko\\
    now	I	alone	husband	1\textsc{sg}-\textsc{poss}	\textsc{fac}-die-\textsc{fac}	\textsc{fac}-go-\textsc{fac}	\textsc{fac}-go-\textsc{fac}\\
    \glt `Now I am alone. My husband has died. He has gone. He has gone (for good).'
\z

\ea\label{ex:nrexkamkam}
    Yala ko kama nake. Ari. Ko so solo nake. Ko kama.\\
    \gll yala	ko	kama	nake	ari	ko	so	solo	nake	ko	kama\\
    now	I	alone	sit	no	I	like.this	only	sit	I	alone\\
    \glt `Now I live alone. No. I live only like this. I am alone.'\footnote{The sentential negation \textit{ari} `no' refers to Margaret's former way of life with her husband who died several years ago.
}
\z

\ea\label{ex:nrexpisiko}
    Uki kopi balok. Balopisiko.\\
    \gll uki	ko-pi	ba-le-ko	ba-le-pisi-ko\\
    husband	1\textsc{sg}-\textsc{poss}	\textsc{fac}-go-\textsc{fac}	\textsc{fac}-go-\textsc{cpl}-\textsc{fac}\\
    \glt `My husband has gone. He has gone for ever.'
\z

\ea\label{ex:nrexarpoww}
    Yala ko kama nake. Ruri kopi ari, ko ar powai.\\
    \gll yala	ko	kama	nake	ruri	ko-pi	ari	ko	ar	powai\\
    now	I	alone	sit	child	1\textsc{sg}-\textsc{poss}	no	I	\textsc{neg}	give.1\textsc{sg}.\textsc{or}\\
    \glt `Now I live alone. My children, no, they don't give me anything.'
\z

\ea\label{ex:nrexikapno}
    Ko kama so solo nake. Ko an ikapno painimpi moni. Ko an ikapno lipeli.\\
    \gll ko	kama	so	solo	nake	ko	an	ikap-no	painimpi	moni	ko	an	ikap-no	lipeli\\
    I	alone	like.this	only	sit	I	hand	1\textsc{sg}.\textsc{poss}.\textsc{emph}-\textsc{ins}	find(\textsc{tp})	money(\textsc{tp})	I	hand	1\textsc{sg}.\textsc{poss}.\textsc{emph}-\textsc{ins}	find\\
    \glt `I live alone like this. I look for money with my own hands.\footnote{In this clause, O follows V as in Tok Pisin.} With my own hands I (need) to earn it.'
\z

\ea\label{ex:nrexkapyon}
    Yip ko ikapyo nake. Ko so solo nake. Yip ba as.\\
    \gll yip	ko\_ikap-yo	nake	ko	so	solo	nake	yip	ba	as\\
    house	1\textsc{sg}.\textsc{poss}.\textsc{emph}-\textsc{loc}	sit	I	like.this	only	sit	house	other	none\\
    \glt `I live in my own house. I live only like this. There is no other house (for me to live in).'\footnote{Old parents usually prefer to live in the household of one of their children to secure their livelihood.}
\z

\ea\label{ex:nrexkurkur}
    Klokni solo ko riyo nake. Kuru. Bo kuru.\\
    \gll klokni	solo	ko	ri-yo	nake	kuru	bo	kuru\\
    one	only	I	\textsc{dist}-\textsc{loc}	sit	be.finished	story	be.finished\\
    \glt `One (person) only I live there, that's it. The story is finished.'\footnote{The concluding formula \textit{kuru} is partially integrated into Margaret's reasoning of being all alone. Her situation won't change anymore. Therefore it is taken here as part of the last sentence of the story.}
\z
