In her youth, Margaret Osi lived a traditional life. Making grass skirts was an important personal task for every woman and seemed to take up a lot of time. The skirts were worn day and night and did not last long. Apparently Margaret did not particularly enjoy the labour of making grass skirts. Only at the end of the report does she say which types of sago palms are most suitable and produce skirts that do not break so quickly. In other areas of New Guinea, grass skirts are made from real grass and not from the fibres of sago palm fronds.

Unfortunately, I could not witness the making of grass skirts. No woman makes them these days. So Margaret Osi's description remained somewhat theoretical for me. But photos show that grass skirts were densely woven from numerous individual, very thin strips of grass or fibre. Apparently they would start with the front, which covered the belly, and then continue with the back side (Sequence \ref{ex:nrexwayeup}).