The bodies of the deceased were placed on a mortuary platform in a tree and burned with a fire below the platform until only the bones remained. This story is not the only Kilmeri source for this procedure. During a visit to the village of Awol, the local consultants also spoke of this tradition, which had only recently disappeared.

This method of treating corpses is widespread. It has also been documented for the Simog and Daonda people, who both speak a language of the Border family, south of the Bewani Mountains (\cite[223]{Seiler:1985vo}). The bones of the deceased are preserved and are supposed to transfer the power of his or her spirit to the living. They are usually carried in a (special?) net bag and are part of the hunting gear when a close relative of the deceased goes hunting (see Sequence (\ref{ex:t6exkaepul})). This bone-related custom has also been reported for the Telefomin people, who live on the upper reaches of the Sepik (\cite{craig2018}).  

The conceptual difference between the traditional and the Christian way of burying the dead seems to be as follows: In the past, the dead were placed in a tree, whereas today they are enclosed in a coffin and placed in the ground. A tree might symbolise openness, while a coffin covered with earth symbolises the locking up of the dead. This may mean that it is believed that the spirits of the dead are no longer able to help the living.