Three of the texts in this chapter describe a major killing of people. In two cases, this is done by an individual in revenge for an unpleasant or morally inhibiting situation. In both stories, a large number of people are killed, namely the population of an entire village (Texts \ref{sc0203} and \ref{sc0206}). In one of these stories, the killing is projected onto an evil bush spirit that devours the people of an entire village (\textref{sc0203}). In two stories, the killing is associated with a natural disaster, namely rain floods (\textref{sc0206}) and an earthquake (\textref{sc0207}).

One can also speak of massacres concealed by a narrative that takes centre stage. In two cases, the killing of a large number of people is probably the result of the occupation of land by new arrivals. The third case with the earthquake is more difficult to interpret. The killing can be seen as a punishment for a dance festival that somehow did not conform to the rules. This would correspond to the fact that natural disasters have always been regarded as supernatural due to their devastating consequences (\textref{sc0207}). The site of the festival and the fissure in the earth is known as Yi in the Kiliwes area.