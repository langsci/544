When a married couple returned from a stay in the bush and a successful hunt, they encountered a crocodile while crossing a river. The woman was caught by the crocodile, which turned out to be a bush spirit. The villagers tried to free her from the evil spirit, but it had already eaten all but a few remains of this human meal. The remaining bodyparts were given a usual Christian burial. This story combines a death due to the old way of life with a newly introduced Christian custom.

Like all the large rivers in the area, the Puwani was also the habitat of crocodiles. The crocodile population must have been quite large, otherwise the crocodile hunters would not have come to the village. It therefore makes sense that there would have been crocodile attacks and villagers would have fallen victim to these predatory animals. It is likely that after a person was killed by a crocodile, other people tried to recover a limb or the head of the deceased by diving into the river, as the story tells us.

The narrative begins and ends in Tok Pisin. In the first sentence Andrew Wapi introduces himself as a story teller. 