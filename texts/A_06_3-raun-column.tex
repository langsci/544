\begin{Parallel}{0.47\textwidth}{0.47\textwidth}
    \ParallelLText{\textit{Ripi Ripaekyo bûri seino. Epe aino basuiweko; ki kama inakap rupperie. Yena rapue ar poninipop. Kuso inakap: rapue an kepno riyepomapoip royeniulipop. Epo puakuyo ppaliyeinipop, ya pose puakuyo ppaliyeinipop, ya pose ya ise. Ruri dupua umul nekpamui, ``Nuko ile kaikai painimpi. Yena nuko ar ponien.''}}
    \ParallelRText{Ripi and Ripaek are brother and sister. Their parents are dead, and the two of them live alone, as orphans. The villagers didn't give them any food. So they lived like this: they searched for food themselves. They walked around and ate what they found. The people used to rub faeces on their heads, and they used to rub spoiled sago on their heads, foul sago, burned sago. The two children thought: ``We go and search for food, because the people didn't give us any.''}
\end{Parallel}
    \medskip
\begin{Parallel}{0.47\textwidth}{0.47\textwidth}
    \ParallelLText{\textit{Pu riyepoi. Pu dob sekui, pu eli dob sekui. Pu baîko. ``Eh pu ulili.'' Ri epeyo lilip. ``Puyo nuko yopi saulno.'' Bayo\-piko, baîko. Riyopuno wal luwapoi, waeripi wamo waeaup. Waeaup sepolo. Ri epeyo piapoiro. ``Eh erepe? Anayo?'' ``Koyo? Koyo memi depi.'' ``Oke ri epeyo ari, yip kopi. Awe! Deyo iminep!'' Dob soreyewap, ``Erepe? Deyo ba powe?'' ``Yena koyo epono ya poseno ya iseno ppaliye\-en. Dop koyopi ppulae. Koyo maki\-na ar inakap.'' Pul mopien. Epo ya puno pusiyeen, seke sayeen. Dop maki, ereru. Dipsu sien, wîlno ripien. Riyopuno royenen. Kinoipno dipsu neno ponien. Ropyo uleien, muelien: ``Deyo ilap! Ko yala pu makopi.''}}
    \ParallelRText{They caught sight of water. They found a pond with brackish water. The water has dried up. ``Eh, here is still some water!'' Pieces of wood were poking out of the water. ``We'll scoop the remaining water with a shovel.'' They scooped it, the pond was dry, and they caught the fish with their hands, \textit{waeripi}-fish, \textit{wamo}-fish, \textit{waeup}-fish. The \textit{waeup}-fish vanished. They uncovered the wood (at the bottom of the pond). They heard a voice: ``Eh, what is this, who are you two?'' The children replied: ``We two? We two are your grandchildren.'' The woman said: ``This is not uncovered wood, this is my house. Come, come here you two!'' She was looking with surprise on them: ``What is this, what did you two do?'' The children replied: ``The villagers rubbed us with faeces, with spoiled sago, with burned sago. Our bodies are bad, we are not living well.'' She bathed them. She washed the faeces and the sago off their bodies with water and cut their hair. Now their bodies are good and strong. She cooked rice for them and distributed it on plates. Then she gave it to them. When they had eaten, she gave them raw rice. She put it in a basket and said to them: ``Go! I will perform rain magic now.''}
\end{Parallel}
    \medskip
\begin{Parallel}{0.47\textwidth}{0.47\textwidth}
    \ParallelLText{\textit{Riyopuno sowo sepiana. Pu silepokûne. Pu poro dupikau. Pepuol peia kûno, ani duruwa. Punipino ilo, pu riyeip. Piu bakûnko, luwapoi. Piu baluwaoiko, loloi, klokni weloi yipyo. Ripi Ripaekyo yena muelien, ``Ine molap, piu kauna oso mape.'' Yena molo kiniyo. Epe aino ruri roise piu luwapo. Pu busukna, kimike nomoina, boyopuno losna die poniyena. Waeripi pule. Ere mini wiye\-yap! Waeripi sepolo. Mi rika baka lo. Waeripi ere mini, ere mini oke rkaro. Pu inerna walina, yena puyo silekûnwepu.}}
    \ParallelRText{Then she ground and scattered ginger. Soon the rain was dripping down steadily. It rained heavily until darkness. \textit{Pepuol}-frogs and \textit{peia}-frogs jump\-ed down to water pools, in daylight at dawn. In the morning the children went and saw all the water. Frogs have jumped down, and they caught them with their hands. They wrapped them into leaves and carried one package to their former house in the village. Ripi and Ripaek said to the people: ``Go down there, there are frogs in great numbers, there are more there.'' So the people went, all of them, mothers and fathers together with their children, and they caught the frogs.{\footnotemark} The water reached their ankles, then their shins, later their thighs and waists. The \textit{waeripi}-fish came. Here they come, catch them! Then the \textit{waeripi}-fish vanished. The fish swam there, to the other side; then the\textit{ waeripi}-fish came here again, they moved about here and over there. The water reached up to the armpit, to the neck, and people started drowning in the water. 
}
\end{Parallel}\footnotetext{Frogs were and still are valuable food and people like to eat them.}
    \medskip
\begin{Parallel}{0.47\textwidth}{0.47\textwidth}
    \ParallelLText{\textit{Buoko buar wepulo. Yena kiniyo lelie\-wepu. Yena supuli basupuliko. Yena ule\-wole\-layep, Inuges ol Ir Inuges pur Ir. Isko lopapien dopyo. Kuru.}}
    \ParallelRText{Buoko brought an axe and killed all the people. The people's bodies were lying everywhere: the Inuges from the mountain Ir and the Inuges from the plain Ir. He marked their bodies with black paint. It is over.}
\end{Parallel}
    \medskip
\begin{Parallel}{0.47\textwidth}{0.47\textwidth}
    \ParallelLText{\textit{Amou ai kep sano, ``Yara ako kopi ari? Dupu arka lo?'' ``Balok. Dupu yena wulien; yena piu lu.'' Monomno molo, dob sopop: yena kiniyo supuli. Amou yena kiniyo lakwepu. Ako kepyo nekip. Ako elno; epul kepyo isau re polip. Buoko dor epe Amoupi suelo. Amou suelno. Dor lumî solo upuna wo\-puem. Amou ol epi baka sesiyo, Buoko nisenap. Ol epi baka nisenap, baka lenap, ol Asaul. Sepue pele suko, bopap sepalo. Sepue pele wauna. Riyopuno wami bî sepeipana, dob pop, ``Buoko upule.'' Amou lu paliya. Amou kike, kikero kikero. Ber pûke, ber yelo mono powolap. Yeloyo ye, boyo puana. Yena muelien, ``Ko Buoko baluik.''}}
    \ParallelRText{Amou asked his father Yara: ``My wife is not there? Where did Dupu go?'' The father replied: ``She has left. Dupu followed the people, who were catching frogs.'' Amou and his father Yara walked along the path and realised: All the people died. Amou recounted the  names of all people. He was standing next to his dead wife. The woman had been pregnant; in her ear was an \textit{isau}-feather. Buoko cut Amou's big toe. Now Amou has a gash. The cut left a scar; alright, he will be marked.\footnotemark Amou fled to the other side of the mountain and lay in wait for Buoko. He lay in wait for him on the other side; he waited for him at Mount Asaul. He cut \textit{limbum}-ribs and fenced a pig trap. He put the \textit{limbum}-ribs side by side. Then he drilled a spying hole and was looking through: ``Here comes Buoko.'' Amou shot him dead. Amou ran and ran. He lost his tongue as the tongue was dragging on the ground behind him. He fell over and got up again. He said to the people: ``I shot Buoko.''}
\end{Parallel}\footnotetext{Amou is one of the survisors of the killing. He must have encountered Buoko who cut his foot. Amou fled and planned to take revenge.}
    \medskip
\begin{Parallel}{0.47\textwidth}{0.47\textwidth}
    \ParallelLText{\textit{Pu bueso lili. Pu ikoiele. Pu nem kep roki: Pu ppulae. Puro yena lil lili, Inuges lil lili. Yena sukei riyo mape Pu\-ppulaeyo.}}
    \ParallelRText{The water is salty like the sea. The lake is very large. The name of the water is: Pu ppulae. The water is the blood of the people, the blood of the Inuges. The spirit of the people lives there in Lake Puppulae.}
\end{Parallel}
    \medskip
\begin{figure}
    \centering
    \includegraphics[width=0.9\linewidth]{figures/06LakePuppulae_20250222_0010.jpg}
    \caption{Pu ppulae}
    \label{fig:puppulae}
\end{figure}
    \medskip