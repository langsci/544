Right at the beginning of my field research, Susan Bisam was eager to describe the process of sago production to me, starting from cutting down the sago palm to mixing the sago pudding for consumption. Here she lists the activities and steps without explaining them in much detail. Susan spent most of her time processing sago for the family and also to sell in Vanimo if there was a means of transport to town. She was a strong woman and was capable of felling a fully grown sago palm by herself. The fibres of the palm bark are so hard that they were referred to with the word \textit{re} meaning `feather’. They are also similar in shape.

Susan's Kilmeri is very basic, but it contains the essential vocabulary. 

A female cousin of Susan Bisam told her the following little story about pigs eating sago during night time, which Susan re-narrated in Tok Pisin in December 1999. It is included here because it adds a certain flavour to the processing of sago. 

\begin{Parallel}{0.47\textwidth}{0.47\textwidth}
    \ParallelLText{\noindent \textit{Pik man ol kaikai saksak. Em tokim pik meri: ``Yu kam nau. Yumi go nau. Em i tulait nau. Man bai kam, em bai sutim liklik pikinini pik.''}}
    \ParallelRText{\noindent All the pigs eat sago. Once, a male pig said to its wife: ``Come on, let’s go. It’s becoming morning. The humans will come, and they will shoot the little piglets.''}
\end{Parallel}

\vspace{.4cm}

\begin{Parallel}{0.47\textwidth}{0.47\textwidth}
    \ParallelLText{\noindent \textit{Em tokim pik meri: ``Ol i sutim pikinini pik bilong yu pinis. Yu ken krai i stap. Wari stap bilong pikinini bilong yu. Ol i karim pikinini bilong yu na kaikai. Mi tokim yu: `Yumi go nau. Yumi mas hait long haus bilong yumi.' Yu no harim tok bilong mi. Yu sakim tok.''}}
    \ParallelRText{\noindent Later on the male pig said to its wife: ``Now they have shot our young ones. Now you are crying. Your little ones weren't lucky. They took the little ones and ate them. I told you: `Let’s go. We must hide in our place.' But you didn’t listen to my words. You were disobedient.''}
\end{Parallel}

\vspace{.4cm}