This hunting story is about a supernatural power that enables the younger of the two brothers to change his appearance. He transformed into a dog and behaved like a hunting dog. In this way, he managed to secure two tree kangaroos as prey. The older brother did not notice the younger brother's transformation (see Sequence (\ref{ex:nrexkosiye})). He only became aware of it later in Sequence (\ref{ex:nrexwoworr}). After the hunt, the younger brother miraculously regains his human form. The story is set at a bush camp, and such camps are usually established in the deep forest for a few days in order to procure new food supplies. Therefore, the younger brother did not enter the main village in his changed appearance.

Dogs are indispensable for hunting marsupials. Hunters knock or shoot them down from trees and the dogs catch them on the ground (cf. \cite[135-138]{Brumbaugh:1984tn}). 

Andrew Wapi's text contains some code-switching to Tok Pisin. Notably, we find Tok Pisin combined with the Kilmeri light verb \textit{pi}, for example \textit{kamap-pi} `become', \textit{tanim-pi} and \textit{senis-pi}, both `change'. An example is Sequence (\ref{ex:nrexkowors}). Even though the change of appearance is a narrative topos and an established concept in the Kilmeri tradition, there does not seem to be a verb with this meaning. In the story \textit{Wapues} (\textref{sc0205}) in Sequence (\ref{ex:nrexaesika}), the phrase \textit{dop kep pûke} `lose his skin' is used, but it is followed by Tok Pisin \textit{kamap-pi} and \textit{senis-pi}. In the story \textit{Kusudua} (\textref{sc0304}) in Sequence (\ref{ex:nrexduapon}), the expression \textit{dop kep poname} `give his skin to sb' occurs, again followed by \textit{senis-pi}. For the Tok Pisin verb \textit{lukautim} `take care of' there is a clear equivalent in Kilmeri, viz., \textit{yai} `take care of'. At the same time, the speaker uses Kilmeri serial verbs when necessary, for example in Sequences (\ref{ex:nrexlippue}), (\ref{ex:nrexsskaka}), (\ref{ex:nrexbipued}), (\ref{ex:nrexastaim}), (\ref{ex:nrexdiriim}), and (\ref{ex:nrexmikimi}), and thus exhibits good language competence.

Another comment concerns the narrator's style. The text is characterised by a lot of repetition. For example, the fact that the younger brother has turned into a dog is repeated several times. The catching and killing of the tree kangaroos is also repeated several times with almost the same wording. As a result, the text is not a fluid narrative. Instead, the narrator appears to constantly reassure himself about the plot.