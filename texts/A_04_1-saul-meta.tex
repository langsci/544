The story \textit{Haus tambaran} was told by Margaret Osi in Ossima village in February 2000. The Tok Pisin title, given by Margaret herself, refers to the male cult house. The text contains references to older customs.

Note that the text shows some code-switching to Tok Pisin. For example, food is only referred to by the Tok Pisin word \textit{kaikai}. Although the story is told in plural with exclusive \textit{uke} `we', the agentive suppletive plural \textit{ile} `eat.\textsc{pl}.\textsc{a}' of the verb \textit{ni} `eat' is not used. Instead, the singular inflection \textit{ni} is used.