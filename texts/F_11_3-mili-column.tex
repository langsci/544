\begin{Parallel}{0.47\textwidth}{0.47\textwidth}
    \ParallelLText{\textit{Ai Milipi yilauyo pulo. Opono pulo, lakiyoko, woko Vanimoyo. Ko muelpup: ``Epee, de ruri kopi wokap haus sikyo! Aepu ikoina pi. Aepu kana po.'' Dokta muelien: ``Deyo yala haus sikyo wo\-nake. Aepu bueyo pusiyekap!'' Ko Mili wepulo. Marasin pono aepuyo, lolono, dokta muelo: ``Fonde operesenyo le, de ni kepem! De yala lam.'' Fonde punipino tenklokno lo. Aepu dûkû ikoiele. Dokta kiniyo riyemayo. Ppulae, sut ponamo. Yeniyo wenepu. Nanayo puenpo, sesiye-piyo. Aepu ppulae puppuli polip. Mara-sin pono, aepuyo lolono. Mili wemipu yeniyo, yeni koyopi. Yeniyo nuweikûpu. Yeniyo nuip aua klokni. Riyopuno puana, ko muelpup: ``Bubu, ko ni muli.''  Nes: ``Kenem, umul maki amapoipe!'' ``Bubu, ko el sui, ni muli!''  Ko ni ponamo, ya yûr dû yûr su roise. No. Ya kesiyo.}}
    \ParallelRText{Mili's father came to the village. He came by car, picked Mili up and brought her to Vanimo. He said to me: ``Mother, accompany my child to the hospital! The ulcer is big, it grew very fast.'' After Mili's examination the doctor said to us: ``The two of you stay together in the hospital now. But first you go and wash the ulcer in the sea!'' Then I brought Mili back to the ward. She got medicine on the ulcer, and it was dressed. The doctor said to her: ``Thursday you undergo the surgery. You must not eat anything, you just come.'' Thursday morning at 10 o'clock she went to the treatment room. The ulcer smelled strongly. The doctor looked carefully at everything. It looked bad. He gave her an injection. The nurses carried her to the operation table. The doctor cut with a lancet. He cut the ulcer out. The ulcer is bad. There were depositions of fat. He put medicine on the ulcer and dressed it. Then they brought Mili to her bed, to our bed. They laid her down on the bed. She was sleeping for one hour. Then she woke up and said to me: ``Granny, I want to eat.'' The nurse said: ``She must not eat! First her mind should be clear.'' After a while Mili said again: ``Granny, I am hungry and want to eat.'' I gave her some food, sago  with chicken meat and eggs. She ate. She finished the sago.}
\end{Parallel}
    \medskip
\begin{Parallel}{0.47\textwidth}{0.47\textwidth}
    \ParallelLText{\textit{Aepu ikil am poli, puppuli polip. Tunde seken operesen lo. Nanano puenpo, sesiyepiyo, aepu suelpiyo. Lolono, wemi-pu yeniyo. Wemipu yeniyo nuweikûpu. Aua klokni nuip, bo mulanapop: ``Epo ko ni, pili yasiyap! Ekuyo laye.''  Puana: ``Ko aiyo le.''  Mi due nu, due nuro. Mipuana: ``Bubu, ko el sui. Ko ni pi.''  Bo mulanapi layepana, kuru. Puaku maki pi. ``Ni ko powaip!''  Ko wal dû yûr su roise yano ponamo. Banok, bakesiyok.}}
    \ParallelRText{The ulcer was still dirty. There were still depositions of fat. On Tuesday she underwent the second surgery. The doctor cut with a small knife, cut along the ulcer, and excised it. He dressed her wound. Then they brought her here to her bed. They brought her here and laid her down on the bed. She slept for one hour and was talking confused things: ``I eat faeces, spread a cloth! Put it on my behind.'' She woke up: ``I go to my father.'' She slept again and was sound asleep. Then she woke up again: ``Granny, I am hungry, I do eat.'' She quit babbling nonsense. It had come to an end. Her head was fine. ``Give me something to eat!'' repeated Mili. I gave her fish and eggs with sago. She has eaten it up. She finished all the food.}
\end{Parallel}
    \medskip
\begin{Parallel}{0.47\textwidth}{0.47\textwidth}
    \ParallelLText{\textit{Ukenakap. Yena kiniyo mapap wot operesenpiyo. Aepu solo loloulipop wîs dupua. Aepu milolap, milolap, lolou-lipop. Dokta aepu riyeno: ``Aepu depi maki. Pepualso bapok. Aeppu pon suli.'' Dokta muelo: ``Ko lil depi riyeipe.'' Lil aska. Dokta Wi ko musiyo: ``De ko upuna wulimonpi lil riye. Milipiso upuna, riye solo pi. Dedukoyo ilei.'' Dokta lil riyeno. Lil poro. Lil Charlespi riye: ``Lil depi upuna, Milipiso.'' Lil wepulo. Mili yeniyo nuip, lil sutyo po. Lil sut mono na, ppae mono na. Ain riyo koliyo nes lil riyo koliyo. Lil dop kepyo lo. So solo nakap, lil leip. Amakesiyowole. Bakesiyowoloko, aska. Sut pûke, an plastano penei. Lil mi yala keminem.}}
    \ParallelRText{We remained staying together. Many people were staying in the surgery ward. For two months, the nurses just kept dressing Mili's ulcer. They redressed the ulcer again and again. They kept dressing it. One day the doctor examined her wound again: ``Your ulcer is good. It has become like sound flesh. It is red, with straight closing edges.'' The doctor said to her: ``Before you go home, I'll check your blood.'' There aren't enough red blood cells. Doctor Wi sent me for a blood sample: ``Please follow me kindly to check your blood. In case it is like Mili's, alright. Only to check it, let's go.'' The doctor checked my blood for Mili's benefit. He took some blood. Then he checked her father Charles's blood: ``Your blood is alright, it is like Mili's.'' So they brought blood of him in order to infuse it Mili. Mili lay on the bed. They prepared the blood transfusion. The blood entered through the needle and flowed through the blood vessel. There was a metal stand. The nurse hung the blood bag there. The blood went into Mili's body. Mili was staying like this; the blood was flowing. It has still to be used up. Now it is used up. No blood is left in the bag. The nurse took away the needle and pressed a bandage on the crook of her arm. Now the blood won't come out.}
\end{Parallel}
    \medskip
\begin{Parallel}{0.47\textwidth}{0.47\textwidth}
    \ParallelLText{\textit{Koyo inakap, Dokta muelien: ``De operesenyo le. Ko dop pili depi srene. Pili monina srene aepuyo pi. Aepuyo panapne.''  Mili replied: ``Ko ba muli!''  ``De lap! De epemna upuna lam!''  Umul kep nek, ``Ko le operesenyo.'' Yip bîyo lupuana, sut ponamo. Due nuip. Nes dupua, nes kini dokta kini wonino. Bo ar muel. Aepu nepeino, pusiyono. Aepu maki. Bou pili srono, aepuyo pono. Ri\-yo\-puno aepu lolono, bou lolono. Nes wemon yeniyo. Nes klokni nuweiko numuelna nuro. Puana, ``Ko el sui. Ni muli.''  Ko ya namo yûr su wal roise. Kesiyei. Ko nes muelno: ``Aepu sipine, ikoina lolono. Ikoina kikipiyo. Minepeinap.'' Nes ko boi malo. Auna lolono. Upuna ar sipi. Aepu sipi kep as. Nes muelien: ``Due an baka dupua!''  Koyo due an baka dupua inakap. ``Mili de awe! Uke aepu depi nepei.'' Kene-peipno dob aepu riyeno: ``Aepu depi maki.''  Aepu epi pusiyeno, marasin peneino, aepuyo lolono. Koyo yeniyo ilo. Koyo inakap, dokta Wi pulo: ``Deyo yala Fraideno ilei.'' Pulo, muel: ``De lap, kuru. De yilauyo lap! Deyo yilauyo ilap!''}}
    \ParallelRText{The two of us stayed on. The doctor informed us: ``You need to undergo a further surgery. I will scrape off a piece of skin from you. Scrape off a small piece and put it on the ulcer. I'll put it on the ulcer for you to let it heal completely.'' Mili replied: ``I don't want to.'' Doctor Wi repeated: ``You go, please you should go quickly.'' She thought about it: ``I will go for the surgery.'' She went into the treatment pavilion. They gave her an injection. She fell asleep. Two nurses, no, a nurse and a doctor each called her. She didn't answer. They undressed the ulcer and washed it. The ulcer is good. The doctor scraped off a piece of skin from her thigh and put it on the wound. Then they dressed her ulcer and dressed her thigh. The nurse brought her to the bed. One nurse laid her down before she slept for a long time. She awoke: ``I am hungry and want to eat.'' I gave her sago with eggs and fish. She finished the food. Later I said to the nurse: ``Her ulcer hurts, you dressed it tightly. It held too tightly, please take off her bandage!'' The nurse listened to me. She redressed her wound carefully. Alright, it doesn't hurt anymore. No hurting ulcer. Soon the nurse said to us: ``Seven days!'' So we stayed on for seven more days. ``Mili, come, we'll undress your ulcer!'' Having undressed it, they looked thoroughly at her wound: ``Your ulcer is healing well.'' They washed the edges of her ulcer, pressed medicine upon the wound, and redressed it for her. We went back to our bed and were staying on, until Doctor Wi came: ``The two of you will leave on Friday.'' Some days later he came and said: ``You leave now, it is finished. Go to your village! You two may return to the village!''}
\end{Parallel}
    \medskip
\begin{Parallel}{0.47\textwidth}{0.47\textwidth}
    \ParallelLText{\textit{Koyo Waisan Camp ilo. Koyo due dupua inakap. Ai kep ka hairimpo. Mili mike lo yilauyo. Ko boyo lo due bano. Ko Theresiayo opo ikoino iloi. Simon koyo ukel. Junksenyo wilikûno, koyo dorno iloi dupuni. Jerry disei kopi wonino. Yilau kepyo nakap, koyo riyo inu yip kepyo. Ako kep yûr si pewo si. Ya aska. Yûr pewono koyo inoi. Uke kumune due sap. Yûr bo mo. Yaep bo mo ``kukukuku'', yipp bo mo ``nananana'', yopp bo mo ``kopokopo lolololo''. Duruwa. Uke dueyo puana. Sû mappo pewo si. Ipino waeupp si biper si. Puliyo ipino layewo. Pupuol polip pupuol nisi. Wîlyo wapo. Ko powa Theresia namo. Biper waeupp roise, koyo pewono ino, Theresia ruri kep roise Joanna. Kukuno\-pno uke molo dorno monomno. Uke moloro pu Pual\-yo, uke pulmopip. Uke pul komopipno uke mapap pewo ilo. Molo yilauyo moloro yilauyo paeau.}}
    \ParallelRText{We went to Waisan Camp and were staying there for two days. Mili's father hired a car, and Mili went back to the village first. I went later, three days later. Theresia and me went on a big truck. Simon picked us up. At the junction he let us off, and we went by foot until night. We called out for Jerry, my brother. We were staying in his village and slept there, in his house. His wife cooked chicken, and she cooked bananas. There was no sago. We ate the chicken with bananas. We all slept. Then the rooster crowed. Birds called ``kukukuku'', the wild fowl called ``nananana'', other birds called ``kopokopo lolololo''. It was dawn. We rose from sleep. Jerry's wife lit a fire and cooked bananas. In a pot she cooked eel and possum. She took it off the fire. She put the cooked food aside with the pot. The heat still lasted. The heat has ceased. She served it in a dish. She gave some to me and gave some to Theresia. We ate possum and eel with bananas, me and Theresia together with her child Joanna. Having eaten we walked along the trail. We went to the Pual river and forded it. When we had crossed, we were sitting a while and ate bananas. Then we walked on to our village. We walked to the village of Ossima and arrived there.}
\end{Parallel}
    \medskip
\begin{Parallel}{0.47\textwidth}{0.47\textwidth}
    \ParallelLText{\textit{Bo kuru.}}
    \ParallelRText{The story is finished.}
\end{Parallel}
    \medskip