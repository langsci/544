Margaret's third autobiographical story begins with the year of her birth. This is followed by the years of her marriage, the birth of her first child and finally the year of her husband's death. In this way, she embeds her life in a modern way of reckoning time. Along with this kind of mental framing, she tells of her  adolesent life in steps that correspond to those of girls and young women of the same age. The capacity of helping one's parents in daily life is a big step in growing up. It prepares young women for their future responsibilities after marriage. Margaret herself did not marry early. She mentions three times that she currently lives on her own (Sequences \ref{ex:nrexppilin}, \ref{ex:nrexboyonu} and \ref{ex:nrexmemiko}). According to the dates, she must have been in her early twenties when she got married. Then she married a man who was much older and she became his second wife (Sequence \ref{ex:nrexkoukip}). The husband was the head of the village at the time. He made all the decisions and arrangements for the Catholic mission in Ossima. Obviously there was some tension between the stepbrothers, i.e., the sons from the two wives, which I could witness myself. The fact that Margaret had usurped the position of Lis Osi's wife may have been one of the reasons why his firstborn son Paul Waia  killed Margaret's sow (Sequence \ref{ex:nrexpaulwa}). Margaret's own oldest (surviving) son David seemed to have a certain influence in the village by mediating disputes. 

Margaret always admired the staff at the Mission. The head of the agricultural section of the Mission, Brother James Coucher CP, was a charismatic personality and was adored by all the villagers and the wider community. His esteem is apparent in Sequences \ref{ex:nrexyaeaul}-\ref{ex:nrexepueka}. The private cattle business also continued after `Brother Jim' had left Ossima. For example, I was able to buy a cow for Margaret from one of the cattle owners. Margaret was proud to be the owner of a cow and was happy when her cow became pregnant. She finishes her story with this fact.