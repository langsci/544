This story is a legend about the origin of the moon. People have always wondered how the light of the night came into being. Here, a beautiful but kleptomaniac woman abandoned her confused and sad husband and fled into the sky, where she became the moon. She was the man's second wife. His first wife had understood that she was not like other people. She warned the husband not to affront her and hide her. But he became jealous. So she decided to abandon the place and built a ladder to the sky.

The last two sequences of the story reflect the kindness as well as the usefulness of the moon. It turns the darkness of the night to light and enables people to move safely.

Moon stories are quite common in New Guinea. The moon is usually associated with a woman. But in some oral traditions, the human who changes to the moon is male, for example, in Muyu (\cite[33]{Zahrer:2025oj}). In a legend of the Orokolo (Gulf Province), sun and moon are two brothers; the moon is the younger brother. During the day, the sun is a beautiful young man. But during the night, the moon receives all the beautiful adornment and radiates like his older brother (\cite[42-26]{Kiki:1969tx}). Generally, I want to point to Beier's collection of Papuan tales about sun and moon (\cite{Beier:1974xu}). 

\begin{figure}
    \centering
    \includegraphics[trim=0 10 0 10,clip,width=0.5\linewidth]{figures/09StalkBananas_20250222_0015.jpg}
    \caption{A big stalk of bananas}
    \label{fig:bananas1}
\end{figure}

\begin{figure}
    \centering
    \includegraphics[width=0.9\linewidth]{figures/10KindsBananas_20250222_0016.jpg}
    \caption{Three kinds of bananas, including the red \textit{Musa troglodytera}}
    \label{fig:bananas2}
\end{figure}
