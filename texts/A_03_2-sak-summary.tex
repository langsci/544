The story recounts the creation of the ``golden world'' in which the Kilmeri once lived. Sakou was the creator of all kinds of food, especially wild animals and sago. He grew up as an abandoned boy far away from his family and lived alone in the forest. By staying in a tree, he was able to sustain himself. Through clever behaviour, he also managed to trick an evil bush spirit and get rid of it. He eventually moved to a second tree that grew very tall. Two sisters were living in its crown and took care of him. When the bush spirit returned, the three people tried to trick it. At first, the two women tried to overpower it, but without success. Finally, Sakou came up with the idea of feeding the people of a village to the bush spirit. The bush spirit was unsuccessful in devouring a large man, whereupon it died. After the bush spirit's death, the three returned to Sakou's original village, where his nephew recognised him as his uncle. The clan welcomed him, but Sakou saw that the people were eating poor food. So he showed them what he himself was eating. Whilst the people were asleep, he scattered all kinds of seeds on the ground, and after sunrise, in the early morning, the world looked different and offered plenty of good food. Since then, all the clans have been living together with Sakou.

The outline of the story is: (i) The hero grows up away from his family, (ii) he defeats a bush spirit, sacrificing the inhabitants of an entire village in the process, (iii) he receives support from two women, (iv) he returns to his people and his village and helps them to a much better life.

An interpretation of the story leads to the question why an entire village was sacrificed. We can even rephrase the question as: The inhabitants of which village were massacred? The answer to this is speculative, but it may reflect the fact that the Kilmeri invaded the Pual Puwani river basin. So the story would describe the displacement and killing of the putative former inhabitants. These earlier people may have travelled inland from the coast and had a different diet, more fish and less sago. This would explain two details of the story. Firstly, even Sakou's own people had ``fish mouths''; and secondly, Sakou was the creator of wild animals and various types of nutritious sago. In addition, the bush spirit, as a representative of evil, is a legacy of the forest people inland.