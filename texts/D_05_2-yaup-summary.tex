When my older consultants were young, their lives were based on materials that were largely provided by the natural environment. For example, the women did not use metal pots for cooking, but roasted food over the fire (bananas, breadfruit) or used vessels that they could make from bush materials. One type of such vessels used for cooking are bamboo tubes. Bamboo trunks are naturally segmented. If you cut off a long trunk, you get several tubes with a base and an opening. Such a tube served as a kind of pot. Because of the small diameter, several such tubes were needed to prepare a single meal. The tubes also needed a firm base. This meant using suitable timber to build a kind of grid in which the tubes could stand upright. Note that bamboo is quite fire-resistant (See also \textref{sc0201}, Sequence \ref{ex:nrexupulbo} and \textref{sc0208}, Sequence \ref{ex:nrexdueyam}).