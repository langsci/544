Margaret Osi sings a song about the sun and the moon. She composed the song herself. Sun and moon are supposed to provide light to overcome the darkness of the night. Her Kilmeri paraphrase of the song was \textit{nini napiya}, which means ‘the sun comes in and fills the room with light’. The song consists of four sequences structured in verses: 1 1 2 2 3 3 4 4 1 1 2 2. Each sequence begins with a long melodic vowel, either \textit{eee} or \textit{aaa}.

The song was Margaret's response to my request to hear her sing a traditional song. Unfortunately, this was the only Kilmeri song I ever heard her sing.

The phonological structure of the song differs from that of spoken Kilmeri. Stressed vowels at the end of a word sound like a rising diphthong (Sequences \ref{ex:nrexpunial} and \ref{ex:nrexdupueen}). The verb \textit{pini} changes to \textit{rini}, probably due to assimilation of the place of articulation to the preceding and following /n/ phonemes (Sequences \ref{ex:nrexpiniya} and \ref{ex:nrexriniya}). Lastly, the emphatic suffix \textit{-ya} becomes \textit{-ye}. Again this can be seen as assimilation to the high vowel /i/ which dominates Sequences \ref{ex:nrexpiniya} and \ref{ex:nrexriniya}.