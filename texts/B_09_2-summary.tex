The title \textit{Masalai pukpuk} `The crocodile bush spirit' was given by the narrator.

The narrator Susan Bisam interprets the husband's hesitant behaviour as fear, an aspect that does not appear in Andrew Wapi's story. In terms of content, there are a few differences to Andrew Wapi's version. Firstly, in Susan's version, the villagers eat all the meat that the husband had brought back from the couple's stay in the bush. Secondly, the crocodile only eats one arm of the victim, whereas in Andrew's story it eats half of the body, leaving no \textit{meri tru} `full woman' to bury.

Susan had a remarkable knowledge of old stories and so it would have been a loss if she had not been given the opportunity to tell some of these stories in the language that suited her best. She is a great storyteller and would embed her tales in an atmosphere of wonder and drama. Indeed, she spoke with a theatrical voice and expression, making the audience feel like they were watching scenes on a stage. She speaks in short sequences with many repetitions and prolongations of vowels; these prolonged vowels are given a tonal melody. The transcription tries to preserve this as much as possible.