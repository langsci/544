\begin{Parallel}{0.47\textwidth}{0.47\textwidth}
    \ParallelLText{\textit{Epe kopi nako 1942. Ko moniseso, wepul\-yo nakap. Ko ireri, nakero, puana, ko dorno lo. Woa polip. Epe kopi muelpop monem. Riyopuno woa kuru. Ko nowo ikoiele. Ko Silaso. Ko Silaso kupuapno mi ko Nancyso. Ko Nancyso nakap boyo ko Deiso. Deiso ko ikoiele. Ko yili melipop.}}
    \ParallelRText{My mother bore me in 1942. I was very small and sat in a baby sling. Then I crawled, I became able to sit, I stood up, and I walked on my feet. But there was war. My mother kept telling me to be quiet. Eventually the war was over. I grew big. I was like Sila. After reaching Sila in height I became like Nancy. Then I became as big as Dei. I was really big and could carry heavy loads.}
\end{Parallel}
    \medskip
\begin{Parallel}{0.47\textwidth}{0.47\textwidth}
    \ParallelLText{\textit{Ko ppili nakap epe aino ikap roise Omoiyo. Ko due mekiyap. Diri kopi weri kopi ko epe ikap mekiyap, ko ai ikap mekiyap. Ai ikap bi luipop. Bi sali ropno wepulap yilauyo epe aino ikap roise. Boyo nuni ko ikap roise inakap, nuni Esau. Ko uki am ar piyap, ppili nakap. Ppili nakap memi ko ikap roise. Mi ko memiyo lo, memi kopi Lipi. Ko riyo nakap. Saul Iwan Bopule polip. Bo klokni solo. Mi ari kuru. Yena layepana.}}
    \ParallelRText{I lived unmarried at my parents' place in Omoi. I helped processing sago. I helped my younger siblings and my mother and my father. My father went hunting and shot animals. I brought the dried meat in a basket to the village, together with father and mother. Later I lived at my uncle Esau's place. I had not yet taken a husband, but lived single. I lived unmarried together with my grandmother. I had gone to my grandmother, to my grandmother Lipi. I stayed with her at her place. The men's house Saul Iwan Bopule was still in use. It is just one name, \textit{Saul Iwan Bopule}. Then it came to an end. Nothing is left. People abandoned the men's house.}
\end{Parallel}
    \medskip
\begin{Parallel}{0.47\textwidth}{0.47\textwidth}
    \ParallelLText{\textit{1965 ko uki piyo. Ko Luppapyo nakap. 1966 ruri kopi sui roipi moniseso. Ai Jeffreypi bûri aska. Ki kama solo. Ko luono lakiyo. Ai kopi ponamo nuni kopi ponamo luo namayo. Memi kopi kau ponamo Lipi. Lipi pusayepo de luo ko ar powa. Umul kep kauno sileno. Umul kep sneiwap. Kuru kuru. ``Kau de ko baponamko.'' Ruri kopi ipei klokni sui. Ruri ipei makiro. Yala ko makina wo\-nake. Kui kopi makiro ko makina wo\-nake. Kui ppulae ko makina ar wo\-nake. Kui kopi ko rapue ar powa, olo pi. Kui kopi Imelda bo kep as. Pon riso ar pi. Kui roke umul maki.}}
    \ParallelRText{In 1965 I took a husband. I got married with Lis Osi and lived in Luppap. In 1966 my child died, a little baby boy. Jeffrey's father hadn't had a sister. He was alone. He took me for money. He gave my father money, and he gave my uncle money. They shared the money. To my grandmother Lipi they gave a cow. Lipi had complained that she didn't receive money. With the cow her heart became placated. She felt appeased. All right. ``You gave me the cow.'' My firstborn child, this one child died. If the firstborn child is good, then I will live fine in his company. If my daughter-in-law is good, I will live fine with her. If the daughter-in-law is bad, I won't go on well with her. My first daughter-in-law, wife of my oldest son, didn't want to give me food. She is greedy. My daughter-in-law Imelda doesn't talk bad to me. She doesn't look grimly. She is good-hearted.}
\end{Parallel}
    \medskip
\begin{Parallel}{0.47\textwidth}{0.47\textwidth}
    \ParallelLText{\textit{Ko ruri ppusi nako David Yau Lu\-ppap\-yo nako. Ruri ba kini Charles Imop, ruri ba kini Rafael Ppisi. Kiniyo Lu\-ppap\-yo. Yip ikoi Lu\-ppap\-yo polip. Yip moni duyo polip, Ouwinyo polip. Dete ol nem ponamo. Dete ol doriye pu lakiwepu, kiniyo lakiwepu. Pu Bilouppueki sowelaye. Pu lûpi maki. Nuko yilauyo molap, yilauyo pulup. Ko bi yaip biopo biep biwi. Ko yaiwepu. Suo yasiyo, ral yasiyo, ul luan yasiyo, sawa yasiyo, puel yasiyo, due doriyeyo yasiyo. Paul Waia bi kopi lu. Bipupi lu. Bipupi ruri nainpela nako. Ko smep paliyen, kiniyo yalaka molo ewo lipelip, ewo ilap. Ko bi ikap lipelip. Lipeliou ari. Ana luro? Ono ko ar reyo. Ko bo solo malo, bo pulo.}}
    \ParallelRText{I had my second child. I bore my son David Yau in Luppap. The other children are Charles Imop and Rafael Ppisi. All born in Luppap. Our big house was in Luppap. Our small house was in the bush, at Ouwin. The ancestors had named the hill. The ancestors named the hills, the swamps, and the water bodies. They counted them all by their names. The small lake Bilouppueki is filled up now. Its water was clear and good. We went to the bush place at Ouwin. We often came there. We housed some pigs, domesticated pigs, a boar and a sow. I looked after them. I planted coconut palms, \textit{ral}-trees, bamboo and breadfruit I planted, mango and betel pepper, and in the swamp I planted sago palms. Paul Waia shot my pig. He shot Bipupi. Bipupi had nine piglets. I left the door open, and all pigs went outside the yard to find and eat worms. I searched for my pig. I searched in vain, nothing. Who had shot? I didn't see the man. I heard only some gossip. The rumour came to me.}
\end{Parallel}
    \medskip
\begin{Parallel}{0.47\textwidth}{0.47\textwidth}
    \ParallelLText{\textit{Uki kopi ruri duyo woko. Du mosupinap. Depi oki, depi oki. Du dob riyewap, mosupien. Du kiniyo mosupiwepien. Kompani kiniyo pulup oil lipelip. Yena kiniyo mapap. Riyopuno Jeffrey David Jerome Simon Tapi Joe Samou Sepik yilauyo kumune molo kompani roise. Wok pop Angoramyo krismas dupua. Riyopuno midoripulupip yilau kepyo.}}
    \ParallelRText{My husband took his sons to the bush. He showed the bush to them. Here is yours, here is yours. They carefully acknowledged the bush slots. He showed his sons all the bush slots they will own after his death. One day a survey company came to the village. Many people were there. Then Jeffrey Osi, David Osi, Jerome Osi, Simon Tapi, and Joe Samou went to exploration places in the Sepik area together with the company staff. They worked for two years near Angoram. Then they returned to the village.}
\end{Parallel}
    \medskip
\begin{Parallel}{0.47\textwidth}{0.47\textwidth}
    \ParallelLText{\textit{Riyopuno ko el piamu. Ko karimpop Grace rumkari klokni solo. Ko wîs nainpela kanakapno ko ruri nako. Dob ko riye rumkari. Yena ruri kopi woko. Brata Jim dob reyo. Pilot dob reyo. Dop kep ukeso! Yena mi koyo wepulupip. Josepin Bewa, Krisa yako, Bewa ruri nem ponamo. Grace neuli. Baikoipoko skulyo lo priskul. Helen skulim em tisa Amerikapi.}}
    \ParallelRText{I got pregnant. I got pregnant with Grace, my only daughter. I was pregnant for nine months and then bore a baby. I see that she is a girl. Some people brought my child to Brother Jim. He looked at her. The pilot looked at her, too. ``She has our skin!'' Later the people brought her back to me. Josepin Bewa, a woman from Krisa, gave a name to her. She named her Grace. She grew up and went to school, to preschool. Helen, an American teacher, instructed her.}
\end{Parallel}
    \medskip
\begin{Parallel}{0.47\textwidth}{0.47\textwidth}
    \ParallelLText{\textit{Uki kopi basuik. Yar 1990no sui, Sarere September 16. Ko Imeldayo Charlesyo uke yip riyo mapap. Uki kopi puapno yena bo ppulaena womui womoliye. Uki kopi ar mariulipop. Upunaro maki\-na nakap. Yena lotu yaeau nuro duruwa. Yaeau puni nuknoko. Brata Jim wo mop umul polenap. ``Ko yala ar nake. Ko yala lam. De nakapno ko nakap. Yalaka ko le.'' Jeffrey bo ar mui. Ruri kep ari. Bo ar mui. Ar mekiyo. Brata Jim Vanimoyo lo. Bisnis kep ppulaepowolo. Kimike upuna polip, kanakapno. Yalaka ari. Umul kep maki yilau makina polip. Epue ar polip. Yalaka epue kauna. Epue sowelaye. Yol kaupi epue sowelaye. Epue ikoina po. Ri luap poyana, ri lop poyana, ri bayana roise. Ri rur ri rupopin kauna. Kiniyo poyana. Ko	yilauyo nake. Ko ruripiyo nake. Yip kopi Charles ako nake. Kau kopi Ouwin elno.}}
    \ParallelRText{My husband died. He died in 1990, on Saturday, September 16th. Me, Imelda, and Charles were sitting in the house there. While my husband was busy with this and that, people spoke bad words about him. My husband wasn't sick. He was always fine. The people held a traditional wake for him. During the night they held the wake. Brother Jim wept and was very sad. ``Now I won't stay any longer. I want to leave. As long as you were here, my friend Lis, I was also staying in the village. Now I'll go.'' Jeffrey Osi is not articulate. Neither are his children. They aren't eloquent. They didn't help with the farm business either. Brother Jim went to Vanimo. His farm business stagnated. Before everything was fine, when he was still there. But now it doesn't work. Back then Brother Jim was happy. The place was in a good condition, and there were no weeds. Now the weeds abound. Weeds cover everything densely. The cow fence is covered throughout by weeds. The weeds are a plague. \textit{Luap}-trees rose everywhere, \textit{lop}-trees rose everywhere as well as other trees. \textit{Rur}-trees and Pandanus are numerous. All kinds of unwanted trees grew big all over the place. I myself live on in the village. Currently I live at my son David's place. In my house, Charles's wife is living now. My cow Ouwin is pregnant.}
\end{Parallel}
    \medskip