The text begins with a genealogical overview of the narrator Usi Kul and his clan. He then continues with the story of the ancestor Am, as announced in Sequence \ref{ex:nrex}. However, in Sequence \ref{ex:nrexrimpiy} a person called Bu is introduced who is supposed to fulfil people's basic needs. Am, on the other hand, is associated with the upbringing of boys and social life in general (Sequences \ref{ex:nrexrokpip}-\ref{ex:nrexyipsep}). In the past, it was common for boys to be separated from their mothers at a certain age and grow up in a male-dominated environment. Interestingly, four women are mentioned who took care of the boys in this male-dominated world. They may have been foster mothers and even breastfed the boys. This has not been confirmed, however.  

Then it continues with some historical facts about the settlement and how the settlements have changed. Access to water is important, but so is the personal connection to a particular piece of land. Finally, the story ends with a reference to Bu and Am, both of which seem to be of equal importance. 

In essence, the story of Am represents the genealogical lineage of the clans of the people living in Omoi and the surrounding hamlets today. The narrator blends the local indigenous tradition with modern developments; in particular, he interprets the loss of the old male cult houses of discussion and consensus, namely the \textit{house tambaran}, in the light of the new House of Parliament in Port Moresby, where the nationwide political debates now take place. Secondly, he integrates a Christian biblical figure, namely Noah, as an ancestor who gave the people special power, thus relating them to God as the ultimate creator. Seen from the outside, the parallelism of two completely different ancestral lines may seem strange; for the people themselves, it is very probably an attempt to anchor themselves in both worlds, the traditional and the modern. In economic terms, the ancient period is often talked about as a ``golden age'' in which people lived in abundance.