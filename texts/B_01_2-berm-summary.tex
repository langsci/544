There are two versions of this story. The first version was told in January 2000, at the beginning of my fieldwork. It was the first story Margaret Osi told at my suggestion. She seemed a little unsure about telling traditional stories and disputed that she was a good storyteller. During their marriage, her husband Lis Osi used to tell these stories while she listened. But Margaret agreed to tell the story with the following Tok Pisin words: \textit{Mi save dispela stori long binatang}, which is `Yes, I do remember a story about sago grubs'. The next day, I set up the recording environment and she told the story.

The second version of the story was told in August 2004. Why a second version? I had the impression that there were some unclear passages that needed to be clarified. Then Margaret said she would retell the whole story, with the intention of producing it in a more coherent way and in better Kilmeri. She had no doubt that the new version was better, because both her own ability to narrate and my ability to understand Kilmeri had improved over the years of language work.

There are some differences between the two versions:
\begin{itemize}
	\item Version 1 starts with several address formulae. In particular, I am explicitly addressed, which never happened again this way on later occasions. (See below, Kilmeri Glossed Text.) The second version does without introductory formulae. 
	\item In Version 2 the fight between the bush spirit and the man is more elaborately told and mentions four kinds of ``weapons'' (Sequences \ref{ex:nrexeponoi} and \ref{ex:nrexbisepn}). 
	\item Version 2 explicitly mentions that the bush spirit has eaten up the two people including their brains (Sequence \ref{ex:nrexneppin}), while Version 1 speaks only about cooking them. 
	\item In Version 1 the bush spirit has small children (Sequence \ref{ex:nrexnapwap}).
\end{itemize}

Among the Kilmeri it is still customary to mark the claim of a family or a clan to certain bush goods by erecting a taboo sign. But often a dispute arises between two parties because they have not respected such a taboo. So the conflict between Kopukei, the human, and Sukupu, the bush spirit, is a normal social and economic conflict. The story partly shifts into the realm of an evil spirit with the expected bad ending for the human party.

For an outsider, taboo signs are difficult to recognise, as the plant parts with which they are erected are almost unrecognisable to someone who is not familiar with this custom. In their descriptions, people used the Tok Pisin word \textit{tanget} (\textit{Cordyline fruticosa}), a species endemic to New Guinea. This is a shrub with long green leaves (30-60 cm long, 5-10 cm wide) at the top of a woody stem that produces 40-60 cm long panicles of small fragrant yellowish, light purple, or red flowers that ripen into red berries. The wilted panicles can be slung around something as a taboo sign, cf. Kilmeri \textit{aipo nopuane}, which contains the stem \textit{nopi} `to tie with a rope', and can also be used for tying beams together by means of a \textit{liana}-rope.

The death of a bush spirit whose house caught fire is confirmed by a bang, which comes from the bush spirit's gall bladder bursting.

The object of dispute are sago grubs, viz., the larva of \textit{Rhynchophorus ferrugineus ssp. papuanus}, also called \textit{Rhynchophorus bilineatus}, a very nutritious part of the people's diet. The following basic remarks about the grubs and their use go back to \citet[32-37]{Chan:2014qo}.

In Papua New Guinea, the larva of \textit{R.bilineatus} is the most commonly eaten insect, as it is a by-product of sago starch production there. The sago weevil larvae are eaten alive, boiled, roasted or mixed into sago pancakes. Typically, weevils are produced from the sago stump and cabbage remaining after the harvest and processing of the sago palm, \textit{Metroxylon sagu}. In some areas, the spiny-trunk sago palm, \textit{Metroxylon rumphii}, which is not eaten because of inferior-quality starch, is cut and prepared for weevil production, each palm producing 500-600 grubs. The production and harvesting of sago grubs is still an important nutrient of the Kilmeri who live in forest settlements away from the coast and the town of Vanimo. 

\begin{figure}
    \centering
    \includegraphics[width=0.9\linewidth]{figures/07StringSagoGrubs_20250222_0018.jpg}
    \caption{A string of sago grubs on the market in Ossima Asples}
    \label{fig:sagogrubs}
\end{figure}
