The young girl Wapues was abandoned by her family and left alone in the bush. One day, a stranger came to her bush house wanting to stay with her. He looked ugly, black and fat and she wanted to get rid of him. While she was away in the sago swamp, he went to the river and cleansed his body of all the salt it contained. In this way, it is said, he introduced people to the resource of salt. Consequently, his appearance changed, and he became young and handsome. As a result, the girl accepted him as her husband, and they had five children. Together with their son, Wapues' parents left the old village of Ossima to look for a new, perhaps more resourceful place. Wapues and her husband Siyu stayed in Ossima.

The name Siyu suggests that the bearer's ancestry goes back to the ancestor Si. Possibly the name means ‘Si place’ and contains the noun \textit{yo} ‘place’. The name \textit{Siyu} was sometimes pronounced as \textit{Siyou}.

Interpretation and message: The story bears witness to the founding of a new settlement, the present-day village of Omula and its surroundings. It also attests to the common practice of exchanging wives between clans, as Wapues’ clan is obliged to provide wives to Siyu’s clan that settles in Ossima. The girls had to give up their ties to their families in favour of their husbands' families. This does not seem to have been easy, as shown by the image of the adult Wapues still clinging to her mother's breast. Wapues is reluctant to fulfil her marital duties.

Finally, the story explains the origin of salt as a resource. 