According to Susan Bisam, the story originates from the Sepik region. She probably refers to some of the larger tributaries of the Sepik that originate in the Bewani Mountains and the Border Mountains. However, she also mentions the village of Omula, the neighbouring village of Omoi further downstream of the Puwani in the Kilmeri region. Today, the villages of Omoi and Omula are located on the left bank of the Puwani. A few decades ago, however, Omula appears to have been on the right bank of the Puwani, as shown on the official map of Australia/Papua New Guinea, sheet 7192. The area to the south-east of ancient Omula borders the Eastern Pagi language area and people. Presumably the inhabitants of Omula and a group of East Pagi were once hostile to each other. Much of the area between the old village of Omula and the East Pagi village of Imbrinis is covered by swamps in which sago palms are cultivated.

In the story of Muruk, the inhabitants of an entire village met a gruesome death. They had set a fire on kunai grass (\textit{Imperata cylindrica}) to hunt for the fleeing animals, pigs and cassowaries. However, this is the opportunity for the cassowary man to avenge the death of his cassowary mother by killing them all. It is possible that the cassowary man is based on a ritual cassowary dancer, who is the first-ranking character in a fertility ritual called \textit{Ida}, which is performed by a number of peoples of the Border language family (\cite[168-169]{Gell:1975hl}). This ritual was performed annually by the Waina/Umeda, the Punda, and the Waris, who live south of the Bewani Mountains. The ritual was performed up until (at least) the 1970s and 1980s and is described in detail by Gell (\citeyear{Gell:1975hl}). Its purpose was to ensure the availability of sago (\citeyear[15-16]{Gell:1975hl}). The Waris dictionary of \citet{Brown:1986fe} contains about a dozen words and expressions that refer to this fertility ritual. Susan herself used the word \textit{yîs} to describe a day-long break in sago work. In the Ida ritual, this word refers to hot sago jelly, which the dancers throw in the air in the early morning hours after the cassowary dance has ended. The women have to leave the scene and have to sit idle in their houses (\cite[185]{Gell:1975hl}; cf. \textref{sc0704}, Sequence \ref{ex:nrexraunpo}). No other consultant used the collocation \textit{yîs nake} `sit idle'.

The young cassowary's search for his mother\footnote{Cassowary eggs are hatched by the male and the chicks remain in the nest for around nine months under the protection of their father. The female bird is polygamous. The search for the ``mother'' is therefore an over-generalisation of the widespread breeding and caring behaviour of some bush animals. The lowland area north of the Bewani Mountains is the habitat of the Northern Cassowary (\textit{Casuarius unappendiculatus}).}, which is where the story begins, may indicate an important spiritual difference between the eastern Pagi villages and the Omula villages. The Omula, and all Kilmeri, do not engage in the fertility ritual centred around sago and the cassowary dancer, a cultural heritage of the southern groups of the Border people. From the perspective of the Eastern Pagi, the Omula are not authorised to hunt cassowaries without fulfilling the appropriate ritual requirements. Therefore, a young cassowary searching for its mother may convey the message that cassowaries are viciously killed by the people of Omula. Such a sacrilege was to be prevented which is why the cassowary man took action. 

Margaret Osi never mentioned a fertility ritual similar to the \textit{Ida} ritual described and analysed by Gell (\citeyear{Gell:1975hl}). According to her, the spiritual source of the Kilmeri was the men's house that belonged to every Kilmeri village (Cf. \textref{sc03tp2}). I myself know of three men's houses that were in use until the 1970s (or even longer). In contrast, Gell explicitly denies the existence of men's houses in the village of Umeda (\citeyear[10-11]{Gell:1975hl}). This difference in spiritual tradition is not the only difference between the Kilmeri and the Pagi. Linguistically, the Pagi language has more grammatical features in common with Waris and Imonda than with Kilmeri. In \citet{Gerstner-Link:2023lo}, I hypothesise that the Pagi have migrated into the Puwani Basin from the south by crossing the Bewani Mountains. This hypothesis is backed up by the fact that the Waris people are described by Gell as fierce and are said to have repeatedly driven neighbouring peoples/clans out of their settlements (\citeyear[5; 22]{Gell:1975hl}). The same could have happened to the Pagi, who then crossed the mountains to the north in search of a better habitat. Later on however, with the arrival of the Kilmeri clans, they became victims of displacement again.\footnote{A look on the language map in \figref{fig:languagemap} shows that the Pagi language territory is split into an Eastern part and a Western part, which is probably due to the intrusion of the Kilmeri from the west via the Bewani valley. The Kilmeri have ties to the Manem language area and some clans in the west (Jeffrey Osi, p.c.).}

Note that the title \textit{Muruk} in the story only refers to a cassowary. The animal is not called a \textit{masalai} `evil bush spirit'. Therefore, this title is in contrast to the other titles Susan has chosen for her stories which include the word \textit{masalai}. However, the cassowary man is said to have practised cannibalism and is described as truly savage: \textit{Em i wel, wel olgeta}. His wild behaviour corresponds to the behaviour of cassowaries in the wild, which are shy, solitary and aggressive. These birds grow up to 1.70 metres tall, equivalent to a human; and they can jump as high as 1.5 metres. This makes its appearance threatening.

\begin{figure}
    \centering
    \includegraphics[width=0.5\linewidth]{figures/26SusanBisamStoryTeller_20250403_0003.jpg}
    \caption{Susan as story teller}
    \label{fig:susanstoryteller}
\end{figure}
