Processing sago is hard work. Normally you spend the whole day in the sago swamp. Often the family works together, but in this story it is only the mother and her young daughter who go to the swamp together. It is commonly assumed that children are fairly independent once they are able to walk properly and climb smaller palm trees. They would look for something to eat during the long hours in the swamp. When they were hungry, they would look for small animals such as geckos and frogs and roast them in the fire. Here Margaret Osi recalls her own childhood. Back then, she took pride in catching these small creatures, which would be the only food or sometimes a nice addition to a small portion of sago jelly prepared by her mother.

When I was invited to accompany Susan Bisam to the swamp, she cut down a young palm tree to eat the white flesh at its upper end. It looked like a 30 to 40 cm high cone and tasted fresh. It was a special treat for the children who had come with us.

\begin{figure}
    \centering
    \includegraphics[width=0.75\linewidth]{figures/11Kneading Sago_20250222_0021.jpg}
    \caption{Susan kneading sago pith}
    \label{fig:kneadingsagopith}
\end{figure}
