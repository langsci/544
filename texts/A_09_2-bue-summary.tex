Brigitte Esau mapped out the route through the bush by naming several places that the ancestor Wumeye visited (\figref{fig:sketchomoi}). He set off from the Pual River near Omoi. First he reached the Apilaua hillside and the Pusemo stream. Next he reached the second hill and the hamlet of Wupepp, where he spent the night. This place is mentioned in the story. Then he crossed the Puwalei stream and passed Walili, the last village (or even the last settlement?) before reaching the coast. Now he arrived at the coast at Okuli, where he made camp. People would later boil salt water at this place to extract salt.

Unfortunately, the names mentioned by Brigitte Esau could not be identified on official maps (\cite[Sheet 7192]{Australia.-Army.-Royal-Australian-Survey-Corps:1969fx}). The places were ``in use'' a century ago or even longer. Today they are only recognisable to people who are familiar with all the details of the oral history.

The text confirms the argument that the Kilmeri are an inland people who first had to ``discover'' the coast and the sea.
