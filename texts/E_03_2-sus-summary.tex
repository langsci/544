This is a short account of Susan's life, beginning with her marriage to a man from the Sepik. She focuses on the things her husband was doing without mentioning anything about her own daily activities. Nevertheless, she may have beaten the drum, as the final sequence might seem to indicate. The crocodile carvings on the slit drum (Sequence \ref{ex:nrexuraiga}) show that the crocodile is an essential feature of Sepik culture.

Susan also told her life story in Tok Pisin. This version is longer and talks about the rush on crocodile skins in the area that eventually led to her marrying a man from Angoram (Lower Sepik). Note that Susan tells the story in the third person. The Tok Pisin text is not glossed, but is presented as a parallel text with an English translation below:

\vspace{,5cm}
\begin{Parallel}{0.47\textwidth}{0.47\textwidth}
    \ParallelLText{\noindent \textit{Wanpela de pukpuk i bin kilim wanpela man bilong Ossima. Stori bilong dispela man pukpuk i bin kilim i go olgeta long Sepik. Olsem planti man bilong Sepik i kam long wara Puwani bikos ol i laik kilim pukpuk na salim skin bilong pukpuk. Wanpela man namel long ol dispela man em Arnold. Sampela man namel long ol dispela man i stap long haus bilong Margaret. Susan i stap long haus bilong Margaret tu. Margaret em i anti bilong Susan. Olsem Susan i bungim man bilong en. Tupela i marit. Arnold i painim pukpuk olgeta de. Bihain kilim olgeta pukpuk ol i kisim skin. Skin bilong pukpuk em wait na gre na blak. Skin bilong pukpuk i nogat nil, rat tasol. Taim ol pukpuk i dai pinis long hia em i go long Amanab. Bihain long Imonda, Utai, Wasengla. Bihain tupela i go bek long Sepik. Long ples Angoram Arnold i bin mekim bikpela haus bilong famili bilong en. Haus bilong ol i stap arere long wara Sepik. Arnold i mekim planti kanu. Arnold i mekim planti garamut tu. Susan i gat bikpela gaden long Sepik. Em i planim kaukau, banana, pinat, melen, yam na mami.}}
    \ParallelRText{\noindent One day a crocodile killed a man from Ossima village. This story found its way as far as to the Sepik. So many men from the Sepik came to the Puwani river because they wanted to catch crocodiles and sell their skins. Among them was Arnold. One of the crocodile hunters stayed in Margaret's house. Susan lived there, too. Margaret is a close relative of Susan. That's the way Susan met her husband. The two married. Arnold went hunting crocodiles all the days.  They killed the crocodiles and took their skins. The skins were white, grey, and black. The skins had no spikes, only a few rat bites. Later on, when all the crocodiles in the Puwani were caught, the hunters went on to Amanab. Then they went on to Imonda, Utai, and Wasengla. After the crocodile rush Susan and Arnold went back to the Sepik. In Angoram Arnold built a big house for his family. The house stood close to the Sepik river. Arnold built canoes. He also made many \textit{garamut}-drums. Susan got a big garden along the banks of the river. She planted sweet potatos, bananas, peanuts, melons, yams, and cassava.}
\end{Parallel}