The Kilmeri title of the story, \textit{Walpop bo}, was chosen by Margaret Osi herself. It translates as `story of a small turtle'. The story revolves around a small turtle whose tracks two sisters saw on the riverbank while catching fish. The word \textit{walpop} refers to a small kind of turtle. Such an animal does not have much meat, and therefore provides only a small addition to a meal. Nevertheless, one of the sisters wants to get it, even though they have already filled two baskets with fish. Climbing the often steep bank of a river can be dangerous as one might lose one's hold. The story could therefore be read as a warning not to do anything rash that is not worthwhile. 

Further on, the story features a man who helps the unfortunate sister and takes her back to her village and her family. The generous behaviour of a stranger is culturally surprising. Normally, a stranger would seduce a lonely woman whom he meets by chance in the forest. Here, the contrary is happening.