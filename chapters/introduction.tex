\section{The Kilmeri people} \label{sec:people}

The Kilmeri people live in the area south of Vanimo (Sandaun province) and south of the eastern table mountains of the Oenake range, which are up to 800 metres high (\figref{fig:22tablemountains}). From the village of Ossima -- the place where the fieldwork took place over the years -- you have a beautiful view of these table mountains, which look like real landmarks. They constitute the northern boundary of the Kilmeri area and settlements. These settlements are located in the basin of the Puwani and Pual rivers; the headwaters of the Pual originate in the west, while the headwaters of the Puwani originate on the northern slopes of the Bewani Mountains, which rise above 1500 metres on the Papua New Guinea side of the international border with Indonesia (See the sketch of the villages in \figref{fig:kilmerivillages}). Nowadays there is a gravel road from the coast and the provincial capital of Vanimo to the administrative centre of Bewani at the foot of the Bewani range. Almost all villages of the Kilmeri people are located east of this road. Only Ilup, the most remote village, is situated west of the road, and Sosi, the northernmost village, is situated directly along this road. It can therefore be said that the settlements form a contiguous area. In 1990, when a population census was carried out, the Kilmeri had a population of around 2,500 in 15 villages (cf. \sectref{sec:vitality}). 

\begin{figure}
    \centering
    \includegraphics[width=0.9\linewidth]{figures/22aTableMountainsDämmerung_0008.jpg}
    \caption{Table mountains of the Oenake range in twilight}
    \label{fig:22tablemountains}
\end{figure}

\begin{figure}[t]
    \centering
    \includegraphics[width=0.75\linewidth]{figures/Kilmeri-Villages.png}
    \caption{Map of the Kilmeri speaking villages by Godehard Link}
    \label{fig:kilmerivillages}
\end{figure}

\subsection{Horticultural subsistence}\label{sec:subsistence}

As in the past, horticultural subsistence is a strong economic pillar of the Kilmeri. Close to the villages and hamlets, people create their gardens by slash-and-burn. There are usually several gardens: one or two gardens are more or less ready for harvesting, one is newly planted and one is in the preparation phase. Bananas are often planted also in a plot right next to the house (cf. \textref{sc0820}). The most important crops include the following species: sago (\textit{Metroxylon sagu}) as primary staple food, banana (\textit{Musa}) as secondary staple food; taro (\textit{Colocasia}), breadfruit (\textit{Artocarpus}), and sweet potatoes (\textit{Ipomea batatas}) as valued supplements. These sources of starch are accompanied by greens and coconuts: \textit{aibika} (\textit{Abelmoschus manihot}), \textit{tulip} (\textit{Gnetum gnemon}), and coconut (\textit{Cocos nucifera}). 

Horticulture is accompanied by the gathering of food from the bush, which is equally important to ensure a healthy, sustainable diet. Women collect wild greens and mushrooms on a daily basis. Women also look for nests of wild fowls, cassowaries, and others birds that breed on the ground. When gathering eggs, they tend to empty the nests (cf. \textref{sc0706}). As a result, breeding cycles are interrupted and the number of birds declines in the long term. Women and children also catch small fish in ponds that are hidden in the bush. The size of the fish that I saw was about 15 - 30cm in length. It is not known to me which varieties of freshwater fish live in these small ponds. Furthermore, small frogs were collected after it had rained. After a heavy rainfall frogs appear in large numbers and a bucket fills quickly (cf. \textref{sc0206}). I myself once saw a bucket with dozens of small yellow-greenish frogs, which the family soon started to roast in the fire for a meal. Moreover, frogs and geckos often rest in the deep ribs of sago palms and can be caught by climbing palm trees (cf. \textref{sc0403}). In freshwater streams, men and children collect crabs, which are another small animal food source. Shrimps are a delicious but rare addition to meals. Seasonal bush foods are various nuts and fruits gathered by groups of children. The fruits of the \textit{ton} tree (\textit{Pometia pinnata}) are particularly valued. They are a bit akin to rambutan fruits and lychee fruits.\footnote{The species \textit{Pometia} belongs to the same family \textit{Sapindaceae} as rambutan and lychee trees.}

Cultivating sago grubs is done by all families who have access to sago swamps or other places where palms grow that feed grubs (cf. \textref{sc0301}). The production of sago grubs is part of the sago business: It supplies the protein, while the sago pith provides the starch. A typical meal consists of sago jelly, \textit{tulip} vegetable or \textit{aibika} vegetable cooked in the milk of grated copra. Mushrooms, sago grubs, or some meat are also served, depending on availability (cf. \textref{sc0402}, \ref{sc0403}, \ref{sc0708}, \ref{sc0711}). 

\newpage
Hunting animals used to be the traditional task of men: wild pigs, ground kangaroos, tree kangaroos, possums, and cassowaries were hunted. Once in a while, someone spotted a deer, and then everyone was eager to get it. But success was rare, as I experienced it in the field. Bandicoots could be caught by hand (\textref{sc0813}) when a piece of overgrown garden was cleared. The story \textit{Ai kopi basuiko} (\textref{sc0401}) is a good example of the former abundance of forest animals. But the story can also point to the danger of excessive hunting (cf. \textref{sc0303} and \ref{sc0812}). In addition, (pregnant) female animals all too often become prey (\textref{sc0407}; p.c. Anita Osi as a sad comment on sharing the meat of a kangaroo with me, in whose pouch there was a young animal). Hunting was supplemented by semi-domesticated pigs, i.e. young wild boars were caught and fed (\textref{sc0605}).

\subsection{Cash economy}\label{sec:casheconomy}

Nowadays, garden crops and sago are often produced beyond subsistence and sold for money at markets. In Ossima there was and still is the Sunday market, which flourishes after the weekly Sunday service. Women from the neighbouring settlements come to this market to earn some money. Customers include the school's teachers and the parish with its staff, as well as guest families like me and my family. But the villagers would also buy goods that they currently lack. Quite a few people produce for the big markets in Vanimo and Dasi. Demand is high, especially for sago flour, as the squatters in Osi Camp (on the outskirts of Vanimo) have no access to the local sago swamps near the coast. The local sago palm grounds belong to the people of the coastal villages. 

The establishment of the Catholic mission in Ossima in 1961 was accompanied by a cattle breeding project. Gradually, many people became cattle owners. The mission farm ran a small abattoir where the cows were slaughtered in order to bring the meat to the supermarkets in Vanimo. This provided some income for several families. The mission's cattle, pig, and vegetable farm also provided paid work. For about thirty years, this was a valued opportunity for young men to receive some education in agriculture. At the beginning of the 1990s, however, the farm gradually began to decline. When I arrived in Ossima, it was still in operation, but its heyday was over. By 2024, the farm business is almost at an end. 

Another source of income are the royalties from the logging concessions. The primary forest of the entire Kilmeri area was certified for logging, and the logging company designated six so-called blocks, which were logged one after the other (\cite[9-11]{Gerstner-Link:2018un}). The valuable trees were cut down with chainsaws and moved with huge bulldozers. I witnessed this myself and even had the opportunity to fly over an area where the trees had been dislodged. At times, one could hear from my house the sound of a huge tree falling over. Logging also offered young men paid jobs, but often at the cost of serious injuries due to a lack of safety measures.

\begin{figure}
    \includegraphics[width=\linewidth]{figures/Ossima-Village.png}
    \caption{Map of Ossima village by Godehard Link}
    \label{fig:ossimavillage}
\end{figure}

In 2010 the company ``Bewani Oil Palm Plantations Limited'' started their palm oil business in the Kilmeri and East Pagi areas. For this purpose, a road was built from Vanimo to Imbio/Imbinis, a half-day's walk east of the Kilmeri villages. The road follows the coast and crosses the Puwani River at its mouth into the sea; previously the Puwani had been an obstacle to industrial development. The former road to Ossima ended on the north side of the river and one had to cross it on foot or by boat/canoe to reach Ossima and the mission, including the school there. The BOPPL oil plantation also offers paid jobs. After remaining in isolation, a new centre is now developing around the plantation in the East Pagi area. 

Last but not least, some descendants of the old Kilmeri have also settled in Vanimo. Over the years, the provincial capital offered more and more job opportunities. Simultaneously, the Kilmeri youth had received the necessary education to work in town. So there is a certain backflow of money from the town to the villages.

\section{Clan lands}\label{sec:clanland}

The Puwani-Pual basin is not the old homeland of the Kilmeri. There is good evidence that they migrated into this area from the west along the Bewani river valley. Their homeland was probably the area south-west of the Sentani Lakes on the other side of the state border with Indonesia, which more or less borders on the territory of the Nimboran people and the Nimboran language family. This hypothesis is outlined and linguistically argued for in (\cite{Gerstner-Link:2023lo}). It corresponds to the genealogical memory of the Kilmeri, who count their generations back to their shared ancestor Si who is said to have come to the area from outside and claimed the land. Margaret Osi from Ossima and Usi Kul from Omoi are aware of this oral history. The sons of Si are Bu and Nakei, who are the ancestors of the people of Ossima and Omoi, respectively. They lived nine generations back, calculated from the year 2000. These genealogical memories are identity-forming. 

Near the eastern Kilmeri villages there are (or were) several remarkable old Kwila trees (\textit{Intsia bijuga}) that are considered ancestral heritage. These trees have names that relate to the ancestors. For example, near Oiru there was a tree called \textit{Isimu}, which is said to have been named by Si. It burned down in 1999; its interior had completely dried out. Near Ossima stood a Kwila tree called \textit{Ppaimu}, which was sacrificed for the construction of the ``highway'' -- a dirt road roughly north of the Pual River from Kiliwes to the eastern villages -- although many people wanted to preserve it. The Kwila tree \textit{Ploumu} close to Omoi is still standing. Near Omula, the easternmost village, there was another tree called \textit{Noumu}. It toppled over because of a landslide. Finally, on top of the hill Oimu located between Oiru and Omoi a Kwila tree named \textit{Mu} rises. Note that \textit{Mu} is a phonetic variant of \textit{Bu}, emerging from prenasalisation of the plosive. In fact, all these tree names contain the syllable \textit{mu}, which refers to the ancestor Bu. One of the Kwilas even contains the names of both ancestors, Si and Mu/Bu, namely the tree \textit{Isimu}. According to Margaret Osi the name bearing Kwila trees are a certain subtype of \textit{Intsia bijuga}: The Kilmeri name for this species is \textit{ri maro} `tree maro', but a slightly different type of \textit{Intsia bijuga} is called \textit{ri mu} `tree of Bu'. The labelling of certain trees with the names of distant ancestors is a means of preserving clan history and securing land claims. These trees can reach an age of more than 200 years, so it is possible that the aforementioned trees were planted by the aforementioned ancestors and have been revered for generations and centuries. 

The names of the clans that occur in the area also provide a good clue to the population and settlement history of the Puwani-Pual basin. The clan names are linked to the clan settlements, but today's villages and hamlets have different names from the clan names. Table \ref{clannames} presents a list of ancient clan names found in the Puwani-Pual Basin, together with their administrative equivalents as found on official maps (\cite[Sheet 7192]{Australia.-Army.-Royal-Australian-Survey-Corps:1969fx}). When asked where they live, people name their village with the (new) administrative names.

\begin{table}
\caption{Correspondence of clan names and settlements}
\label{clannames}
    \begin{tabularx}{\textwidth}{llX}
	\lsptoprule
        \textsc{clan}  &  \textsc{village names}  &  \textsc{comment} \\
    \midrule
        \textit{Ilá}  &  Kilmeri, Kiliwes, Ossima  &  The name \textit{Ilá} refers to these three villages \\
        \textit{Ilup}  &  Anau  &   \\
        \textit{Imar}  &  Ossima  &  Greater area of Ossima around the village proper \\
        \textit{Imiri}  &  Osol  &   \\
        \textit{Ipies}  &  Airu/Oiru  &  The name of the village comes from two phonetic varieties \\
        \textit{Irer}  &  Omoi  &   \\
        \textit{Isa}  &  Krisa  &  The inhabitants of Krisa speak the Sko language Isaka \\
        \textit{Isep}  &  Kilipau  &   \\
        \textit{Isi}  &  Isi I, Isi II, Esau/Olul  &  The third village is known under two names \\
        \textit{Isipi}  &  Omula  &   \\
        \textit{Iulep}  &  Awol  &   \\
        \textit{Iwes}  &  Kiliwes  &   \\
    \lspbottomrule
\end{tabularx}
\end{table}

I would also like to mention that \textit{kili}, which appears in several clan and village names, is a Kilmeri word meaning `bone'.\footnote{The name \textit{Kilimeri} is not a village name nowadays. Its components, \textit{kili} `bone' and \textit{meri}, can shed light on its origin. The component \textit{meri} relates to Nimboran \textit{méndy} `mouth' or \textit{mendú} `skull'. The sound correspondence /nd/ <> /r/ is regular and attested within the Border family as well as between Kilmeri and non-Border languages (cf. \cite{Gerstner-Link:2023lo}). In Nimboran society \textit{méndy} was a gift that the parents of a newly married woman received from the husband's family after the birth the first child, thus, acknowledging her fertility (\cite[25]{Kouwenhoven:1956mg}). The naming of a new settlement with a phrase meaning `bones (for the) mouth' points to the fact that there was plenty of game which would secure the livelihood and fertility of the people living there. Moreover, there is the story of the hero Sakou, who throws out bones and seeds, thus providing the people with an abundance of game and crops (cf. \textref{sc0203}). The village name \textit{Kiliwes} consists of the words \textit{kili} `bone' and \textit{wîs} `moon'. The moon is also a reference to hunting, as the great hunts in the forests used to take place during the full moon phase (cf. \textref{sc0306}, Sequence (\ref{ex:nrexpulwis}).} In addition to those clan names that correlate with settlements, there are further clans with similar names in the region: Iles, Imo I, Imo II, Imo III, Imo IV, Imop, Inuges, Ipualu, Iu, Iuwi, Iwom, and Iwopai.\footnote{\citet[27]{Kouwenhoven:1956mg} states about Nimboran clans -- which are called ``tang'' -- that a clan or ``tang'' subdivides when it has exceeded a certain size. The same seems to hold in Kilmeri society, an example being the Imo clan, which now has four divisions.} 

Evidently, all the clan names listed here begin with the vowel /i/, which is by far the most common in the Kilmeri lexicon (\cite[63]{Gerstner-Link:2018un}). Since *i- is the class marker of flying animals, i.e., birds and bats (\cite[647]{Gerstner-Link:2018un}), one could speculate that animals of these groups might have had a special spiritual function for the clans. Furthermore, in Nimboran the word for bird is \textit{iý} (\cite[14]{Anceaux:1965ct}). Probably it is not too far-fetched to think of a transfer relation, which would have been possible in either direction (cf. \cite[654]{Gerstner-Link:2018un}, \citeyear{Gerstner-Link:2023lo}). However, there are also clan names that do not begin with the vowel /i/, but have completely different phonological forms. The ratio of documented clan names appears to be roughly half /i/-initial names and half other names.

\begin{figure}
    \centering
    \includegraphics[width=0.85\linewidth]{figures/PuwaniRiver_0004.jpg}
    \caption{Puwani River}
    \label{fig:puwaniriver}
\end{figure}

\section{Language vitality and literacy}\label{sec:vitality}

\subsection{Speakers and fluency}\label{subsec:speakers}

In the beginning of the 1990s, the Kilmeri comprised around 2,200 people in 15 Kilmeri-speaking villages (\cite[107]{Nekitel:1998gy}, based on \cite{Grimes:1990fm}). The National Population Census of 1990 states a population of 3,607,954 people for Papua New Guinea minus the North Solomon Province (\cite[81]{Nekitel:1998gy}).\footnote{This census is also mentioned on the website of the National Statistical Office, but without numbers.} According to the National Statistical Office, the population of Papua New Guinea reached 7,275,324 in 2011, and 11,781,559 in 2021.\footnote{Cf. \url{https://www.nso.gov.pg/statistics/population/}} If we take the national census as a guide, with an average annual population growth of 3.81 per cent between 1990 and 2021, and apply this to the population of Kilmeri in 1990, which was 2,200, then by 2021 the population would be around 7,000. From my own experience, I can add that I met several families in Ossima who had around seven to ten children.\footnote{Margaret Osi told me that in the past, when the men's house still existed, women had no more than four children. Only when a child was about the age of four the next pregnancy was agreed on. Albert Maori Kiki says that in his mother's group of the Elema people at the Purari River (Gulf Province) women were supposed to have only two children (\cite[30]{Kiki:1969tx}). Apparently, in traditional times, birth control was a behaviour that was accepted as it was considered to balance the life and the resources of people.}


Most likely, the 15 Kil\-meri-speaking villages would still be referred to as ``Kil\-meri speaking'' by the people themselves, as the language is identity-forming to a certain extent. However, the number of fluent speakers has decreased considerably. The last generation, who spoke Kilmeri fluently as their only or dominant language, are no longer alive. The middle generation, born in the 1960s, 1970s and 1980s, grew up speaking Kilmeri but switched to Tok Pisin in large numbers. They used Kilmeri only to talk to the older people. This generation has a reasonably good command of the language. The younger generation has grown up bilingual and has mostly heard (some) Kilmeri from the adults, while speaking Tok Pisin. Many of them have Tok Pisin as their first language. It should also be noted that the middle generation often married spouses from outside the Kilmeri community. To give an example: Lis Osi had seven children, and three of them are married to non-Kilmeri partners: one son is married to a woman from Popondetta (Oro province; they live in Port Moresby); another son is married to a woman from Vanimo Lido; another daughter is married to a man from Krisa. This pattern of mixed marriages continues in the younger generation. Furthermore, this can certainly be generalised to other families. For the vernacular Kilmeri, this marriage pattern breakes with the tradition and has the dire consequence that the language quickly loses ground. The common language within the family is often Tok Pisin and is therefore intruding all situations. In addition to what has been said, many Kilmeri families squat in Ossima Camp located in the outskirts of Vanimo. Almost every large family has such a base in the city to stay overnight on business or just ``to be in town.'' The language of Vanimo is of course Tok Pisin.

My impression is that Kilmeri can no longer be described as a living language. I do not have exact figures on speakers and their language proficiency, but the overall picture is not encouraging. Even if the language is used in daily life by a few families in about a third of the Kilmeri villages, this situation does not guarantee the language will survive. Yes, some sentences and words will survive, produced now and then as short utterances with simplified grammar. But unfortunately this is not the language that I analysed in my grammar (\cite{Gerstner-Link:2018un}) and that can be found in the stories in this collection.

\subsection{Literacy and Language Shift}\label{subsec:shift}

Until the 1960s the Kilmeri speaking community fostered their oral language tradition in daily and ceremonial life without being literate. Western education with schools and literacy training was first introduced by the Catholic Mission, which soon founded a school in Ossima. In 1971 Grade 1-6 were in operation; the school had an enrolment of 200 children with boarding facilities for 180 children.\footnote{Cf. (\cite[20]{Patrol-Reports:1971ud}) and (\cite[8]{Patrol-Reports:1971il})} The language of instruction was English. Those children were the first generation to become literate. The older generation, born in the 1940s and 1950s, has remained largely illiterate. However, literacy was slow to spread and even in the 1990s not all children went to school regularly. Often the children were already 8 to 10 years old when they started school, and some left school after the fourth grade. The most successful students went on to high school and could choose between the high schools in Vanimo, Lumi, Green River and Aitape, which are all boarding schools with grades 7-10. In addition, Aitape Secondary School also offers grades 11 and 12.

Although the official language of instruction remained English after independence in 1975, Tok Pisin became the de facto language of schooling. This is also due to the fact that many teachers in the Vanimo-Bewani region and throughout the province come from other parts of the country and their common language is Tok Pisin. Even the local teachers are often not fluent in their local language.\footnote{Cf. (\cite[19]{Sandaun-Provincial-Government:2007ix})} This means that the indigenous Kilmeri language is not supported at school. As a consequence, the younger generation has become more and more convinced that the local language is of no practical use for them. It will not contribute to a livelihood better than the subsistence farming practised in the villages. I myself have heard this judgement several times.

One should also mention that the language favoured by the mission staff is Tok Pisin. Hardly anyone took up the challenge of learning Kilmeri, apart from a few phrases or short songs. The liturgic texts were all in Tok Pisin, including the Bible translation. The High Holidays were celebrated with annual biblical performances, the Passion Plays and the Christmas Plays, in which many people from the villages took part. These performances were held in Tok Pisin. Church life was based on Tok Pisin from the very beginning and is still based on this language today. 

This is the background to the language shift of the Kilmeri people who received education in Ossima mission and became members of the church. In the villages surrounding the mission people shifted to Tok Pisin, and it was not only used as a lingua franca, but also within families. In Osi Camp, for example, the household language was already Tok Pisin, when I arrived in November 1999. We found the same situation in Ossima Asples, Omoi and Airu. The village of Awol, on the other hand, was more remote and only accessible on foot, so Kilmeri was not pushed out as much. My daughter had a schoolmate from Awol who visited us at home. To some degree, she grew up speaking Kilmeri, as her family spoke the language at home. 

When reading a chapter of Otto Nekitel's book ``Voices of yesterday, today and tomorrow'' one repeatedly comes across the linguistic scenario that would feed the society of Papua New Guinea best (\citeyear[46-61; 78-90; 169-182]{Nekitel:1998gy}). Clearly, a country needs a common lingua franca that can be used at all levels of private and public communication. This would be Tok Pisin. Secondly, the vernacular languages should not be banned, neither by individuals nor by groups of people. Ideally, the family would communicate in the parents' vernacular -- or maybe two vernaculars in the case of mixed marriages -- and the children would retain their mother's language even if leaving their home for higher education. That could result in a widespread creative bilingualism that preserves the old culture and meets the necessities of modern life and mobility. English as a possible third language would serve academic communication. At least in theory, language shift and language preservation are not mutually exclusive. 

The secular cultural change also led to the abandonment of the indigenous language. (i) Film screenings: During my last field visits in 2006 and 2007, public film screenings were organised in Ossima for evening entertainment. A medium-sized screen was set up on someone's plot of land, and the audience -- around 20-30 people -- sat in front of it on the ground or on seats made from felled coconut palms. The adults watched American, Indian, or Chinese blockbusters, and the children watched Chinese cartoons. I did not witness the introduction of smartphones. Perhaps film watching has now become a form of private, individual entertainment. (ii) Dancing: Traditional dancing has disappeared entirely. Instead, dancing now takes the form of \textit{banis}, which is a Tok Pisin term for 
`a fenced-off place where people dance to Western music'. Several \textit{banis}-grounds were set up during my time of field research. \textit{Banis} is a popular weekend entertainment from Friday night to Sunday morning. The dancing usually starts in the evening and goes on until morning, and people rest during the following day. \textit{Banis} is commercialised and people have to pay an admission fee. The owner of the \textit{banis} site earns a good income despite his expenses for loudspeakers and other technical equipment. 

Against this background, records of oral traditions are crucial to the ethnographic understanding of peoples worldwide. In the Kilmeri region, the people who ``own'' stories will be gone in the very near future. Margaret Osi, for example, celebrates her 82nd birthday in 2024, and my landlord Jeffrey Osi, who certainly knows a few old stories, is now in his sixties. It is not certain whether he succeeded in passing on ``his'' ancestral stories to the next generation. 

\chapter{The topics of the narratives}\label{sec:topics}

The texts of this collection are grouped into seven topical chapters. The first four chapters and their texts describe the traditional life of the Kilmeri people. They address clan history, ancestral stories, and traditional activities and experiences. The following three chapters describe contemporary village life, including events I have witnessed in around half of the stories.

The texts go back to the inhabitants of four villages or hamlets, namely the settlements of Ossima, Isi Camp, Omoi, and Awol. The village of Ossima consists of three settlements, namely Ossima Asples at the Puwani river, Ossima Station, and Osi Camp. Ossima station consists of the houses of the mission staff, the houses of the school teachers, the community buildings, the school, the church, and a few private houses. However, most of the people from Lis Osi's lineage live in Osi Camp, about 300 metres away, where I also had my house. This settlement consisted of five families at the time of the field research, but more houses may have been added, as the descendants of Jeffrey Osi have since founded their own families. Ossima Asples is the settlement of the Bisam lineage of the Imo clan. It consists of about ten houses with six to eight families, including the families of Susan Bisam and Andrew Wapi, two knowledgeable storytellers. Some people live apart from Ossima Asples on the ridge above the river towards Asue, in poorly built houses. Isi Camp is a small settlement of about five families (and some more houses) who came here some years ago from one of the Isi villages located further west near the Bewani road that connects the administrative centre of Bewani with the capital of Vanimo at the coast. The people of Isi Camp belong to the greater Imo clan and are thus entitled to live on that strip of land near Ossima. Like the Ossima people they are cattle owners. 

The village of Omoi lies north of the Puwani River opposite Ossima Asples and stretches along the river, but a second settlement lies uphill, about 15 minutes' walk from the river. This is where the storytellers Usi Kul and Brigitte Esau had their houses. The village of Awol is located on a ridge above a large sago stand south of the Pual River and west of the Puwani River. Sei Walup, who ``owned'' one of the stories of the clan history, lived there. Some inhabitants of Awol settled in Warabung, the confluence of the Puwani and Pual rivers, but no one from there was involved in the language documentation.

Seven people have contributed to this text collection, and as they come from different villages and different ``microcultures,'' the collection covers more than one narrative perspective and narrative heritage within the Kilmeri community.

\section{Ancestral stories}\label{subsec:ancestral}

There is no vernacular Kilmeri word for this genre. Although everyone knows exactly what kind of story is meant, they always refer to it with the Tok Pisin term \textit{stori tumbuna}, which means `stories of our ancestors'. Sometimes the opening formula of a story contains names of family members or a clan. All texts relate to characters rather than to an abstract literary concept, which may nevertheless exist implicitly.

Many of the ancestral stories are about bush spirits. In fact, the bush spirits are the main protagonists, and without them these stories would not have become a piece of oral history and collective identity. For the Kilmeri, I think, the world of the bush spirits forms a parallel world to that of humans. At least the stories are crafted in this way. The abstract structure is as follows: (1) One or sometimes two people leave the settlement and go into the bush or to a river to find food. (2) At this place, a bush spirit appears, which takes the form of a human or an animal. (3a) If it is a female, the bush spirit seduces her to follow it or (3b) if it is a man, the bush spirit kills him on the spot. (4) In revenge, the bush spirit is trapped in its house and killed by the people there.

In general, the bush spirit can change its appearance freely between an animal and a human being. If it takes a human by surprise outside the settlement, it can appear as a snake, goanna, crocodile, turtle or cassowary. In contrast, a bush spirit never appears as a pig, wallaby, possum or bandicoot. This means that mammals and marsupials are excluded from entering the world of bush spirits as agents of evil. This can be understood in terms of an animal hierarchy in the narrative structure.

The habitat of bush spirits is usually underwater, as they are thought to live at the bottom of bodies of water. This becomes evident in the stories when the people set off to take revenge, and visit the bush spirit at its home. It is usually a house, built in exactly the same way as human houses, with doors and ladders (\textref{sc0303}, Sequence \ref{ex:18exwalp13}) and a roof (\textref{sc0206}, Sequence \ref{ex:nrexkeriep}). The bush spirit's house is equipped with the very same utensils that humans need to live (\textref{sc0302}, Sequence \ref{ex:nrexomosop}).

The human characters of these stories usually remain unnamed. Instead, they are described as ``a couple,'' ``two sisters,'' ``two brothers,'' or as ``father and son.'' It is obvious that the protagonists represent stereotypical social pairings and are not individualised. This is mirrored in people's preference for addressing each other with the appropriate kin terms. I also observed the avoidance of personal names during my fieldwork. My husband, for example, was always called \textit{poro} `friend' by our Kilmeri friend and landlord Jeffrey Osi. Children, on the other hand, were often addressed by their names, preferably by their Kilmeri names. In the ancestral stories, only a few characters bear names, namely the man Kopukei in \textref{sc0301}, the man Wau in \textref{sc0310}, the man Bipep in \textref{sc0303}, and the girl Kusudua and the bush spirit Mawatkawi in \textref{sc0304}. The individualisation of a bush spirit through a personal name is interesting, but this topic was not pursued further. In a few texts friendly spirits appear, for example in \textref{sc0206} and \textref{sc0308}. They reside at the bottom of bodies of water, just like the evil spirits (or \textit{masalai} in Tok Pisin). It seems that they appear in human form, but one should be cautious as the narrative data on friendly spirits is sparse. 

In addition to evening entertainment, the stories have two further functions. Firstly, they have the educational aim of warning people about the dangers of the bush. Although in most stories two people set off together to find hunting animals, one of them usually falls victim to the bush spirit. So even two pairs of eyes cannot withstand the challenges of the bush. Secondly, the stories describe the consequences of social misbehaviour within the family; in particular, they describe the fact that some male members are too greedy and unwilling to share food with others, be it with their younger brother or with their wife, as in the Texts \ref{sc0303} and \ref{sc0304}. These disadvantaged family members then fall prey to the bush spirit because they wander off alone to find food in the bush.

\section{Old village life}\label{subsec:oldvillage}

Five texts in this chapter recall Margaret Osi's experiences and memories of her life as a child and young woman. Some are more narrative in nature, others could also be classified as procedural texts, in particular \textref{sc0404} about constructing a new family home after the old house had fallen into disrepair.\footnote{Barry Craig (South Australian Museum) undertook a survey of housing in central New Guinea to be found under (\cite{craig2018}). His description and photos give a valuable impression of types of house and house building back in the 1960s. The description of Margaret Osi matches Craig's findings in the Upper Sepik area in many aspects.}

In all cases, it was Margaret's expressed wish to represent an integral part of her earlier life as well as the earlier life of her people. Compare the two texts that deal with the death of a person. \textref{sc0401} about Margaret's father Apai ends with a (more or less) Christian burial, whereas \textref{sc0405} describes the traditional burning ritual performed in the bush after a person's death. The ritual of mourning and exchanging gifts is not mentioned in the texts, although this custom is still alive. It was still practised at the time of my field research.

Two other stories in the chapter were told by Andrew Wapi, and they describe hunting events in the bush. Andrew owned a large bow and was known as a skilled bow maker.

\section{Procedural texts}\label{subsec:procedural}

The texts in this chapter are all by Margaret Osi, with the exception of \textref{sc0508}, which lists the activities involved in sago processing and was told by Susan Bisam. Presumably, this genre of text did not have a place in traditional Kilmeri life, since all the necessary steps for producing a certain item of material culture were learnt by doing rather than by oral description. Again, the texts were produced to inform me about things connected with traditional life. The production of grass skirts and their colouring as well as the production of phallocrypts used to be common, but have been replaced by Western clothing for almost 50 years. Cooking in bamboo tubes has also been abandoned and people use tin pots nowadays. However, the brooms are still produced as described in \textref{sc0506} (See \figref{fig:broom}). The same applies to the processing of sago, as sago pudding or pancakes are still the staple food. These have not been replaced by rice because it is simply too expensive, and so rice is only a supplement to the old sago-based diet.

\section{Autobiographic texts}\label{subsec:autobiography}

Recollecting and narrating one's own life is an artificial undertaking for the Kilmeri people. Yet is was mastered by all three consultants who were asked to do so. Susan Bisam chose Tok Pisin to tell the main episodes of her life (\textref{sc0603}). Although she delivered a Kilmeri version, which is included in this collection, it is a rather rudimentary story. Andrew Wapi produced a longer narrative (\textref{sc0604}) that comprises his childhood and the main steps of his life.
 
The third life story of Margaret Osi can be called a masterpiece both in depicting the central episodes of her life and in her Kilmeri language style (\textref{sc0605}). In addition, her admiration for the mission becomes evident in her recollection of the agricultural manager's grief at the death of her husband, Lis Osi (cf. \textref{sc0602}).

In 1968 the autobiography of Sir Albert Maori Kiki (1931-1993) appeared. He was one of the founders of the Pangu Pati of Papua New Guinea, whose activities eventually lead to independence. Kiki's book was titled ``Ten Thousand Years in a Lifetime'' (\citeyear{Kiki:1968tx}). A similar headline could be given to the lives of Margaret Osi, Susan Bisam, and Andrew Wapi. Despite the impact of World War II they lived a childhood that was entirely traditional without any Western goods and ideas. In the region west of the Sepik, Australian influence was low. Had Margaret, Susan, and Andrew known about their prominent predecessor, they would have been even more keen to let know a small auditory in and beyond Papua New Guinea about their lives and their contribution to history. 

Both Andrew and Margaret grew up with relatives. Such an arrangement was very common in earlier times due to the numerous early deaths. So the two orphans in \textref{sc0206} were a common occurrence only a few decades ago.

\section{Contemporary village life}\label{subsec:contemporary}

The chapter is a collection of stories and reports about happy and less happy events during the time of fieldwork. Again, Margaret Osi is the main storyteller, which reflects her talkative nature and her ability to tell long stories about any kind of event. Every time I returned to Papua New Guinea and to Ossima, Margaret was eager to tell me about the most important events. She did this for the fun of telling stories, but also to create a common ground about social incidents for the time of fieldwork. Although many more stories were told in Tok Pisin, she wanted to embed specific stories in the `language of the place' (Tok Pisin: \textit{tokples}) in order to frame them appropriately. Following the universal human urge, stories of illness and accidents were told for the most part, including accidents in which I myself was involved.

\section{Episodes of daily life}\label{subsec:dailylife}

This chapter is a collection of short episodes that Margaret Osi remembered by association. It may be a special experience of hers, such as her flight to the village of Green (\textref{sc0801}), or she may have wanted to inform me about certain habits in the village and in the bush. The special value of these short episodes lies in their high degree of spontaneity, as these contributions were not planned in any way. In contrast, the long narratives in other chapters were explicitly planned in advance and scheduled for a particular session or day. Moreover, these episodes reflect daily life in its incidental occurrences. The topics of these stories are completely random.

\section{Overview of the texts}

\tabref{tab:texts} provides an overview of the texts included in this collection. Note that the word count excludes the three texts that were narrated in Tok Pisin (\ref{sc02tp1}, \ref{sc03tp2}, \ref{sc03tp3}).

\begin{xltabular}{\textwidth}{llXr}
    \caption{The texts in this collection.} \label{tab:texts}\\

    \lsptoprule
        \textsc{text} & \textsc{title} & \textsc{topic} & \textsc{words}\\
    \midrule
    \endfirsthead

    \multicolumn{4}{c}
    {\tablename\ \thetable{} -- continued from previous page}\\
    \lsptoprule
        \textsc{text} & \textsc{title} & \textsc{topic} & \textsc{words}\\
	\midrule
    \endhead

    \hline \multicolumn{4}{r}{{Continued on next page}}\\
    \endfoot

    
    \endlastfoot

		\ref{sc0201}&\textit{Am} &Genealogy of an Omoi clan&413\\
        \ref{sc0202}&\textit{Si yelo piyo} &Genealogy of the clan of Lis Osi&166\\
        \ref{sc0203}&\textit{Sakou}&A creation myth of some Kilmeri clans&838\\
        \ref{sc0204}&\textit{Haus tambaran}&Remembering the times past away&206\\
        \ref{sc0205}&\textit{Wapues}&Marriage ties between the villages of Ossima and Omula&374\\
        \ref{sc0206}&\textit{Pu ppulae}&A massacre of a clan, revenge against the leader, and a tabooed lake&373\\
        \ref{sc02tp1}&\textit{Muruk}&A massacre of a clan, committed by a cassowary man&--\\
        \ref{sc0207}&\textit{Ome na Lapi}&An earthquake caused by social misbehaviour&120\\
        \ref{sc0208}&\textit{Bue}&Reaching the sea and getting the taste of salt&86\\
        \ref{sc0301}&\textit{Bermepu}&An encounter with a bush spirit&277\\
        \ref{sc0302}&\textit{Urual bekulu}&An encounter with a bush spirit&617\\
        \ref{sc0303}&\textit{Walpop bo}&An encounter with a bush spirit&573\\
        \ref{sc03tp2}&\textit{Masalai piaune}&An encounter with a bush spirit who is successfully tricked by a human&--\\
        \ref{sc0304}&\textit{Kukumbina na Kusudua}&An encounter with a bush spirit&262\\
        \ref{sc0305}&\textit{Pu paek}&An encounter with a bush spirit&404\\
        \ref{sc0306}&\textit{Ruri onona ruri pialna}&An encounter with a bush spirit&402\\
        \ref{sc0307}&\textit{Urai ako wiye}&An encounter with a bush spirit&279\\
        \ref{sc03tp3}&\textit{Masalai pukpuk}&An encounter with a bush spirit&--\\
        \ref{sc0308}&\textit{Nana puyo seku}&An encounter with a friendly bush spirit&176\\
        \ref{sc0309}&\textit{Wîs yako}&A moon story&386\\
        \ref{sc0310}&\textit{Wîs puli}&A moon story&88\\
        \ref{sc0821}&\textit{Nini na wîs}& A song about the sun and the moon&18\\
        \ref{sc0311}&\textit{Bike iwanyo}&Story about the cassowary and the hornbill&67\\
        \ref{sc0401}&\textit{Ai kopi basuiko}&The death of Margaret Osi's father&233\\
        \ref{sc0402}&\textit{Ruri Epek}&A baby obsessed by a spirit in a sago swamp&198\\
        \ref{sc0403}&\textit{Ko lelo piu no}&Finding food in the sago swamp&281\\
        \ref{sc0404}&\textit{Ai kopi yip papi}&Traditional house building&405\\
        \ref{sc0405}&\textit{Ono basuiko}&A man's death and the burning of the corpse&316\\
        \ref{sc0406}&\textit{Urual}&A hunting story&168\\
        \ref{sc0407}&\textit{Diri wor dop lo}&A hunting story&260\\
        \ref{sc0501}&\textit{Ber papi}&Making a phallocrypt&86\\
        \ref{sc0502}&\textit{Die papi}&Making a grass skirt&151\\
        \ref{sc0503}&\textit{Die aeppu pi}&Dying a grass skirt&55\\
        \ref{sc0504}&\textit{Ko oil pi}&Making oil from coconuts&96\\
        \ref{sc0505}&\textit{Yaup ulyo moli}&Boiling water in bamboo tubes&120\\
        \ref{sc0506}&\textit{Kos papi}&Making a broom&39\\
        \ref{sc0507}&\textit{Yûr lui}&Shooting birds&53\\
        \ref{sc0508}&\textit{Due dû papi}&The steps of processing sago&38\\
        \ref{sc0509}&Female fertility&Some pieces of information dealing with female fertility&54\\
        \ref{sc0601}&Margaret Osi I&First autobiography of Margaret Osi&39\\
        \ref{sc0602}&Margaret Osi II&Second autobiography of Margaret Osi&135\\
        \ref{sc0603}&Susan Bisam&Autobiography of Susan Bisam&46\\
        \ref{sc0604}&Andrew Wapi&Autobiography of Andrew Wapi&158\\
        \ref{sc0605}&Margaret Osi III&Third autobiography of Margaret Osi&582\\
        \ref{sc0701}&\textit{Urai ikoiele}&An incident caused by a crocodile&362\\
        \ref{sc0702}&\textit{Bi dupua luwe}&Killing two pigs&130\\
        \ref{sc0703}&\textit{Bo Helenpiro}&The death of Helen Osi&265\\
        \ref{sc0704}&\textit{Ul ko lu}&An incident in the sago swamp&212\\
        \ref{sc0705}&\textit{Ko kipino ye}&An incident when coming back from the sago swamp&108\\
        \ref{sc0706}&\textit{Ko kau yek}&An incident caused by a cow&290\\
        \ref{sc0707}&\textit{Yipp pol}&Finding the nest of a wild fowl&100\\
        \ref{sc0708}&\textit{Ko Vanimoyo lo}&Foot walk from Ossima to Vanimo&146\\
        \ref{sc0709}&\textit{Sû duki}&Story about different types of torch lights&139\\
        \ref{sc0710}&\textit{Claudia ikoina nomari}&Claudia's sickness&311\\
        \ref{sc0711}&\textit{Bo Milipiro}&Mili's surgeries in Vanimo hospital&616\\
        \ref{chap08}&Episodes of daily life&Collection of 21 short texts expressing Margaret's experiences, perceptions and reflections on her daily village life&874\\
    \midrule    
        \textbf{total}&&&\textbf{13191}\\
    \lspbottomrule
\end{xltabular}

\chapter{The narrators}\label{sec:narrators}

\section{Margaret Kai Apai Osi}\label{subsec:magaretkai}

Margaret is known by four names, and her names give already a view into her life. \textit{Margaret} is her Western name of baptism. \textit{Kai} is her native name and the one used by her parents. \textit{Apai} is the name of her father. \textit{Osi} is the family name of her husband Lis Osi. Nowadays, she calls herself simply Margaret Osi.

Margaret is the only surviving child of her mother Puma, who died early, but she has five half-siblings from her father's second wife Es. She said she was born in 1942. She now lives in Ossima, but her father is originally from Ninggera and left the area to settle in Omoi. So Margaret grew up in Omoi. Her knowledge of the Ninggera language comprises only a few words and it was therefore not possible to obtain any significant information about this language from her. Her father returned to Ninggera at some point in his life, presumably with his second wife. This is mentioned in \textref{sc0401}, which is about the death of her father, as the fatally ill man was brought back from the bush and mangrove swamps in Ninggera to Vanimo and to Ossima. In fact, one of Magaret's brothers, Jack, still lives in Ninggera. It is interesting to observe that a family can be spread not only across villages, but also across languages.

Margaret has no formal schooling. She was already 19 years old when the mission was founded in 1961. However, she is passively literate and can read the Bible in Tok Pisin. Unfortunately, she never learned to write and, thus, could not assist in the development of the Kilmeri orthography.

\begin{figure}
    \centering
    \includegraphics[trim=0 18 0 0,clip,width=0.6\linewidth]{figures/15MargaretStirringSago_20250222_0006.jpg}
    \caption{Margaret Osi stirring sago jelly}
    \label{fig:margaretsago}
\end{figure}

Margaret has four children. Overall, she is a very active, clever, and prudent woman. Her youngest daughter Grace Osi has become a teacher and has a Bachelor's degree in education. When the mission came to Ossima with its white staff, Margaret quickly realised that a formal ``Western'' education would be essential in the future. She had a very high opinion of learning and teaching. At times she felt sad saying that she was no more than a ``bus kanaka.''\footnote{A derogatory term in Tok Pisin for an uncivilised person who follows a traditional rather than a modern lifestyle.} On the other hand, she was well aware that she had full command of Kilmeri - a privilege on the threshold of the 21st century, as subsequent generations had lost or abandoned the language at a rapid pace. She was dedicated to the Kilmeri documentation project from the very beginning and after a year had developed an astonishing degree of structural reflection on her language. Without her input, the field work would only have been half as successful as it is now. It is therefore most appropriate that more than half of the texts are authored by her; most of the texts in Chapters \ref{chap07} and \ref{chap08} are her own experiences.

\section{Susan Sumoi Bisam}\label{subsec:susansumoi}

Susan is a bit younger than her \textit{anti} (Tok Pisin for `aunt') Margaret, and she is the oldest of five children. Her native name and first name is \textit{Sumoi}. \textit{Bisam} is the name of her father. Her only brother, Joe, seems to be the leader in Ossima Asples, like his cousin Jeffrey in Osi Camp. Susan grew up without any formal education and is illiterate. Susan is the mother of ten children, including two sets of twins. All the children survived infancy. One of her younger daughters successfully completed the 10th grade at Lumi High School.

\begin{figure}
    \centering
    \includegraphics[trim=0 10 0 10,clip,width=0.6\linewidth]{figures/23Susan BisamWashing Sago_20250403_0007.jpg}
    \caption{Susan Bisam washing sago}
    \label{fig:susansago}
\end{figure}

Her husband Arnold was a crocodile hunter and settled near Angoram on the lower Sepik. At some point, the Puwani River became a renowned place for crocodile hunting, and Arnold moved there. Later, his wife Susan followed him to Amanab, Imonda, Utai, and Wasengla. There were also crocodile populations in the rivers of the shallow, swampy lowland areas south of the Bewani Mountains. Utai, for example, is quite similar to Ossima, with two rivers such as the Puwani and the Pual. In Susan's autobiographic story in \textref{sc0603}, which she delivered in both Tok Pisin and Kilmeri, she speaks of herself in the third person.

When I met Susan, she was still a strong woman and very actively involved in sago production. As she had little to no money, she and her younger children had to subsist on sago. At times, when it was too dry to wash the sago pith, they experienced periods of hunger. She then complained about the harshness of life and the little support she received from her male relatives. Susan seemed to have been immersed in the oral tradition, and she really enjoyed sharing the old stories with me. Unfortunately, her active command of Kilmeri was diminished by her long stay in the Sepik and she preferred to recite stories in Tok Pisin. It was only later that Margaret found a productive way to co-narrate stories in Kilmeri with her. Therefore, we decided to include three of her Tok Pisin stories in \textref{sc02tp1}, \ref{sc03tp2}, and \ref{sc03tp3}.

\section{Andrew Wapi}\label{subsec:andrewwapi}

Andrew Wapi from Ossima Asples was about the same age as Margaret Osi and like her, he had never received any formal education, so he was not literate. Sadly, he passed away in 2009.

After some travelling he settled in Asples. As a young man, he seized the opportunity to work on coconut plantations, which brought him to Rabaul. There he learnt Tok Pisin, the language that became increasingly natural to him over the years. When he returned to the area, the mission was founded, and he was able to find paid work there. For several years he worked as a cook at the mission station in Utai in the south-eastern foothills of the Bewani Mountains, where two of his children were born. He then returned to Ossima. He was the father of six children. One of his sons died in 2000, and on this occasion a mourning singsing took place. Despite his acquaintance with Western life Andrew highly valued the traditional way of life. He owned an impressive bow and was a good hunter. He was able to spot a bird of paradise hidden in a tree top and invisible to curious visitors!

Andrew has contributed five traditional stories (cf. Texts \ref{sc0305}, \ref{sc0306}, \ref{sc0307}, \ref{sc0406}, and \ref{sc0407}). Unfortunately, he seemed a bit intimidated by the recording equipment and might have produced a higher rate of disfluencies as a result. Nevertheless, he was always eager to present his stories about bush spirits and hunting events.

\section{Brigitte Esau}\label{brigitteesau}

Brigitte Esau is a generation younger than the above-mentioned narrators. She went to school in Ossima and continued her education in Vanimo. After completing high school, she found work in Port Moresby for two years. She was married to a man from Sepik, with whom she settled in Omoi. Like Susan Bisam, she is an example of a marriage with a man from another language which practically leads to the use of Tok Pisin as the family language. The respective language communities of husband and wife are not adjacent, nor was there any contact between them at earlier times.

\begin{figure}
    \centering
    \includegraphics[width=0.9\linewidth]{figures/24BrigitteEsauOnume_20250403_0006.jpg}
    \caption{Brigitte Esau (second from left) with her mother and children in Onume}
    \label{fig:brigitteesau}
\end{figure}

Brigitte was keen to take part in the language work and contributed two stories that she had heard from her mother Sara. The brevity of her tales reflects the fact that her active command of Kilmeri was not comparable to that of her mother. Despite this, the two stories enrich the text collection. There are now two versions of the cultural exploitation of salt in \chapref{chap02} and a further story about an encounter between humans and bush spirits in \chapref{chap03}. Also note the sketch by Brigitte Esau showing the route from the Puwani-Pual basin to the coast (\figref{fig:sketchomoi}).

Sadly, Brigitte Esau passed away in 2014. She could still remember the time when the old men's house stood on a hill above the other houses in the Omoi uphill settlement where she had grown up. When she realised it was gone after returning from the city, she recalled feeling a deep sadness. With her deep appreciation of her origin and the old traditions she would have been able to revive the Kilmeri language in a team of like-minded people, yet her early death prevented this.

\section{Usikul}\label{subsec:usikul}

Usikul is a member of one of the Omoi clans. He is always called by his two names \textit{Usi} and \textit{Kul}, which are merged to \textit{Usikul}. He has no Western name. He lived on one of the hills that rise north of the riverside settlement of Omoi. He presents himself as a self-confident person navigating ``between the worlds,'' namely between the traditional and the modern world. He shows this mindset in a particular way that differs from that of Margaret Osi, who in her own way builds a bridge between tradition and ``modernity.'' Usikul is historically and politically aware and consciously tries to combine these diametrically opposed heritages.

\begin{figure}
    \centering
    \includegraphics[trim=0 10 0 0,clip,width=0.6\linewidth]{figures/01NarratorUsikul_20250222_0020.jpg}
    \caption{Narrator Usikul}
    \label{fig:usikul}
\end{figure}

Usikul's settlement is a small place called Ounup, consisting of two houses only, his old house and his new house. There is a waterhole nearby, but the small pool was overgrown with aquatic plants when I saw it in September 2000. For several years there was a road near his property, a logging track, which was only temporarily maintained and disappeared when the valuable timber was felled and transported away. The income from the logging brought him some money, some of which he invested in long trousers, a fine shirt, and leather shoes: his modern appearance. He wore it to celebrate the day when he could pass on his intricate story about the fate of his clan and the old tradition. On this occasion, two feathers decorated his head ( \figref{fig:usikul}).

\section{Sei Walup}\label{saiwalup}

Sei Walup was an old man who lived in Awol and, like Usikul, was only known by his two native names. The village of Awol is 2 to 3 hours' walk from Ossima. The children from there who attend the Ossima Mission school usually live with relatives in Ossima. No one from the village of Awol was involved in the fieldwork as a regular consultant, but when I asked for people who knew old stories, he was mentioned as a knowledgeable person. Unfortunately, there was only one session with him, in which he told a clan story (\textref{sc0206}). He proved to be a passionate storyteller like Susan Bisam. In his lengthy introduction to the story it becomes evident that this story should belong to the heritage of the whole Kilmeri speaking community. He seemed to be aware of all the conflicts between the clans and the sometimes deadly consequences. One lesson of the story he told me could be that one should never forget local history.

\begin{figure}[H]
    \centering
    \includegraphics[trim=0 10 0 10,clip,width=.6\linewidth]{figures/04SeiWalupHouse_20250222_0011.jpg}
    \caption{Sei Walup (on the right) in front of his house}
    \label{fig:seiwalup}
\end{figure}

Sei Walup seemed to live as much as possible under traditional conditions. His house was rather modest compared to the houses in Ossima (\figref{fig:seiwalup}). In Awol, modern material culture was less visible than in the villages along the Pual and Puwani rivers. There were no water tanks, nor corrugated iron roofs  (\figref{fig:awolvillage}). 

\section{John Sol}\label{subsec:johnsol}

John Sol, a middle-aged man from the Isi camp, was not a storyteller himself. Nevertheless, the clan-related text ``Ome and Lapi'' (\textref{sc0207}) is based on his knowledge. Margaret Osi and I had visited him with the intention to buy a cow, as he was one of those who owned a number of cattle. Later, when Margaret was finalising the deal, John brought up this story and provided a short version of it. It was this version that Margaret then reproduced.

\section{Anita Osi and Lillian Bisam}\label{subsec:anitalillian}

In the first months of my fieldwork my older consultants were assisted by young people who had attented the school at Ossima. They were Anita Osi, the oldest daughter of Jeffrey Osi, and Lillian Bisam, the oldest daughter of Joe Bisam, Jeffrey Osi's cousin from Ossima Asples. In 2000, both girls were 16 years old. They took turns attending the language sessions, were able to read what I had written down, and suggested spelling when they thought I had made mistakes in my transcription decisions. This often helped to clarify things. Both had a very good passive knowledge of Kilmeri. Especially Lillian's explanations in particular showed that the person agreement with the recipient argument is a central feature of the grammar, which is retained in the otherwise simplified Kilmeri (See \sectref{sec:gramrel}). I am also grateful for her explanation of the digestive system of a bush spirit, namely it cannot digest what it eats. Instead, everything drops out of its anus undigested. This is therefore a clear indication of whether an unknown male person is a human or a bush spirit.

\section{Personal statement}\label{subsec:personality}

In March/April 1999, when I travelled to Papua New Guinea for the first time to explore where I should conduct my field research, I also visited the University of Papua New Guinea in Port Moresby. It seemed appropriate to meet Professor Otto Nekitel from the Department of Language and Literature, who was teaching linguistics there. Even though I came unannounced and simply knocked on the door of his office, he welcomed me warmly. I immediately sensed his interest in my plans and his intellectual support to document a Papuan language. He encouraged me to go to Ossima, a place apparently known because of its agricultural farm. Otto Nekitel urged me not only to write a grammar of Kilmeri, but collect  the traditional stories of the Kilmeri people in order to preserve these for future generations. Sadly, Otto Nekitel passed away soon after my second visit to him in August 2001.

There is a huge difference between an indigenous scholar and a foreign scholar coming from the other end of the world! My brief field stays never allowed me to gain a comprehensive insight into the society of the people who hosted me. I always remained an outsider. I was able to learn their language to a certain extent, but the links between the language and their traditional way of life remained inaccessable to me. Understanding the structure of a language is one thing, but the stories that have been passed down for generations are quite another. 

At first, the plots of the stories seemed strange to me. My consultants patiently attempted to explain the plot lines to me. They felt at ease with all the protagonists and characters. They felt comfortable with their past, even though they were living in a constantly modernising environment. Their voices are the voices of yesterday, today and – hopefully – tomorrow.\footnote{The title of Otto Nekitel's book ``Voices of yesterday, today and tomorrow'' (\citeyear{Nekitel:1998gy}).}

Behind a ``voice'' there is always the story teller with his or her very own biography and life experiences. I wanted the narrators to become as vivid figures as the characters of their stories. Clearly, this is a subjective perspective. All too often, fieldworkers limit the information provided about the story tellers to a few data points that might be of statistical value. The narrators' personalities, however, remain in the dark. Personally, I find this regrettable. That is why I have decided to introduce them with some biographical details, so that the reader can get to know the real persons behind the stories. I think it is even a matter of respect to their efforts in helping me despite my very slow progress in learning their language and my ``hevi maus'' as my landlord Jeffrey Osi once put it. When reflecting on the years of fieldwork, my encounters with so many different people, and the conversations that I have had with them, I can find a one recurring theme: their desire and hope to participate in the ``greater world.'' 

Unfortunately, I could not obtain the narrators' explicit consent to portray them in this book. At the time of fieldwork, my priority was to write a grammatical description, and all my consultants are simply listed in the grammar. Now, twenty-five years later, the narrators featured in this book have passed away because of age or sickness, and it is impossible to get their explicit consent. I can only hope that my behaviour towards them was such that they would have enjoyed becoming part of this book. 

\begin{figure}
    \centering
    \includegraphics[width=0.9\linewidth]{figures/MargaretCLaudiaSusanAndrew_0006.jpg}
    \caption{Margaret Osi, Claudia Gerstner-Link, Susan Bisam, Andrew Wapi}
    \label{fig:transcriptionteam}
\end{figure}

\chapter{Provenance}
\section{Transcription and editing}\label{sec:editing}

\subsection{General remarks}\label{subsec:generalremarks}

Twenty-five years passed between the beginning of my fieldwork in November 1999 and the publication of this Kilmeri text collection. The recording equipment was technically up to the standard of the year 2000 and consisted of a table microphone and a tape recorder with Maxell or BASF tapes. The microphone was placed on the table about half a metre away from the speaker. The recording is not uniform because the speakers changed their voice intensity during the narration.

The recordings took place in my house at Osi Camp, which was built from bush material. Osi Camp is a hamlet consisting of about six houses belonging to Lis Osi's family. The recordings could not be isolated from the background noises of village life. In some cases, the background noise was so loud that it impeded listening to the tapes for subsequent transcription. The repeated rooster cries are almost comical. In any case, there was no way to avoid the sounds of everyday village life.

About half of the texts were recorded, the other half of the (mostly short) texts were written down by me while the narrator spoke in a relatively slow and well-articulated manner.

A note on the titles of the texts: Only in a few cases the titles were created by the narrators themselves, while most of them were created by me. Instead of a short title, narrators sometimes chose an introductory sentence in Tok Pisin to frame the story. This happened, for instance, with the story of the girl Wapues in \textref{sc0205}. If a narrator chose her own title, then this is stated in the short introduction to the respective text.

The texts are presented in two versions. First, there is a parallel text version with the original Kilmeri and the English translation in two columns. In this version, Kilmeri is written with punctuation marks and paragraphs that roughly correspond to the English translations. Secondly, there is the interlinearised version: a sequenced, morphologically analysed and glossed text version with a translation into English. It should be noted that the parallel text version uses Kilmeri's morpho-phonemic rules (cf. \cite[75-84]{Gerstner-Link:2018un}).

\subsection{Stories of early fieldwork}\label{subsec:earlystories}

The narrations of these stories were scheduled for a specific session or day and took place in my house. Once everyone had arrived, the recording equipment was set up and explained to the narrator and to Margaret Osi, who was present at every recording session on account of her language proficiency. Usually, the transcription started the day after the recording. This was a time-consuming undertaking and required a great deal of patience on both sides. It happened time and again that a few words or sentences in a recording could not be recognised. After listening several times to a passage, Margaret would say: \textit{Mi no save, mi no harim, em no klia long mi.} (`I don't know, I can't understand, it's not clear to me.') We have omitted such passages and proceeded with the transcription. Omissions of this kind are not signposted in the edited texts. Obviously, Margaret Osi was concerned to present a grammatically sound, fairly coherent narrative. She had no interest in interruptions by ``stupid'' repetitions whenever the narrator had lost the plot.

I insisted on including the phrases in Tok Pisin, as code-switching was a typical text feature of the narrator Andrew Wapi. On one occasion, Margaret herself used Tok Pisin when she began to tell a story. She only realised this when I asked her about it, and she then went on in Kilmeri (\textref{sc0205}). 

\subsection{Procedural texts}\label{subsec:proceduraltexts}

The production of procedural texts was not planned in advance, but happened spontaneously. On such occasions, Margaret Osi used to say something like, ``Well, you should learn about penis gourds that my husband used to wear,'' or ``Yesterday I made oil, so listen to how it's done!'' Then Margaret spoke loud and clear, and the text was written down as she spoke. Only rarely did she edit the text herself afterwards. Some of these changes consisted of choosing a serial verb instead of a simple verb or adding a sentence in order to clarify something. 

These texts are not complete instructions on how to do something, but short descriptions to provide the listener with some basic knowledge. The texts were not recorded on tape.

\subsection{Episodes of daily life}\label{subsec:episodesdaily}

These are Margaret Osi's spontaneous thoughts. Setting up a recording device would have been impractical, it would have interrupted the spontaneity of her ideas. Some of these (very) short texts even have an intimate flavour, for example when Margaret Osi recounts a dream she had the previous night (\textref{sc0804}). Such situations are sensitive and must not be disturbed by technical equipment. 

The language of these short events was always free-flowing. I took notes and only afterwards asked questions about some unclear words or constructions. The flow of the speaker's thoughts was therefore not interrupted. These texts are not edited, but reflect what was spoken.

\subsection{Teamwork stories}\label{subsec:teamworkstories}

There are three traditional stories that were narrated in team by Susan Bisam and Margaret Osi: \textit{Urual bekulu} (\textref{sc0302}), \textit{Walpop bo} (\textref{sc0303}), and \textit{Wîs yako} (\textref{sc0309}). The first step in this process was to remember the story. The two women took turns telling the story, using both languages, Kilmeri and Tok Pisin. They reassured each other about the correct unfolding of the events. This step was recorded but could not be transcribed because there was too much ``meta-talk,'' namely discussions about which character did exactly what at what point in time, and so on. Once this was clarified, one of the women continued the story. In order to get a coherent and fluent version of the story, I asked Margaret to retell the whole story the next day. This version, as the second step of the storytelling, was not recorded but written down while Margaret spoke. 

The text \textit{The spellbound lake} (\textref{sc0206}) was narrated by a single person, Sei Walup. But it was narrated in Tok Pisin with only a few Kilmeri insertions. Despite being recorded, it was not suitable as a Kilmeri text in this form. However, Margaret Osi, who was not present at the recording in the village of Awol, also seemed to know this story. After listening to the Tok Pisin version, she retold it in Kilmeri. This version was not recorded on tape, but written down.

\subsection{Stories from later fieldwork}\label{subsec:laterstories}

From 2004 onwards, many stories were no longer recorded but written down as they were spoken. This type of storytelling was much favoured by Margaret Osi. It seemed to enhance her concentration. The face-to-face interaction during the storytelling sessions -- sometimes planned, sometimes not -- came close to the traditional setting, at least to some extent. The eye contact showed Margaret that I was paying attention to her and the story. In comparison, the microphone and recorder seemed to replace the human listener, and the storyteller in question -- in fact all the storytellers -- focussed on the ``machine,'' as they often called it. Without the recorder, the whole situation was more relaxed and simply more natural. There was no fear of ``making a mistake.'' So we mostly avoided recording.

In two cases, the narratives describing old village life were recorded after a simpler oral presentation. In these cases, it is interesting to compare the choice of words and succinctness of the resulting texts. The simpler oral presentation was much more concise and avoided side issues of the story as well as special constructions such as the tail-head linkage constructions. In contrast, the recorded texts were longer and touched on subplots that illustrated the general story scene but were not strictly relevant to the topic of the text. The use of tail-head linkage constructions now seemed to help the speaker -- Margaret Osi -- remember her personal experiences.

After a story had been written down, I read it back to Margaret, and she decided on any editorial interventions. Such changes were minimal: she never deleted sentences, but occasionally added a more sophisticated word.

\section{English translation principles}\label{sec:translation}

The Kilmeri text collection provides two types of English translation for the stories. One is the English running text translation and the other is the translation in the glossed text version. These translations differ in a number of aspects.

The English running text translation can be found in the parallel text version of each text. It retells the story in English and familiarises the reader with the plot of a story and its development. It preserves all direct speech as we find much dialogue in several texts. However, it does not preserve the style of the Kilmeri narration and is not intended to (fully) account for the grammar of Kilmeri. This translation will be helpful to anthropologists and historians, who might not be so much concerned with the intricacies of Kilmeri grammar. But it will also give the linguist a first orientation about the content of the story.

The English translation in the interlinear version, i.e. in the glossed text, stays closer to the original Kilmeri version. It tries to match the syntactic constructions of Kilmeri. For example, backgrounding constructions and tail-head linkage constructions are usually retained in English. The tenses and tense variations of the Kilmeri text are also generally retained. Subjects in the Kilmeri text appear as subjects in the English translation. However, topic-focus constructions of subjects (and sometimes objects) are rendered as plain subjects (or objects). 
Deictic expressions are retained in the English sentence structure whenever possible. Occasionally, deictic forms in Kilmeri texts are closely spaced. This is done to emphasise the process of situational understanding by a protagonist. Narrative pauses or short interruptions of speech are preserved in the English translation by repeating a phrase as in Kilmeri. Often a referential phrase stands alone as an utterance followed by a full stop. Finally, Kilmeri's style of narration is rich in (visual) figurative speech, which the translation attempts to preserve.

Generally, both English translations contain more explicit referential phrases, as the omission of these is part of the concise Kilmeri style. The punctuation in the translations, especially the use of a full stop, reflect the Kilmeri clauses in most cases. Commas are used much less frequently to separate short or single-verb clauses. Listed entities referred to by nouns are not separated by commas in the Kilmeri running text.

I will illustrate my translation principles in the glossed texts with a few examples in the following sections.

\subsection{The use of the continuous past}

In \textref{sc0203} \textit{Sakou}, the phrase \textit{umul (se)neki} `to think about sth, to consider, to reflect upon' appears in the Sequences (\ref{ex:nrexsakouu}), (\ref{ex:nrexakouum}), (\ref{ex:nrexseneka}), and (\ref{ex:nrexkipkoy}). But the continuous past \textit{umul nekip} is only used in Sequence (\ref{ex:nrexkipkoy}), otherwise the punctual past is used. The difference in tense points to the fact that the hero Sakou in (\ref{ex:nrexkipkoy}) realises that he has to fight a bush spirit. His thinking about what to do needs some time now: What will be the best way to trick the evil spirit so that he cannot threaten Sakou's life? Therefore this difference in tense is retained in the translation.

Furthermore, there are verbs in Kilmeri that denote actions with inherent duration like \textit{nise} `laying in wait' or \textit{le(wo)-} `wait (for sb)'. Such actions are preferably coded by the continuous past instead of the punctual past (e.g. \textref{sc0206}, Sequence (\ref{ex:nrexolasau})). The translation retains the continuous past. The verb \textit{nake} `sit, live, stay' is a stative verb with inherent duration. In Kilmeri, it occurs always in the continuous past. The translation is guided by the narrative context.

The verb \textit{dob pi} `to look at' and its variants appear in the punctual past or in the continuous past, depending on whether it refers to a brief glance at something or to a searching look, for example in \textref{sc0206}, Sequence (\ref{ex:nrexdupyen}) and in \textref{sc0207}, Sequence (\ref{ex:nrexkuipip}). The English translation traces this difference.

\subsection{Tense shift}

In the Kilmeri texts, it can be observed quite frequently that the past tense shifts to present tense. It is not entirely clear if this switch follows a strict discourse rule. However, there are certainly cases in which the change to present tense serves a narrative purpose. 

For example, in \textref{sc0205}, Sequence (\ref{ex:nrexohonou}) continues in present tense, although we have past tense in the preceding Sequence (\ref{ex:nrexikapon}). At this point in the story of \textit{Wapues}, the female protagonist realises that the alleged bush spirit is in fact a human being, and she is positively surprised. This narrative turn is expressed by a tense shift, which is continued to Sequences (\ref{ex:nrexyonuid}) and (\ref{ex:nrexdeilan}). The turn of surprise is also present in \textref{sc0204} in Sequences (\ref{ex:nrexehkonu}), (\ref{ex:nrexppuesu}), (\ref{ex:nrexriseku}), and (\ref{ex:nrexmekkwe}). In the story of \textit{Sakou}, the bush spirit, who seemed dead, suddenly gets up and climbs up the mountain/tree. The verb \textit{ppue} `go up, climb' appears in present tense in Sequence (\ref{ex:nrexmekkwe}) even though preceding verbs appeared in punctual past. Also in \textref{sc0206}, Sequence (\ref{ex:nrexkikike}), the escape of the protagonist Amou is expressed in present tense in order to fully capture the dramatic turn of the story.

There are also tense shifts for grammatical reasons. In \textref{sc0206}, the Sequences (\ref{ex:nrexbusukn}) and (\ref{ex:nrexlowaer}) appear in present tense which is triggered by the proximal deictic marking: \textit{waeripi pule ere mini} `waeripi-fish come, here they come hither' with the proximal deictic \textit{ere} and the inherently deictic verb \textit{mini} `come hither/here'. This particular sensitivity to the tense of the original language should not be obscured by the translation.

\subsection{Backgrounding}

Backgrounding by subordinating verb forms is a common construction in Kilmeri. The most frequent form is \textit{k}-VERB-\textit{p}-\textit{no} with sequential subordinating meaning. The most suitable English translation of this verb form is `having + past participle.' This translation works best when the subordinate clause and the main clause have the same subject/agent. But even in this case, two main clauses in the simple past tense may be preferable. 

However, I choose another kind of translation in the case of different subjects/actors, as illustrated in \textref{sc0205}, Sequence (\ref{ex:nrexikapon}). Here, two consecutive events are backgrounded by the verbal forms \textit{k-poname-p-no} `having given' and \textit{k-ni-p-no} `having eaten'. The two events have different agents, the girl Wapues and the bush spirit respectively. This interesting discourse structure is translated into English with two separate clauses `when she had given (the food) to him and he had eaten it.' See also \citet[473-476]{Gerstner-Link:2018un}, where I describe same subject and different subject marking in sequential subordinating verb forms. 

\subsection{Topic constructions}

Consider the following topic-focus constructions in \textref{sc0304} in Sequence (\ref{ex:nrexdekepp}): \textit{weri ki muelno de ke ppulaena hukpo}, which literally translates as `the younger sister, she said to her: ``You, you caught many fish.''' In the English translation, I omitted the topic pronouns, in order to produce a better English. The same can be found in \textref{sc0306}, Sequences (\ref{ex:nrexplayor}), (\ref{ex:nrexpulwis}), and (\ref{ex:nrexmikebu}); and in \textref{sc0307}, Sequences (\ref{ex:nrexkowwwe}) and (\ref{ex:nrexbakabk}). Note that the Kilmeri topic construction often occurs in the context of direct speech.

\subsection{Summary}

The English translation in the glossed text version is intended to make the reader familiar with the characteristic narrative style of Kilmeri. If a free translation were to level out stylistic peculiarities, it would defeat this purpose. I believe that a text collection should go beyond individual sentences and provide insights into the discourse mechanisms of a language and the coherence of the text. However, this is usually not the aim of grammars, which are descriptions of grammatical structures in a narrower sense. One might argue that one only needs to read the Kilmeri wording in detail in order to understand the discourse mechanisms of the language. However, this is not as easy as it sounds; it is rather difficult without a good knowledge of the grammar. Therefore, the English translation in the glossed texts bridges the gap between Kilmeri discourse style and English text style.

\section{Parallel texts}\label{sc0108}

In addition to the glossed text version, this collection includes a running text version in Kilmeri and English. Running texts have a greater potential to invite the reader into the indigenous literature than the rather technical apparatus of interlinearisation. One option was to place the running text versions of the individual stories in sequence. This type of visualisation would come closest to our own -- the fieldworker's and reader's -- tradition. However, there is a drawback in that the texts would have been presented without their inherent correlation.

The parallel arrangement in columns enables a more direct comparison. The most obvious parameter for comparison is the length of the texts. The comparison of the Kilmeri version and the English version shows that the Kilmeri version is shorter and more concise, paragraph by paragraph. This points to a linguistic issue in that languages differ in their way of linking and embedding clauses. Languages such as English and German use a wide range of paratactic and hypotactic conjunctions that are absent in other languages (e.g. Latin and Ancient Greek). Kilmeri also lacks this type of connectivity. Attitudinal adverbials, which often link sentences, do not exist in Kilmeri at all.

Kilmeri can also easily dispense with pronouns. This can lead to sequences of verbs as one-word sentences. Referential explicitness in relation to local expressions is less important in Kilmeri than in English, it seems. But there is one caveat in that context: Kilmeri stories in the text collection were originally aimed at an audience that is capable of inferring places from the immediate context of the action or from common ground.

In addition, Kilmeri often does not embed the direct speech in a matrix clause such as `X says', but instead only reproduces the direct speech. This may even happen at the beginning of a turn-taking sequence of direct speech between two protagonists in a story. On the other hand, the verb \textit{mueli-} `say to someone, tell someone, talk to someone' belongs to the small class of verbs with obligatory recipient agreement, so that the verb agrees with the addressee of the direct speech. It may also be of interest here that Kilmeri has no real counterpart of the verb \textit{mueli} with the meaning of `to answer, to reply'. The serial verb \textit{dori\_mueli} `to turn back + say' can be used to indicate a responding speech, but it is more often used for the speaker's own repetition of what has been said. 

All these characteristics of Kilmeri outlined above result in a more compact text, making paragraphs and entire texts visibly shorter than the English translations. This arrangement as parallel texts reveals the different grammatical and discursive strategies between Kilmeri and English at a glance. 

\chapter{Grammatical overview}\label{sc0109}

\begin{figure}
    \centering
    \includegraphics[width=0.7\linewidth]{figures/18-323_Kilmer lang_PNG_enlarged-03.jpg}
    \caption{Location of Kilmeri in the Border family and surrounding language families.}
    \label{fig:languagemap}
\end{figure}

\section{Introductory remarks}

Kilmeri is a Papuan language of the Border family, which is located in northern New Guinea on both sides of the international border between Papua New Guinea and Indonesia (\figref{fig:languagemap}). The following grammatical overview summarises the main features of Kilmeri grammar. It contains many illustrating and explanatory tables. The focus is on the coding of grammatical relations. Kilmeri combines several independent strategies that together build up an intricate system of marking these relations. One will find all the necessary examples covering the morphological coding devices of subjects and objects as well as of the semantic roles of Agent, Patient, and Recipient. For further and detailed information on all grammatical domains the reader is referred to my book ``A Grammar of Kilmeri'' (\cite{Gerstner-Link:2018un}) that presents the relevant discussions based on rich illustrations.

\section{Phonology}

Tables \ref{tab:02tbvosyki} and \ref{tab:02tbconkil} present the inventories of vowels and consonants of Kilmeri as well as its biphonemic vowel sequences (cf. \tabref{tab:02tbbivose}). The most common syllable structure in Kilmeri is CV. For verbs, this is the only syllable structure, with a few exceptions of the syllable type V. Nouns can also have closed syllables of the type CVC. Some fully inflected verbs also show closed syllables. Serial verbs have the syllable structure CVCV\_CVCV.

\begin{table}[p]
\caption{Vowel system of Kilmeri}
\label{tab:02tbvosyki}
\begin{tabularx}{\textwidth}{XCCc}
    \lsptoprule
    & \textsc{(near-)front} & \textsc{central} & \textsc{(near-)back}\\
    \midrule
    \textsc{high}&i&&u\\
    \textsc{near-high}&ɪ&&ʊ\\
    \textsc{mid}&ɛ&&ɔ\\
    \textsc{near-low}&æ&&\\
    \textsc{low}&&ɐ&\\
    \lspbottomrule
\end{tabularx}
\end{table}

\begin{table}[p]
\caption{Consonant system of Kilmeri}
\label{tab:02tbconkil}
    \begin{tabularx}{\textwidth}{l@{}ccccCc@{~}}
        \lsptoprule
        &\textsc{bilabial}&\textsc{labio-}&\textsc{alveolar}&\textsc{palatal}&\textsc{velar}&\textsc{glottal}\\
        &&\textsc{dental}&&&&\\
        \midrule
        \textsc{plosives}&&&&&&\\
        {\footnotesize{voiced}}&b&&d&&(g)&\\
        {\footnotesize{prenas. with}}&\multirow{2}*{ʙ}&&&&&\\
        {\footnotesize{trilled release}}&&&&&&\\
        {\footnotesize{voiceless}}&p&&&&k&ʔ\\
        {\footnotesize{labialised}}&p\textsuperscript{w}&&&&&\\
        \textsc{nasals}&m&&n&&&\\
        \textsc{rhotic trills}&&&r&&&\\
        \textsc{fricatives}&(ɸ/β)&(f)&s&&&\\
        \textsc{laterals}&&&l&&&\\
        \textsc{approximants}&&ʋ&&j&&\\
        \lspbottomrule
    \end{tabularx}
\end{table}

Biphonemic vowel sequences are quite frequent (cf. \tabref{tab:02tbbivose}). In some words as well as in spoken Kilmeri complex syllable onsets occur with the following structures C(b,p,k,s) + /r/ or C(k,s) + /l/ or C(s) + N.

\begin{table}[p]
\caption{Biphonemic vowel sequences}
\label{tab:02tbbivose}
\begin{tabularx}{\textwidth}{lCCCCCCCc}
    \lsptoprule
    &i&ɪ&u&ʊ&ɛ&ɔ&æ&ɐ\\
    \midrule
    i&&&[iu]&&[iɛ]&&&[iɐ]\\
    ɪ&&&&&&&&\\
    u&[ui]&&&&[uɛ]&[uɔ]&&[ua]\\
    ʊ&&&&&&&&\\
    ɛ&[ɛi]&&[ɛu]&&&[ɛɔ]&&\\
    ɔ&[ɔi]&&[ɔu]&&&&&\\
    æ&&&[æu]&&&&&\\
    ɐ&[ɐi]&&[au]&&&&&\\
    \lspbottomrule
\end{tabularx}
\end{table}

\tabref{tab:ortcon} shows the orthography used in this text collection.

\begin{table}
\caption{Orthographic conventions}\label{tab:ortcon}
    \begin{tabularx}{.65\textwidth}{llll}
    \lsptoprule
        \multicolumn{2}{c}{\textsc{consonants}}&\multicolumn{2}{c}{\textsc{vowels}} \\
        \textsc{grapheme} & \textsc{phoneme}  & \textsc{grapheme}  & \textsc{phoneme} \\
        \midrule
        <b>  &   /b/    & <a>   & /ɐ/ \\
        <d>  &   /d/    & <ae>  & /æ/ \\
        <k>  &   /k/    & <e>   & /ɛ/ \\
        <l>  &   /l/    & <i>   & /i/ \\
        <m>  &   /m/    & <î> & /ɪ/ \\
        <n>  &   /n/    & <o>   & /ɔ/ \\
        <p>  &   /p/    & <u>   & /u/ \\
        <pp> &  /\textsuperscript{m}ʙ/   & <û>  & /ʊ/ \\
        <r>  &   /r/    &     &  \\
        <s>  &   /s/    &     &  \\
        <w>  &   /ʋ/    &     &  \\
        <y>  &   /j/    &     &  \\
    \lspbottomrule
    \end{tabularx}
\end{table}

Note the following morphophonemic changes which occur in certain TAM inflected verb forms: (i) vowel lowering before /p/ and /m/; (ii) regressive vowel assimilation after consonantal prefixes; (iii) apocope, syncope, coalescence of vowels. Some of these morphophonemic changes become apparent when comparing the phonological first line of the glossed texts with the morphological second line.

\section{Word order and focus position}

The basic word order in Kilmeri is SV or AOV. Temporal adjuncts usually precede the subject, while locative adjuncts follow the verb. The focus position of a clause is immediately before the verb. This preverbal position is obligatory for verbal negation and wh-words. If a locative phrase is the focus of an utterance, the locative adjunct appears right before the verb in the clausal focus position. The same happens for subject focus. Instrumental adjuncts usually occur in some position before the verb. 

Although Kilmeri uses personal pronouns frequently, there are often verb-only clauses when the actor of the action indexed by the verb is contextually or situationally known. Many clauses consist of only two constituents, namely, a type of nominal phrase (subject phrase, object phrase, locative phrase, instrumental phrase, temporal phrase) and the verb. Clauses containing three or more constituents are rare. 

\section{Nominal morphology and noun phrase structure}

\tabref{tab:nocasu} shows the nominal case suffixes; it also contains three clitics that occur mainly on nominals. Kilmeri has only peripheral or semantic cases. Grammatical relations are encoded in the verb morphology (cf. \sectref{sec:gramrel}).

\begin{table}
    \caption{Nominal case suffixes and clitics}
    \label{tab:nocasu}
    \begin{tabularx}{.62\textwidth}{lll}
    \lsptoprule
        \textsc{suffix}&\textsc{gloss}&\textsc{function}\\
    \midrule
        \textit{-pi}&\textsc{poss}&possessive case \\
        \textit{-no}  & \textsc{ins} & instrumental-comitative case \\
        \textit{-yo}  & \textsc{loc} & locative-allative case \\
        \textit{-ka}  & \textsc{path} & path indicating case \\
        \textit{-so}  & \textsc{sim} & similative case \\
        \textit{-na}  & \textsc{aff} &  affinitative case \\
        \textit{-e}  & \textsc{voc} & vocative case \\
        \textit{=ro}  & \textsc{emph} & emphasis marker \\
        \textit{=pe}  & \textsc{q} & question marker \\
    \lspbottomrule
    \end{tabularx}
\end{table}

All types of modifiers follow their respective head noun. Examples (\ref{ex:intro-1}-\ref{ex:intro-8}) illustrate some possible structures. Case suffixes appear at the end of the noun phrase, and have scope over the whole phrase. 

\begin{multicols}{2}
\ea \label{ex:intro-1}
    \textup{[\textsc{n adj poss}]}\\
    \gll \textit{yip} \textit{puene} \textit{kopi}\\
    house new 1\textsc{sg}.\textsc{poss}\\
    \glt `my new house'
\z

\ea \label{ex:intro-2}
    \textup{[\textsc{n adj qt}]}\\
    \gll \textit{yip} \textit{ikoi} \textit{an\_baka}\\
    house big five\\
    \glt `five big house'
\z

\ea \label{ex:intro-3}
    \textup{[\textsc{n n qt}]}\\
    \gll \textit{bese} \textit{supue} \textit{dupua}\\
    tulip-greens bunch two\\
    \glt `two bunches of \textit{tulip-}greens'
\z

\ea \label{ex:intro-4}
    \textup{[\textsc{[n n]\textsubscript{poss} qt det}]}\\
    \gll \textit{yûr} \textit{su} \textit{dupua} \textit{ba}\\
    chicken egg two other\\
    \glt `[the] two other chicken eggs'
\z

\ea \label{ex:intro-5}
    \textup{[\textsc{n np}]}\\
    \gll \textit{ako} \textit{dari} \textit{werino}\\
    wife old.sister young.sister.\textsc{ins}\\
    \glt `wives, older and younger sister'
\z

\ea \label{ex:intro-6}
    \textup{[\textsc{n[n poss]}]}\\
    \gll \textit{buka} \textit{ruri} \textit{ikep}\\
    sister's.child child 3\textsc{sg.poss}\\
    \glt `the nephew, his own child'
\z

\ea \label{ex:intro-7}
    \textup{[[\textsc{n adj}] \textsc{n-poss}]}\\
    \gll \textit{yip} \textit{puene} \textit{Jeffrey-pi}\\
    house new Jeffrey-\textsc{poss}\\
    \glt `Jeffrey's new house'
\z

\ea \label{ex:intro-8}
    \textup{[[\textsc{n adj}] \textsc{n-poss-loc}]}\\
    \gll \textit{yip} \textit{puene} \textit{Jeffrey-pi-yo}\\
    house new Jeffrey-\textsc{poss}-\textsc{loc}\\
    \glt `at Jeffrey's new house'
\z
\end{multicols}   

Noun phrases can be connected by the postposed particle \textit{roise} `with, together (with)'. The particle connects simple noun phrases as well as complex noun phrases. For simple noun phrases, the \textit{roise}-phrase immediately follows the first element (\ref{ex:intro-9}). When connecting complex noun phrases, the \textit{roise}-phrase can be postponed after the verb (\ref{ex:intro-10}). 

\ea\label{ex:intro-9}
    bese paepu roise baroko\\
    \gll bese paepu roise ba-re-ko\\
    tulip.leaves mushrooms with \textsc{fac}-get.done-\textsc{fac}\\
    \glt `The mushrooms with \textit{tulip}-leaves are done.' \exsource{Text \ref{sc0403} Sequence \ref{ex:t3expaepsi}}
\z


\ea\label{ex:intro-10}
    bi puaku bou ulap, sû worno roise\\
    \gll bi puaku bou ule-p sû wor-no roise\\
    pig head back.limbs be.there.\textsc{pl} fire dog-\textsc{ins} together.with\\
    \glt `A pig's head and back limbs were there, together with embers and a dog.' \exsource{Text \ref{sc0205} Sequence \ref{ex:nrexumaeau}}
\z

\section{Pronouns}\label{sec:pronouns}

The personal pronouns of Kilmeri distinguish the categories of person, number, and clusivity. These distinctions lead to eleven different pronominal forms. It is worth mentioning that the original pronoun system of the Border language family comes with only four forms: first person, second person, third person, and inclusive. The expanded system of Kilmeri is a new development which can easily be seen in the morphological structure of the newly added forms (cf. Tables \ref{tab:pronouns}, \ref{tab:emphpronouns}, and \ref{tab:posspronouns}).

\begin{table}
    \caption{Pronouns}
    \label{tab:pronouns}
    \begin{tabularx}{.6\textwidth}{lXXX}
        \lsptoprule
        & \textsc{singular}& \textsc{dual}& \textsc{plural}\\
        \midrule
        1 \textsc{incl} & \textit{ko} & \textit{dedukoyo} & \textit{nuko}\\
        1 \textsc{excl} & & \textit{koyo} & \textit{uke}\\
        2 & \textit{de} & \textit{deyo} & \textit{ine}\\
        3 & \textit{ki} $\sim$ \textit{ke} & \textit{kiyo} & \textit{iki}\\ 
        \lspbottomrule
    \end{tabularx}
\end{table}

\begin{table}
    \caption{Emphatic pronouns}
    \label{tab:emphpronouns}
    \begin{tabularx}{.46\textwidth}{lXX}
        \lsptoprule
        & \textsc{singular}& \textsc{plural}\\
        \midrule
        1 & \textit{ko ike} & \textit{nuko ike} \\
        2 & \textit{de eli} & \textit{ine eli}  \\
        3 & \textit{ki=ro} $\sim$ \textit{ke=ro} & \textit{iki=ro}\\
        \lspbottomrule
    \end{tabularx}
\end{table}

\begin{table}
    \caption{Possessive pronouns}
    \label{tab:posspronouns}
    \begin{tabularx}{.68\textwidth}{lXXX}
        \lsptoprule
        & \textsc{singular}& \textsc{dual}& \textsc{plural}\\
        \midrule
        1 \textsc{incl} & \textit{ko-pi} & \textit{dedukoyo-pi} & \textit{nuko-pi} \\
        1 \textsc{excl} & & \textit{koyo-pi} & \textit{uke-pi}\\
        2 & \textit{de-pi} & \textit{deyo-pi} & \textit{ine-pi}\\
        3 & \textit{kep} (\textit{ki-pi}) & \textit{kiyo-pi} & \textit{iki-pi}\\
        \lspbottomrule
    \end{tabularx}
\end{table}

Note that the dual inclusive form \textit{dedukoyo} is often substituted by the plural inclusive form \textit{nuko}. In texts, we frequently find phrases like \textit{nuko i-le} [we.\textsc{incl} \textsc{du}.\textsc{s}-go] `we (two) go'. In contrast, the dual exclusive form is commonly used: \textit{koyo i-le} [we.\textsc{du}.\textsc{excl} \textsc{du}.\textsc{s}-go] `we two go'.

\section{Verbal TAM morphology}\label{sec:tam}

\begin{table}[b]
    \caption{Tense categories}
    \label{tab:tense}
    \begin{tabularx}{\textwidth}{p{3cm}lX}
    \lsptoprule
        \textsc{affix} & \textsc{gloss} & \textsc{function}\\
    \midrule
        bare verb & n/a & present, continuous present, immediate future\\
        \textit{-p} & -\textsc{pc} & continuous past: ongoing action in the past\\
        vowel backshift or lowering (+elision) & .\textsc{pp} & punctual past: bounded action in the past\\
        \textit{-ko} & -\textsc{rts} & relative tense in the past\\
        \textit{-ipe} & -\textsc{ant} & anteriority in discourse\\
    \lspbottomrule
    \end{tabularx}
\end{table}



TAM morphology and semantics in Kilmeri is quite extensive with 26 categories encoded by a variety of marking patterns. There are only few categories of pure tense (\tabref{tab:tense}), but there are a large number of aspect categories (\tabref{tab:aspect}) and especially modality categories. The supercategory of modality can be divided into epistemic modality (\tabref{tab:epistemic}) and deontic modality (\tabref{tab:deontic}). Epistemic modality is speaker-based, while the deontic modality is anchored in circumstances of the state of affairs. I refer the reader to the texts in this collection in which these categories occur in their most natural context. In (\cite[248-321]{Gerstner-Link:2018un}), I provide a detailed discussion of all TAM related forms and meanings. For explanatory purposes, I present a minimal outline of the TAM system in the following tables.

\begin{table}
    \caption{Aspect categories}
    \label{tab:aspect}
    \begin{tabularx}{\textwidth}{llX} 
    \lsptoprule
        \textsc{affix} & \textsc{gloss} & \textsc{function}\\
    \midrule
        \textit{-uli} & -\textsc{prog} & Progressive and habituative: an action is ongoing for quite some time or is somebody's habit\\
        \textit{-nake} & -\textsc{dur} & Durative: an action continues for some time\\
        \textit{mi-} & \textsc{iter}- & Iterative: an action is repeated several or many times\\
        \textit{-ke} & -\textsc{ingr} & Ingressive\\
        \textit{-or} & -\textsc{con} & Conative: an action is attempted\\
        \textit{-ou} & -\textsc{frus} &  Frustrative: an action could not be done successfully\\
        \textit{-we} & -\textsc{ter} & Terminative: an action is done vigourously\\
        \textit{-wole} & -\textsc{cpl} & Completive: an action is completed\\
    \lspbottomrule
    \end{tabularx}
\end{table}

The possibility suffix \textit{-m} occurs in five more categories as one element of marking (Tables \ref{tab:epistemic} and \ref{tab:deontic} below). The six modal categories in which the possibility suffix \textit{-m} occurs have in common that the state of affairs cannot become factual for different types of causes.

\begin{table}
    \caption{Epistemic modality}
    \label{tab:epistemic}
    \begin{tabularx}{\textwidth}{llX} 
    \lsptoprule
        \textsc{affix} & \textsc{gloss} & \textsc{function}\\
    \midrule
        \textit{ba-}verb\textit{-ko} & \textsc{fac}-verb-\textsc{fac} & Factuality based on perception\\
        \textit{u-} & \textsc{dfac}- & Deictic factuality: visual perception\\
        \textit{dV-} & \textsc{lkh}- & Likelihood: will probably happen or has probably happened\\
        \textit{-m} & -\textsc{pos} & Possibility: may happen in the future\\
        \textit{asa} verb\textit{-m} & how verb-\textsc{pos} & Impossibility: cannot happen\\
        \textit{ba} verb\textit{-we-m} & other verb-\textsc{ter}-\textsc{pos} & Supinative: combination of terminative and possibility under the scope of emphatic negation. It denotes sb's negative disposition.\\
    \lspbottomrule
    \end{tabularx}
\end{table}

\begin{table}
    \caption{Deontic modality}
    \label{tab:deontic}
    \begin{tabularx}{\textwidth}{llX} 
    \lsptoprule
        \textsc{affix} & \textsc{gloss} & \textsc{function}\\
    \midrule
        \textit{-p}, \textit{-yep} (\textsc{pl}) & -\textsc{imp} & Imperative of second person: an order towards the hearer\\
        \textit{a-} & \textsc{imp3}- & Imperative of third person: somebody may or should do something\\
        \textit{muli} & `want, say' & Volition: somebody wants something\\
        \textit{kra-} & \textsc{niv}- & Non-intervention: a state of affairs should continue\\
        \textit{kV-}verb\textit{-m} & \textsc{proh-}verb-\textsc{proh} & Prohibitive: something must not be done by the hearer\\
        \textit{boka-}verb\textit{-m} & \textsc{obs-}verb-\textsc{obs} & Obstructive: something impedes an action\\
        \textit{mona-}verb\textit{-m} & \textsc{irr}-verb-\textsc{irr} & Irrealis: counterfactual events in the past or hypothetical events in the future \\
    \lspbottomrule
    \end{tabularx}
\end{table} 

\newpage
\section{Grammatical relations}\label{sec:gramrel}

In Kilmeri, grammatical relations are encoded exclusively in the verb. There are no core cases for subjects and objects. Thus, the verb alone bears the load of marking the syntactic functions and the semantic roles of agent, patient, and recipient. This grammatical task is divided between the marking of number and the marking of person, which are (to a large extent) independent of each other.

\subsection{The coding of number}

In Kilmeri, number marking on the verb distinguishes singular, dual, and plural. The plain verb form denotes a singular referent of the verbal action. \tabref{tab:numberaff} presents the affixes marking duality and plurality. The suffixes \textit{-wepi} and \textit{-mapi} probably reflect a former serial verb construction. Their punctual past forms are regular, as shown in \tabref{tab:tense}.

\begin{table}
    \caption{Number related affixes}
    \label{tab:numberaff}
    \begin{tabularx}{\textwidth}{llX}
    \lsptoprule
        \textsc{affix} & \textsc{gloss} & \textsc{function}\\
    \midrule
        \textit{i-/-i} & \textsc{du}.\textsc{s}/\textsc{a} & dual subject/agent\\
        \textit{-we} & \textsc{du}.\textsc{s}/\textsc{o} & dual subject/patient object;  uncontrolled patient subject\\
        \textit{wo-} & \textsc{accom} & accompaniment, denotes dual or paucal subject\\
        \textit{-wepi} & \textsc{quant}.\textsc{s}/\textsc{o} & plurality of subject or patient object\\
        \textit{-mapi} & \textsc{quant}.\textsc{e} & event plurality\\
    \lspbottomrule
    \end{tabularx}
\end{table}

Examples (\ref{ex:intro-32}) and (\ref{ex:intro-33}) illustrate singular verb forms without any number marking. The participant person is coded by a pronoun that is obligatory in such sentences.

\newpage
\ea\label{ex:intro-32}
    em ko seleyo le\\
    \gll em ko sele-yo le\\
    tomorrow I garden-\textsc{loc} go\\
    \glt `Tomorrow I will go to the garden.' \exsource{overheard conversation}
\z

\ea\label{ex:intro-33}
    de aryo le\\
    \gll de aryo le\\
    you where go\\
    \glt `Where are you going?' \exsource{overheard conversation}
\z

The dual verb form denotes two participants in the verbal action. The dual of S and A is marked by the affix \textit{i-/-i}. The affix occurs as a prefix with the verbs that have a suppletive plural form for S, as in (\ref{ex:intro-11}). It occurs as a suffix (\ref{ex:intro-12}) for verbs that have a suppletive pural form for O, and for all other verbs (Cf. \cite[341, 347]{Gerstner-Link:2018un}, Table 7.2 there also shows the exceptions).

\ea\label{ex:intro-11}
    dedukoyo seleyo ile\\
    \gll dedukoyo sele-yo i-le\\
    we.\textsc{du}.\textsc{incl} garden-\textsc{loc} \textsc{du}.\textsc{s}-go\\
    \glt `We two go to the garden.' \exsource{overheard conversation}
\z

\ea\label{ex:intro-12}
    dedukoyo pewo yasiyei\\
    \gll dedukoyo pewo yasiye-i\\
    we.\textsc{du}.\textsc{incl} banana plant-\textsc{du}.\textsc{a}\\
    \glt `We two will plant bananas.' \exsource{overheard conversation}
\z

The dual of O is marked by the suffix \textit{-we} (\ref{ex:intro-13}). This suffix is also used with verbs whose subject has the feature [-\textsc{contr}] (\cite[329-330]{Gerstner-Link:2018un}), as in (\ref{ex:intro-14}).

\ea\label{ex:intro-13}
    bepu ko nîsî dupua wepulowe\\
    \gll bepu ko nîsî dupua wepulo-we\\
    sago.grub I string two bring.\textsc{pp}-\textsc{du}.\textsc{o}\\
    \glt `I brought two strings of sago grubs.' \exsource{V,83}\footnote{These codes refer to my field notebooks.}
\z

\ea\label{ex:intro-14}
    epe aino kopi mariwe\\
    \gll epe ai-no ko-pi mari-we\\
    mother father-\textsc{ins} 1\textsc{sg}-\textsc{poss} be.sick-\textsc{du}.\textsc{s}\\
    \glt `My parents are sick.' \exsource{V,180}
\z

The marking of plural number in Kilmeri is diverse. Many verbs come with a suppletive plural form denoting the plurality of S (intransitive verbs), as in (\ref{ex:intro-15}) and (\ref{ex:intro-16}), or O (transitive verbs), as in (\ref{ex:intro-17}). A few verbs mark plurality of A by suppletive forms. In addition, there is the suffix \textit{-wepi} that likewise marks the plurality of S and O, but never the plurality of A. Instead of already given duality and plurality, dual and plural forms of verbs can also denote incremental number.

Plurality coded by suppletive forms is distributive and participant related, but there are cases of action/event relatedness. Plurality coded by the suffix \textit{-wepi} is distributive or cumulative, and it can be participant or action/event related (cf. \cite[363]{Gerstner-Link:2018un}) The coding devices of suppletive plural and \textit{-wepi} can also be combined. 

\ea\label{ex:intro-15}
    uke kumune seleyo mole\\
    \gll uke kumune sele-yo mole\\
    we.\textsc{excl} all.\textsc{coll} garden-\textsc{loc} go.\textsc{pl}\\
    \glt `We all go to the garden.' \exsource{overheard conversation}
\z

\ea\label{ex:intro-16}
    dor anno kopi ripepi\\
    \gll dor an-no ko-pi ripepi\\
    foot hand-\textsc{ins} 1\textsc{sg}-\textsc{poss} be.numb.\textsc{pl}\\
    \glt `My feet and hands are numb.' \exsource{V,144}
\z

\ea\label{ex:intro-17}
    ko ul kiniyo kale\\
    \gll ko ul kiniyo kale\\
    I bamboo all lay.horizontally.\textsc{pl}.\textsc{o}\\
    \glt `I lay all the bamboo rods in parallel.' \exsource{VII,91}
\z

In example (\ref{ex:intro-18}) we see a combination of plural and dual marking: The plural verb form encodes the plurality of the patient object via suppletion, but also bears the dual agent suffix.

\ea\label{ex:intro-18}
    deyo ri yipyo melinaip\\
    \gll deyo ri yip-yo melina-i-p\\
    you.\textsc{du}  wood  house-\textsc{loc}  carry.inside.\textsc{pl}.\textsc{o}-\textsc{du}.\textsc{a}-\textsc{imp}\\
    \glt `You two, carry all the firewood inside the (kitchen) house!' \exsource{IV,79}
\z

The following examples illustrate the suffix \textit{-wepi}. Example (24) combines a suppletive plural with \textit{-wepi}.

\ea\label{ex:intro-19}
    pper kiniyo bamonwepko\\
    \gll pper kiniyo ba-mini-wepi-ko\\
    gourd many \textsc{fac}-come.hither-\textsc{quant}.\textsc{s}-\textsc{fac}\\
    \glt `Many gourds have come up (for harvesting).' \exsource{Text \ref{sc0501} Sequence \ref{ex:nrexsikedo}}
\z

\ea\label{ex:intro-20}
    ko dop sipiwepi\\
    \gll ko dop sipi-wepi\\
    I body hurt-\textsc{quant}.\textsc{s}\\
    \glt `My body hurts all over.' \exsource{V,106}
\z

\ea\label{ex:intro-21}
    Buoko buar wepulo yena kiniyo leliewepu\\
    \gll Buoko buar wepulo yena kiniyo lelie-wepu\\
    Buoko axe bring.\textsc{pp} people all kill-\textsc{quant}.\textsc{o}.\textsc{pp}\\
    \glt `Buoko took an axe and killed all the people.' \exsource{Text \ref{sc0206} Sequence \ref{ex:nrexinuges}}
\z


\ea\label{ex:intro-22}
    uki kopi ri puk lil lupapwepu palouyo\\
    \gll uki ko-pi ri\_puk lil lu\_papi-wepu palou-yo\\
    husband 1\textsc{sg}-\textsc{poss} kind.of.tree blood incise\_do.\textsc{pl}.\textsc{o}-\textsc{quant}.\textsc{o}.\textsc{pp} spear-\textsc{loc}\\
    \glt `My husband filled the engravings of the spear with coloured sap of the \textit{puk}-tree.' \exsource{VII,148}
\z

Example (\ref{ex:intro-22a}) illustrates event plurality in combination with the suppletive plural of the patient object.

\ea\label{ex:intro-22a}
    ko solo pusapimapi\\
    \gll ko solo pusapi-mapi\\
    I only wash.\textsc{pl}.\textsc{o}-\textsc{quant}.\textsc{e}\\
    \glt `It is me alone who does the dishes all the time.' \exsource{VI,84}
\z

\subsection{The coding of person}

In Kilmeri, the coding of person is limited to a small class of verbs. Semantically, these verbs encode the role of Recipient. As we saw above, the coding of Patients involves number, but not person. Kilmeri has two- and three-place verbs that require agreement with their Recipient objects, as shown in \tabref{tab:verbperson}.

\begin{table}
    \caption{Person affixes}
    \label{tab:personaff}
    \begin{tabularx}{\textwidth}{Xll}
    \lsptoprule
        \textsc{affix} & \textsc{gloss} & \textsc{function}\\
    \midrule
        \textit{-ipi} & 1\textsc{sg}.\textsc{or} & first person singular recipient object\\
        \textit{-me} & 2\textsc{sg}.\textsc{or} & second person singular recipient object\\
        \textit{-ne} & 3\textsc{sg}.\textsc{or} & third person singular recipient object\\
        \textit{-no} & 3\textsc{sg}.\textsc{or}.\textsc{pp} & third person singular recipient object punctual past\\
        \textit{-ini} & \textsc{nsg}.\textsc{or} & non-singular recipient object\\
        \textit{-en} & \textsc{nsg}.\textsc{or}.\textsc{pp} & non-singular recipient object punctual past\\
    \lspbottomrule
    \end{tabularx}
\end{table}

\begin{table}
    \caption{Verbs with obligatory person indexing}
    \label{tab:verbperson}
    \begin{tabularx}{\textwidth}{lX}
    \lsptoprule
        \textsc{verb} & \textsc{meaning}\\
    \midrule
        \textsc{transitive}&\\
    \midrule
        \textit{mueli-}&`tell sb; talk to sb' (opposed to   \textit{mui}.\textsc{sg}/\textit{moliye}.\textsc{pl}  `say, speak' without person agreement)\\
        \textit{sai-}&`ask sb'\\
        \textit{pele-}&`gossip with sb'\\
        \textit{woni-}&`call sb; call out for sb'\\
        \textit{wui-}&`answer sb'\\
        \textit{lewo-}&`wait for sb'\\
        \textit{wuli-}&`follow sb'\\
    \tablevspace
        \textsc{ditransitive}&\\
    \midrule
        \textit{nie-}&`show sth to sb'\\
        \textit{mosupi-}&`show sth to sb' (involving a change of location as the thing to be shown is at a different place)\\
        \textit{mosaupi-}&`teach sth to sb'\\
        \textit{powa-}/\textit{pona-}&`give sth to sb'\\
        \textit{ripei-}&`share (cooked) food with sb' (opposed to   \textit{ripei}.\textsc{sg}/\textit{rupopi}.\textsc{pl} `distribute (food)' without person agreement)\\
        \textit{supoye-}&`exchange women for marriage with sb'\\
    \lspbottomrule
    \end{tabularx}
\end{table}

As \tabref{tab:personaff} shows, there are distinctive person suffixes only in the singular. Dual and plural have one suffix which agrees with the Recipient argument. This agreement pattern is evidence for the person encoding paradigm even if the person distinction is neutralised for non-singular number values. The paradigm may reflect the old quaternal pronoun system of the Border languages which is still present in Imonda with first, second, third, and inclusive person (\cite[44]{Seiler:1985vo}).

The verb \textit{pona-} `give' exhibits an irregular paradigm of person agreement, as in (\ref{ex:intro-27}-\ref{ex:intro-29}). For more information, I refer the reader to \textcite[386-394; 402-405]{Gerstner-Link:2018un}.
 
The following examples (\ref{ex:intro-23}-\ref{ex:intro-29}) illustrate some of the verbs requiring obligatory person agreement.

\ea\label{ex:intro-23}
    de ko lewoipep\\
    \gll de ko lewo-ipi-p\\
    you I wait.for-1\textsc{sg}.\textsc{or}-\textsc{imp}\\
    \glt `Wait for me!' \exsource{overheard conversation}
\z

\ea\label{ex:intro-24}
    ko de wulime\\
    \gll ko de wuli-me\\
    I you follow-2\textsc{sg}.\textsc{or}\\
    \glt `I'll follow you.' \exsource{overheard conversation}
\z

\ea\label{ex:intro-25}
    ko Eva muelno de inalap\\
    \gll ko Eva mueli-no de ina\_le-p\\
    I Eva talk.to-3\textsc{sg}.\textsc{or}.\textsc{pp} you hurry\_go-\textsc{imp}\\
    \glt `I said to Eva: Hurry up!' \exsource{overheard conversation}
\z

\ea\label{ex:intro-26}
    disaipel molo yena kiniyo mosaupoen\\
    \gll disaipel molo yena kiniyo mosaupo-en\\
    disciples(\textsc{tp}) go.\textsc{pl}.\textsc{pp} people many teach-\textsc{nsg}.\textsc{or}.\textsc{pp}\\
    \glt `The disciples went and taught many people.' \exsource{Bible translation: Mark 6,12}
\z

\ea\label{ex:intro-27}
    de luo ko ar powa\\
    \gll de luo ko ar powa\\
    you money I \textsc{neg} give.1\textsc{sg}.\textsc{or}.\textsc{pp}\\
    \glt `You didn't give me any money.' \exsource{Text \ref{sc0605} Sequence \ref{ex:nrexlipkau}}
\z

\ea\label{ex:intro-28}
    melon de ana poname\\
    \gll melon de ana poname\\
    melon(\textsc{tp}) you who give.3\textsc{sg}.\textsc{or}\\
    \glt `To whom will you give the melon?' \exsource{overheard conversation}
\z

\ea\label{ex:intro-29}
    ko wal dû yûr su roise yano ponamo.\\
    \gll ko wal dû yûr su roise ya-no ponamo\\
    I fish flesh chicken egg with sago-\textsc{ins} give.3\textsc{sg}.\textsc{or}.\textsc{pp}\\
    \glt `I gave her some fish and eggs with sago.' \exsource{Text \ref{sc0711} Sequence \ref{ex:t3exnikowo}}
\z

The examples above illustrate agreement with only one argument, namely the ``dative''-object in the role of the recipient. This is due to the fact that subjects are not marked on the verb except if (i) they are dual or (ii) the verb is intransitive and belongs to the class of suppletive plural verbs. The following examples illustrate subject and ``dative''-object agreement. Example (\ref{ex:intro-36}) shows a two-place verb with subject and ``dative''-object agreement. Example (\ref{ex:intro-37}) shows applicative constructions; the verb form \textit{melinoi} illustrates a very rare instance of indexing three arguments via suppletive plural for the direct object (Theme), the applicative for third person singular and subject dual agreement.

\ea\label{ex:intro-36}
    koyo de wulimei.\\
    \gll koyo		de		wuli-me-i\\
	we.\textsc{du}.\textsc{excl} you follow-2\textsc{sg}.\textsc{or}-\textsc{du}.\textsc{a}\\
	\glt `We two follow you.' \exsource{overheard conversation}
\z

\ea\label{ex:intro-37}
    Leno kredimponoipno, le rapiyenoi, melnoi.\\
    \gll le-no k-redim(\textsc{tp})-pi-ne-i-p-no le rapiye-ne-i meli-no-i\\
	things-\textsc{ins} \textsc{sub}-make.ready-\textsc{lv}-3\textsc{sg}.\textsc{or}-\textsc{du}.\textsc{a}-\textsc{pc}-\textsc{co} things fetch-3\textsc{sg}.\textsc{or}-\textsc{du}.\textsc{a} carry.\textsc{pl}.\textsc{o}-3\textsc{sg}.\textsc{or}-\textsc{du}.\textsc{a}\\
	\glt `After the two of them had made ready (Sakou's) things, they fetched them and carried everything for him.' \exsource{Text \ref{sc0203} Sequence \ref{ex:nrexlenokr}}
\z

Regarding the slots of agreement affixes, we see that the ``dative''-object suffix precedes the subject dual affix.

\subsection{The coding of animacy}

There are two verbs that encode animacy: (i) \textit{riye} `see sth' (\ref{ex:intro-30}) vs. \textit{reye} `see sb' (\ref{ex:intro-31}), and \textit{rili} `see several persons'; (ii) \textit{reyane} `meet sb' and \textit{relane} `meet several persons'. The animate form is also used for higher animals. In fact, the verb \textit{riye/reye/rili} combines animacy with suppletive number. 

\ea\label{ex:intro-30}
    dob de riyoworo\\
    \gll dob de riye-we=ro\\
    eye you see.\textsc{o}.\textsc{inanim}-\textsc{ter}=\textsc{emph}\\
    \glt `You watch out attentively!' \exsource{Text \ref{sc0305} Sequence \ref{ex:t2exdeqedu}}
\z

\ea\label{ex:intro-31}
    ono bekulu pu ipiyo pin dob ko reyo\\
    \gll ono bekulu pu ipi-yo pin dob ko reyo\\
    man huge water bottom-\textsc{loc} come.up.hither.\textsc{pp} eye I see.\textsc{o}.\textsc{anim}.\textsc{sg}.\textsc{pp}\\
    \glt `A huge man came up here from the bottom of the water, I saw him.' \exsource{Text \ref{sc0305} Sequence \ref{ex:t2exdupuaf}}
\z

\section{Voice related constructions}\label{sec:voiceverb}

Kilmeri has three types of constructions that can be broadly described as voice-related. They express reciprocity, malefactive source, and applicativity. There is no active-passive distinction in Kilmeri, i.e., the language follows the usual pattern of Papuan languages.

Reciprocity is marked by the suffix \textit{-paye}, as in (\ref{ex:intro-38}) and (\ref{ex:intro-39}). This suffix shows several assimilation patterns (\cite[422; 527-531]{Gerstner-Link:2018un}).

\ea\label{ex:intro-38}
    uke bono pokapayo.\\
    \gll uke bo-no poka-payo\\
    we.\textsc{excl}	word-\textsc{ins}	scold-\textsc{recp}.\textsc{pp}\\
    \glt `We scolded one another with words.' \exsource{V,96}
\z

\ea\label{ex:intro-39}
    iki	wako mekiyayep.\\
    \gll iki wako meki-yaye-p\\
    \textsc{aph}.\textsc{pl} amongst help-\textsc{recp}-\textsc{pc}\\
    \glt `They were helping one another.' \exsource{VI,122}
\z

The malefactive source of an action is indicated by the suffix \textit{-maye} (\cite[409-410]{Gerstner-Link:2018un}). Malefactive constructions are always transitive. The participant against whom the action is directed may or may not occur in the clause, as in (\ref{ex:intro-40}) and (\ref{ex:intro-41}) respectively.

\newpage
\ea\label{ex:intro-40}
    sukupu dop sei sowemayo.\\
    \gll sukupu dop sei sowe-mayo\\
    bush.spirit body white hide-\textsc{mal}.\textsc{pp}\\
    \glt `The bush spirits hide from the white man.' \exsource{Text \ref{sc0302} Sequence \ref{ex:nrexsukrur}}
\z

\ea\label{ex:intro-41}
    de smep musimayo upunaro.\\
    \gll de smep musi-mayo upuna=ro\\
    you door lock-\textsc{mal}.\textsc{pp} good=\textsc{emph}\\
    \glt `You locked the door, very good [to prevent thieves from entering the house].' \exsource{overheard conversation}
\z

Applicative constructions increase the number of participants of a verb. They can be used with intransitive and transitive verbs. The applicative suffixes are identical with the person marking suffixes. Most often applicatives occur with the third person, but they also can occur with first and second person (\cite[405-409; 445-451]{Gerstner-Link:2018un}).

\ea\label{ex:intro-42}
    de kaeli kopoipem.\\
    \gll de kaeli k-pi-ipi-m\\
    you strong \textsc{proh}-do-1\textsc{sg}.\textsc{or}-\textsc{proh}\\
    \glt `For my sake, don’t be stubborn towards me!' \exsource{overheard conversattion}
\z

\textref{sc0711} \textit{Bo Milipiro} `Mili's sickness' illustrates abundant use of applicativity as all the actions of the hospital staff are directed to the sick girl Mili, i.e., the verb inflections show third person singular agreement. 

\section{Serial verb constructions}\label{sec:serialverbs}

Serial verb constructions are a major syntactic device in Kilmeri. In principle, verbs can become a component verb of a serial verb construction without any semantic constraints. For example, there are many lexicalised serial verb constructions whose meaning emerges from the idiosyncracies of two single verbs. Serial verbs usually consist of two verbs, but can also consist of three verbs. Suppletive plural forms can also occur in serial verb constructions (see \textit{moliye} in Table \ref{tab:serialverbs}).

\tabref{tab:serialverbs} gives an overview of the standardised verbs that have a predictable meaning in serial verb constructions. Note that there is also stative serialisation in which positional verbs (and also the verb \textit{nui} `sleep') can occur as the first element. Moreover, there is lexicalised serialisation, in which some serial verbs constructions have fused together to form verbal compounds.

\begin{table}
    \caption{Standardised serial verb formations in Kilmeri}
    \label{tab:serialverbs}
    \begin{tabularx}{\textwidth}{llX} 
    \lsptoprule
        \textsc{verb} & \textsc{translation} & \textsc{function} \\
    \midrule
        \_\textit{nake}\textsuperscript{a} & `sit' & aspectual: durative\\
        \_\textit{wole} & `move further' & aspectual: completive\\
        \_\textit{paye} & `leave behind' & reciprocal\\
        \_\textit{kûne} & `go down' & directional: `downwards'\\
        \_\textit{pake} & `throw' & directional: `downwards'\\
        \_\textit{pepe} & `put on top' & directional: `on top of'\\
        \_\textit{pane} & `do thither' & directional: `away from'\\
        \_\textit{mini} & `come here' & directional: `towards'\\
        \_\textit{ppue} & `go up' & directional: `upward'\\
        \_\textit{pini} & `come up here' & directional: `up to here'\\
        \textit{dori}\_ & `turn back' & spatial: `back'\\
        \textit{buri}\_ & `go ahead' & spatial: `ahead'\\
        \textit{ina}\_ & `hurry' & modal: `quickly'\\
        \_\textit{maeu}\_\textsuperscript{b} & `belong to' & possession, e.g. `as mine'\\
        \textit{moliye}.\textsc{pl}\_ & `several speak' & speaking while moving\\
        \_\textit{pue} & `stroll, walk around' & hetero-kinetic motion verbs and \textit{riye} `see'\\
        \_\textit{laye} & `put, lay' & spatial: two-dimensional expansion\\
        \_\textit{piye} & `take' & reinforcing verb meaning\\
    \lspbottomrule
    \multicolumn{3}{l}{\footnotesize{\textsuperscript{a} The underscore (\_) shows the position of the non-standardised verb}}\\
    \multicolumn{3}{l}{\footnotesize{\textsuperscript{b} This verb may precede or follow the non-standardised verb}}\\
    \end{tabularx}
\end{table}

\section{Complex sentences}\label{sec:complex}

In Kilmeri, complex sentences are often construed without any grammatical marking, that is, via juxtaposition of clauses or inflected verbs. Sequentiality of events is indicated by a complex affixal structure on the verb: \textit{k-verb-p-no} [\textsc{sub}-verb-\textsc{pc}-\textsc{co}] `having VERB', e.g. \textit{k-pule-p-no} `having come'. Sequentiality marking is frequently used to achieve  coherence; which often results in tail-head-constructions. We find simultaneity much less often, which is marked by \textit{verb-no} [verb\textsc{-co}] `while x-ing'. In addition to these temporal relationships between the sentences, Kilmeri can also express the purpose of an action, in which case the verb is marked with the suffix \textit{-na}, as in (\ref{ex-intro-34}). A frequently used coordinating conjunction is \textit{riyopuno} `then', which appears at the beginning of the sentence.

\ea\label{ex-intro-34}
    ko pe pakono piyeke bisa luina\\
    \gll ko pe pako-no piye-ke bisa lui-na\\
    I arrow bow-\textsc{ins} take-\textsc{ingr} rat shoot-\textsc{purp}\\
    \glt `I'll go get bow and arrows to kill the rats.' \exsource{IV,125}
\z

There is no means to mark indirect speech. Instead, the speech of discourse participants is cited and embedded after the verb \textit{mueli-} `talk', as in (\ref{ex:intro-35}). Such embeddings can also be recursive (\cite[487]{Gerstner-Link:2018un}).

\ea\label{ex:intro-35}
    ai muelne ``ko duyo le''\\
    \gll ai muel-ne ko du-yo le\\
    father talk.to-3\textsc{sg}.\textsc{or} I bush-\textsc{loc} go\\
    \glt `The father says to him: ``I will go to the bush.''' \exsource{Text \ref{sc0305}, Sequence \ref{ex:t2exriyokn}}
\z

\nocite{Bolton:2010lc} \nocite{Beier:1974xu} \nocite{Mercer:1997rv}
