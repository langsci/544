\addchap{\lsPrefaceTitle}

The present collection of Kilmeri texts arose in parallel with the work on my Kilmeri grammar (\cite{Gerstner-Link:2018un}). From the very beginning of the fieldwork, texts were as important for me as grammatical issues of the language; being a researcher who holds a holistic view on language it seemed inconceivable for me to produce a grammar which wouldn't be based on a broad collection of running text. Special grammatical details can be -- and must be -- obtained by elicitation, but the overall character of a language manifests itself in texts of different genres. Unfortunately, grammars usually include only a few texts in an appendix, while most texts of the language in question are not publicly accessible. My own grammar also follows this practice; and, what is more, the glossed texts are somehow ``hidden'' in the Online Supplement of the grammar. Therefore the OTC project of the \textit{Berlin-Brandenburgische Akademie der Wissenschaften} is a most welcome opportunity to finally publish all the Kilmeri texts that I could collect during my fieldwork. 

In addition to texts that are related to Kilmeri heritage, my consultants and I worked on a partial translation of the gospel of Mark. They were highly engaged in this endeavour, which would honor their language in that it would prove their ``bush'' vernacular to be a language on the same level as English. The translation would show the ability of Kilmeri to put into words and express the texts of the Bible. 

Regrettably, the present collection does not include those texts due to the constraint the publishers needed to observe to publish only \textit{original} texts of a language community. It is true that translated texts don't belong to the literary and cultural heritage of a people, but attest influence from outside. They certainly bring a foreign note to the inner coherence of narrative topics of an indigenous community. However, there is always an unavoidable break of tradition as soon as outsiders -- be it missionaries, traders, hunters, or linguists -- enter a community. For instance, when my consultant Susan Bisam traveled with her husband, a crocodile hunter from the Sepik, to the area south of the Bewani mountains, she acquired some knowledge of the culture there and brought it to the Kilmeri area. These traits can be seen in one of her stories that she told in Tok Pisin. With the arrival of ``modern life'' a new genre of text emerges that can be called ``narratives situated in contemporary time and life.'' Texts of this type form one chapter of the present collection, and they certainly widen the narrative ability of their narrators and show their eagerness to tell the stories in question. 

My fieldwork on Kilmeri (Border language family, northwestern Papua New Guinea) spread over seven years starting in November 1999 and ending in February 2007. During this time I spent about 15 months in the field. The first preliminary version of the glossed texts was finished in 2010. Subsequently, glosses and translations were continuously adapted and adjusted in parallel with the writing of the grammar. For this publication, all texts have been reviewed and, where necessary, revised in order to integrate a different perspective on the best translation of sentences or phrases. I am also finalising my Kilmeri dictionary (to be published in the ``Dictionaria'' series), which brings to light meanings of words that I was previously unaware of; such new findings have also been taken into account. The collection contains the six texts of the online supplement to my Kilmeri grammar. These texts were also revised and, in some cases, expanded when an (unintented) gap was found in the original transcription.