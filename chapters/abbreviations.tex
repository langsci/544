\addchap{\lsAbbreviationsTitle}
% \addchap{Abbreviations and symbols}
%The category labels for abbreviations follow the Leipzig Glossing Rules.\footnote{\url{http://www.eva.mpg.de/lingua/resources/glossing-rules.php}}
%\vspace{.5cm}

The glossing conventions follow the Leipzig Glossing Rules.\footnote{Cf. LGR revised version of February 2008: \url{http://www.eva.mpg.de/lingua/resources/glossing-rules.php}}  Categorial abbreviations not found in the Leipzig Glossing Rules are my own. For explanations of categories that may seem rather idiosyncratic, I refer the reader to my grammar of Kilmeri (\cite{Gerstner-Link:2018un}).

Circumfixes are written as \textsc{aff}- … -\textsc{aff} in the gloss line. Multi-word lexemes in Kilmeri, such as serial verbs or collocations, are written with an underscore on the morpheme line when their components need to be spelled out. Serial verbs are indicated by the notation V\_V(\_V). Collocations are indicated by the notation N\_N or N\_V. Multi-word translations in the gloss line are separated by full stops.

All pronouns receive a lexical gloss if they occur by themselves, e.g. \textit{ko} `I' and \textit{de} `you.\textsc{sg}'. If they are affixed, they are glossed with an abbreviation (1\textsc{sg} and 2\textsc{sg}, respectively), as in \textit{yip ko-pi-yo} [house 1\textsc{sg}-\textsc{poss}-\textsc{loc}] `at my house'. 

Tok Pisin insertions are marked with (\textsc{tp}) after the respective item in the gloss line. Tok Pisin words that are loanwords, however, are not marked in this way, as they are part of the contemporary Kilmeri lexicon.

Finally, translations of lexical items are glossed according to context, i.e. for some Kilmeri words the lexical gloss may differ because the Kilmeri word has several meanings.

All abbreviations used in this text collection are given below:

\vspace{.4cm}

\begin{tabularx}{.45\textwidth}{lQ}
    | & syncretism (e.g. \textsc{a}|\textsc{s})\\
    \textsc{1} & first person\\
    \textsc{2} & second person\\
    \textsc{3} & third person\\
    \textsc{a} & agent/transitive subject\\
    \textsc{accom} & accompaniment\\
    \textsc{adj} & adjective\\
    \textsc{adv} & adverbial\\
    \textsc{aff} & affinitative\\
\end{tabularx}
\begin{tabularx}{.45\textwidth}{lQ}
    \textsc{anim} & animate\\
    \textsc{ant} & anterior\\
    \textsc{aph} & anaphor\\
    \textsc{aug} & augmentative\\
    \textsc{co} & connective\\
    \textsc{coll} & collective\\
    \textsc{con} & conative\\
    \textsc{cpl} & completive\\
    \textsc{deic} & deictic\\
\end{tabularx}

\begin{tabularx}{.45\textwidth}{lQ}
    \textsc{det} & determiner\\
    \textsc{dfac} & deictic-factual\\
    \textsc{dist} & distal\\
    \textsc{du} & dual\\
    \textsc{dur} & durative\\
    \textsc{emph} & emphatic\\
    \textsc{excl} & exclusive\\
	\textsc{fac} & resultative-factual\\
    \textsc{frus} & frustrative\\
    \textsc{imp} & imperative 2nd person\\
    \textsc{imp3} & imperative 3rd person\\
    \textsc{inanim} & inanimate\\
    \textsc{incl} & inclusive\\
    \textsc{ingr} & ingressive\\
    \textsc{ins} & instrumental\\
    \textsc{irr} & irrealis\\
    \textsc{iter} & iterative\\
    \textsc{lkh} & likelihood\\
    \textsc{loc} & locative\\
    \textsc{lv} & light verb\\
    \textsc{mal} & malefactive\\
    \textsc{mod} & modal\\
    \textsc{n} & noun\\
    \textsc{neg} & negation\\
    \textsc{niv} & non-interventional\\
    \textsc{np} & noun phrase\\
    \textsc{nsg} & non-singular\\
    \textsc{num} & numeral\\
\end{tabularx}
\begin{tabularx}{.45\textwidth}{lQ}
    \textsc{o} & patient object\\
    \textsc{obs} & obstructive\\
    \textsc{or} & recipient object\\
    \textsc{part} & partitive\\
    \textsc{path} & path\\
    \textsc{pc} & continuous past\\
    \textsc{pl} & plural\\
    \textsc{pos} & possibility\\
    \textsc{poss} & possessive\\
    \textsc{pp} & punctual past\\
    \textsc{prog} & progressive/habituative\\
    \textsc{proh} & prohibitive\\
    \textsc{prox} & proximal\\
    \textsc{purp} & purposive\\
    \textsc{q} & question marker\\
    \textsc{quant} & quantificational suffix\\
    \textsc{qt} & quantifier\\
    \textsc{recip} & reciprocal\\
    \textsc{rts} & relative tense\\
    \textsc{s} & intransitive subject\\
    \textsc{sg} & singular\\
    \textsc{sim} & similative\\
    \textsc{sub} & subordinating\\
    \textsc{ter} & terminative\\
    \textsc{top} & topic\\
    \textsc{tp} & from Tok Pisin\\
    \textsc{voc} & vocative\\
    &\\
\end{tabularx}