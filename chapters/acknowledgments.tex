\addchap{\lsAcknowledgementTitle} 

My first thanks go to the project team of the OTC Project, Sebastian Nordhoff, Christian Döhler and Mandana Seyfeddinipur, for inviting me to contribute to this project, which provides a substantial extension of language documentation beyond grammars and dictionaries. OTC editor in chief is Christian Döhler, and the project is based at the \textit{Berlin}-\textit{Branden\-burgische} \textit{Aka\-demie} \textit{der} \textit{Wissenschaften}. Döhler is also regional editor of the Papuan section of OTC. So I owe greatest thanks to him: for his technical support, for his stylistic suggestions, for his editing ideas, and for generosity and enthusiasm during my work on the manuscript. His OTC volume ``Speaking the map. Komnzo texts'' (\cite{Dohler:2024xw}) was a fine guide for my reflections and comments on the Kilmeri narratives. I also thank two anonymous referees whose careful comments helped to considerably improve the text. Finally, I thank Matthew Korte for his meticulous proofreading.

The preliminary version of the text collection dates back to the year 2010, and here I want to thank Inken Kaumann M.A. for her patient and probing help in compiling the texts. This work took place within my DFG project ``Duale Grammatikographie: semasiologische und onomasiologische Analyse ereignis- und raumbezogener Konstruktionen in Kilmeri''\footnote{GZ: MO728/7-1,2; AOBJ: 529555: September 2006 -- August 2009, cost-neutral extended until October 2010.} under the supervision of Ulrike Mosel.

The narrators to whom I owe the texts are introduced in \chapref{sec:narrators} below. They were always eager to contribute stories belonging to their heritage as well as those of their current life. Many photos supplemented my fieldwork. Some of them are included here, and the narrators would feel honoured by that. But meanwhile, about 20 years after the fieldwork, they have deceased as all of them belonged to the oldest generation. I regard the narrators as my colleagues in entrenched linguistic knowledge. My greatest thanks go to them. For me, they are a model in the wisdom of heritage. 

Last but not least, I deeply thank my husband Godehard Link who accompanied my research and publications on Kilmeri for decades with untiring interest, encouragement, and love. 